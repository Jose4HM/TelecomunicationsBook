\documentclass[11pt,fleqn]{book} % Default font size and left-justified equations

\input{structure.tex} 
%----------------------------------------------------------------------------------------

\begin{document}
%----------------------------------------------------------------------------------------
%	CHAPTER
%----------------------------------------------------------------------------------------

\section{Multiplexor y Demultiplexor}
\subsection{Multiplexor}
Circuito combinacional al que entran varios canales y solo sale uno de ellos.
\begin{center}
\includegraphics[scale=0.8]{SD/SD39.png}
\end{center}
\subsubsection{Multiplexor Simple}
Existen dos posibles simbologías para los multiplexores:
\begin{center}
\includegraphics[scale=0.8]{SD/SD40.png}
\end{center}
\begin{multicols}{2}
\begin{center}
\begin{tabular}{|c|c|c|c|}
\hline
S & $I_1$ & $I_1$ & F \\ \hline
0 & 0     & 0     & 0 \\ \hline
0 & 0     & 1     & 1 \\ \hline
0 & 1     & 0     & 0 \\ \hline
0 & 1     & 1     & 1 \\ \hline
1 & 0     & 0     & 0 \\ \hline
1 & 0     & 1     & 0 \\ \hline
1 & 1     & 0     & 1 \\ \hline
1 & 1     & 1     & 1 \\ \hline
\end{tabular}
\end{center}
\columnbreak
\begin{center}
\begin{karnaugh-map}[4][2][1][$I_1I_0$][$S$]
\minterms{1,3,6,7}
\maxterms{0,2,5,4}
\end{karnaugh-map}
\begin{displaymath}
F=\overline{S}\cdot I_0 + S\cdot I_1
\end{displaymath}
\end{center}
\end{multicols}
\begin{remark}
Es usual que en las tablas de verdad, la entrada que se encuentra a la izquierda es el más significativo(MSB) mientras que el que se encuentra más a la derecha es el menos significativo(LSB).
\end{remark}
Si analizamos la tabla: vemos que la entrada de selección S, poseerá dos estados; si analizamos cuando S=0 vemos que la salida F estará activada siempre y cuando la entrada $I_0$ tenga la entrada activa, mientras que si $I_0$ posee una entrada 0, no importa el valor que posea la entrada $I_1$(alto o bajo) la salida F siempre será 0. Esto nos quiere decir que cuando S=0, la salida F solo le importará los estados de la entrada $I_0$ pero no de $I_1$. Cuando la entrada de selección S=1, notaremos le mismo comportamiento, la salida F solo tomará en cuenta los estados de la entrada $I_1$ puesto que la entrada $I_0$ no nos importa. En conclusión, el la entrada de control S es capaz de seleccionar entre dos entradas.
\subsubsection{Multiplexor 2 entradas selección}
\begin{multicols}{2}
\begin{center}
\includegraphics[scale=0.6]{SD/SD41.png}
\end{center}
\columnbreak
\begin{center}
\begin{tabular}{|c|c|c|}
\hline
$S_1$ & $S_0$ & F     \\ \hline
0     & 0     & $I_0$ \\ \hline
0     & 1     & $I_1$ \\ \hline
1     & 0     & $I_2$ \\ \hline
1     & 1     & $I_3$ \\ \hline
\end{tabular}
\begin{displaymath}
F=\overline{S_1}\cdot\overline{S_0}\cdot I_0+\overline{S_1}\cdot S_0\cdot I_1+S_1\cdot\overline{S_0}\cdot I_2+S_1\cdot S_0\cdot I_3
\end{displaymath}
\end{center}
\end{multicols}
%----------------------------------------------------------------------------------------
%	END CHAPTER
%----------------------------------------------------------------------------------------

\end{document}