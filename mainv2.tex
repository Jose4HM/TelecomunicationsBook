%%%%%%%%%%%%%%%%%%%%%%%%%%%%%%%%%%%%%%%%%
% The Legrand Orange Book
% LaTeX Template
% Version 3.1 (February 18, 2022)
% Compiling this template:
% This template uses biber for its bibliography and makeindex for its index.
% When you first open the template, compile it from the command line with the 
% commands below to make sure your LaTeX distribution is configured correctly:
%
% 1) pdflatex mainv2
% 2) makeindex mainv2.idx -s indexstyle.ist
% 3) biber mainv2
% 4) pdflatex mainv2 x 2


%Atención
%Está prohibido usar simbolos en \index{•} como tildes o simbolos matematicos.
% After this, when you wish to update the bibliography/index use the appropriate
% command above and make sure to compile with pdflatex several times 
% afterwards to propagate your changes to the document.
%
%%%%%%%%%%%%%%%%%%%%%%%%%%%%%%%%%%%%%%%%%

%----------------------------------------------------------------------------------------
%	PACKAGES AND OTHER DOCUMENT CONFIGURATIONS
%----------------------------------------------------------------------------------------

\documentclass[
	12pt, % Default font size, select one of 10pt, 11pt or 12pt
	fleqn, % Left align equations
	a4paper, % Paper size, use either 'a4paper' for A4 size or 'letterpaper' for US letter size
	oneside, % Uncomment for oneside mode, this doesn't start new chapters and parts on odd pages (adding an empty page if required), this mode is more suitable if the book is to be read on a screen instead of printed
]{LegrandOrangeBook}

% Book information for PDF metadata, remove/comment this block if not required 
\hypersetup{
	pdftitle={Telecomunicaciones II}, % Title field
	pdfauthor={Jose Hancco}, % Author field
	pdfsubject={Educación}, % Subject field
	pdfkeywords={Telecomunicaciones, UNSA, Pregrado, Señales}, % Keywords
	pdfcreator={LaTeX}, % Content creator field
}

\addbibresource{sample2.bib} % Bibliography file

\definecolor{ocre}{RGB}{63, 76, 83} % Define the color used for highlighting throughout the book, replace it with average color for a better presentation:
% Background.pdf -> Average color=63,76,83
% Background1.pdf -> Average color=31,31,40
% Background2.pdf -> Average color=123,119,101
% Background3.pdf -> Average color=103,61,76
\chapterimage{Final1.jpg} % Chapter heading image
\chapterspaceabove{6.5cm} % Default whitespace from the top of the page to the chapter title on chapter pages
\chapterspacebelow{6.75cm} % Default amount of vertical whitespace from the top margin to the start of the text on chapter pages

%---------------------------------------------------------------------------------------
\begin{document}

%----------------------------------------------------------------------------------------
%	TITLE PAGE
%----------------------------------------------------------------------------------------

\begingroup
\thispagestyle{empty} % Suppress headers and footers on the title page
\begin{tikzpicture}[remember picture,overlay]
\node[inner sep=0pt] (background) at (current page.center) {\includegraphics[width=\paperwidth]{background1.pdf}};%here you change your bookcover background
\draw (current page.center) node [fill=ocre!30!white,fill opacity=0.6,text opacity=1,inner sep=1cm]{\Huge\centering\bfseries\sffamily\parbox[c][][t]{\paperwidth}{\centering Telecomunicaciones II\\[15pt] % Book title
{\Large Notas de un estudiante}\\[20pt] % Subtitle
{\huge Jose Antonio Hancco M.}}}; % Author name
\end{tikzpicture}
\vfill
\endgroup

%----------------------------------------------------------------------------------------
%	COPYRIGHT PAGE
%----------------------------------------------------------------------------------------

\newpage
~\vfill
\thispagestyle{empty}

\noindent Copyright \copyright\ 2022 Jose Hancco\\ % Copyright notice

\noindent \textsc{Libro libre de usos}\\ % Publisher

\noindent \textsc{https://github.com/Yasperterian}\\ % URL

\noindent Con licencia de Creative Commons Attribution-NonCommercial 3.0 Unported License (la ``Licencia''). No puede usar este archivo excepto de conformidad con la Licencia. Puede obtener una copia de la Licencia en \url{http://creativecommons.org/licenses/by-nc/3.0}. A menos que lo exija la ley aplicable o se acuerde por escrito, el software distribuido bajo la Licencia se distribuye \textsc{``tal cual'', sin garantías ni condiciones de ningún tipo}, ya sea expresa o implícita. Consulte la Licencia para conocer el idioma específico que rige los permisos y las limitaciones en virtud de la Licencia.\\ % License information, replace this with your own license (if any)

\noindent \textit{Primera edición, septiembre 2022} % Printing/edition date

\noindent Si existe algún error, crees que una sección se puede mejorar o dar cualquier tipo de \textit{feedback} acerca del libro no dudes y mándame un correo a \textit{jhanccoma@unsa.edu.pe}, te responderé lo más pronto que pueda y gracias por mejorar este libro de todos y para todos.
%----------------------------------------------------------------------------------------
%Dedicate
%----------------------------------------------------------------------------------------
\clearpage
\begin{center}
    \thispagestyle{empty}
    \vspace*{\fill}
    \textit{Realmente no sé a quien dedicar este texto, pues no es más que una recopilación de lo que voy aprendiendo para convertirme en ingeniero en telecomunicaciones, visto de manera dedico este libro a todos los estudiantes que por lo menos toman en cuenta la información contenida en el libro y más aún con las ganas de aprender de otro estudiante, como lo soy yo.}
    \vspace*{\fill}
\end{center}
\clearpage
%----------------------------------------------------------------------------------------
%	TABLE OF CONTENTS
%----------------------------------------------------------------------------------------

%\usechapterimagefalse % If you don't want to include a chapter image, use this to toggle images off - it can be enabled later with \usechapterimagetrue

\chapterimage{chapter_head_generalindex3.pdf} % Table of contents heading image

\pagestyle{empty} % Disable headers and footers for the following pages

\tableofcontents % Print the table of contents itself

\cleardoublepage % Forces the first chapter to start on an odd page so it's on the right side of the book

\pagestyle{fancy} % Enable headers and footers again
%----------------------------------------------------------------------------------------
%	PART
%----------------------------------------------------------------------------------------

%----------------------------------------------------------------------------------------
%	NEW CHAPTER
%----------------------------------------------------------------------------------------
\part{Antenas}
\chapterimage{chapter_head_ANT.pdf} % Chapter heading image

\chapter{Conceptos básicos sobre antenas}
\section{Introducción}
Como siempre, antes de este curso hay que recordar algunos términos o conceptos para poder entender cosas que se vienen. Empezamos con las unidades logarítmicas:
\begin{align*}
Belio&=\log\left(\frac{P_{out}}{P_{in}}\right)\\
Decibelio(dB)&=10\cdot\log\left(\frac{P_{out}}{P_{in}}\right)\\
Decibelio(dB)&=20\cdot\log\left(\frac{V_{out}}{V_{in}}\right)\\
Neper(Np)&=ln\left(\frac{V_{out}}{V_{in}}\right)
\end{align*}
Asimismo debemos tener en cuenta las demás medidads respecto a un valor como 1mW, 1W, 1V, etc.\footnote{Estas puedes ser vistas en la sección \textbf{Decibelios} en el capítulo de \textbf{Ingeniería en mantenimiento}.}
Otros conceptos importantes a recordar son:
\begin{definition}[Longitud de Onda]
\begin{equation}
\lambda=\frac{v}{f}
\label{eq:longitud de onda}
\end{equation}
Donde:
\begin{itemize}
\item $\lambda$: Longitud de onda. (m)
\item \textbf{v}: Velocidad. Si el medio es el aire o espacio libre: $v=c=3\times\basedec{8}$ m/s (m/s)
\item $f$: Frecuencia. (Hz)
\end{itemize}
\end{definition}
\begin{definition}[Temperatura]
\begin{equation}
\frac{°C}{5}=\frac{°F-32}{9}=\frac{K-273}{5}=\frac{R-492}{9}
\end{equation}
Despejando podemos obtener:
\begin{align*}
°C=\frac{5}{9}\left(°F-32\right)\\
K=°C+273\\
R=°F+460
\end{align*}
\end{definition}
Además debemos recordar las bandas y frecuencias designadas por la ITU:
\begin{table}[H]
\centering
\begin{tabular}{|c|c|c|c|}
\hline
\rowcolor[HTML]{9698ED} 
N° de banda & Rango de frecuencia & Indicativo & Propagación                    \\ \hline
2           & 30-300 Hz           & ELF        & Onda terrestre                 \\ \hline
3           & 0.3-3 KHz           & SLF        & Onda terrestre                 \\ \hline
4           & 3-30 KHz            & VLF        & Onda terrestre                 \\ \hline
5           & 30-300 KHz          & LF         & Onda terrestre y superficial   \\ \hline
6           & 0.3-3 MHz           & MF         & Onda superficial               \\ \hline
7           & 3-30 MHz            & HF         & Onda superficial y Ionosférica \\ \hline
8           & 30-300 MHz          & VHF        & Onda Ionosférica y directa     \\ \hline
9           & 0.3-3 GHz           & UHF        & Onda directa                   \\ \hline
10          & 3-30 GHz            & SHF        & Onda directa                   \\ \hline
11          & 30-300 GHz          & EHF        & Onda directa e infrarojo       \\ \hline
12          & 0.3-3 THz           &            & Luz infraroja                  \\ \hline
13          & 3-30 THz            &            & Luz infraroja                  \\ \hline
14          & 30-300 THz          &            & Luz infraroja                  \\ \hline
15          & 0.3-3 PHz           &            & Luz visible                    \\ \hline
16          & 3-30 PHz            &            & Luz ultravioleta               \\ \hline
17          & 30-300 PHz          &            & Rayos X                        \\ \hline
18          & 0.3-3 EHz           &            & Rayos X                        \\ \hline
19          & 3-30 EHz            &            & Rayos cósmicos                 \\ \hline
\end{tabular}
\caption{Designación de bandas CCIR por la ITU.}
\end{table}
Los medios de transmisión:
\begin{table}[H]
\resizebox{\textwidth}{!}{
\begin{tabular}{|c|c|c|c|}
\hline
\rowcolor[HTML]{9698ED} 
Medio de transmisión                                                                         & Banda de frecuencia & Longitud de onda & Aplicación principal                                                                                                                                         \\ \hline
\begin{tabular}[c]{@{}c@{}}Par de alambres,\\ cable multipar\end{tabular}                    & 30-300 Hz           & 10000-1000 Km    & Comunicación submarina                                                                                                                                       \\ \hline
\begin{tabular}[c]{@{}c@{}}Par de alambres,\\ cable multipar\end{tabular}                    & 0.3-3 KHz           & 1000-100 Km      & \begin{tabular}[c]{@{}c@{}}Telefonía, transmisión de datos,\\ telex, fax.\end{tabular}                                                                       \\ \hline
\begin{tabular}[c]{@{}c@{}}Par de alambres,\\ cable multipar,\\ ondas de tierra\end{tabular} & 3-30 KHz            & 100-10 Km        & \begin{tabular}[c]{@{}c@{}}Telefonía de onda portadora baja,\\ capacidad, navegación y radiotelegrafía.\end{tabular}                                         \\ \hline
\begin{tabular}[c]{@{}c@{}}Par de alambres,\\ ondas de tierra\end{tabular}                   & 30-300 KHz          & 10-1 Km          & \begin{tabular}[c]{@{}c@{}}Telefonía de onda portadora mediana\\ capacidad, radiofaro, navegación,\\ radiodifusión onda larga.\end{tabular}                  \\ \hline
\begin{tabular}[c]{@{}c@{}}Cable coaxial, ondas\\ de cielo\end{tabular}                      & 0.3-3 MHz           & 1000-100 m       & \begin{tabular}[c]{@{}c@{}}Radiodifusión, AM, radio aficionados,\\ radio móvil.\end{tabular}                                                                 \\ \hline
\begin{tabular}[c]{@{}c@{}}Cable coaxial, cable UTP cat 3-4,\\ ondas de cielo\end{tabular}   & 3-30 MHz            & 100-10m          & \begin{tabular}[c]{@{}c@{}}Radio aficionados, comunicaciones milirares,\\ marítimas, radio telefonía movil.\end{tabular}                                     \\ \hline
\begin{tabular}[c]{@{}c@{}}Cable coaxial, cable UTP cat 5,\\ ondas directas\end{tabular}     & 30-300 MHz          & 10-1 m           & \begin{tabular}[c]{@{}c@{}}TV, radiodifusión FM, multiacceso radial,\\ radio enlaces, direccionales.\end{tabular}                                            \\ \hline
Ondas directas                                                                               & 0.3-3 GHz           & 100-10 cm        & \begin{tabular}[c]{@{}c@{}}TV, telemetría por radar, comunicaiones\\ militares por satélite, telefonía celular, radio\\ de espectro ensanchado.\end{tabular} \\ \hline
Guía de onda, línea visual                                                                   & 3-30 GHz            & 10-1 cm          & \begin{tabular}[c]{@{}c@{}}Comunicaiones vía satélite, radio enlace\\ direccional analógico y digítal, operación\\ aérea por radar.\end{tabular}             \\ \hline
Guía de onda, línea visual.                                                                  & 30-300 GHz          & 1-0.1 cm         & \begin{tabular}[c]{@{}c@{}}Comunicación militar por satelite, \\ radio astronomia, aterrizaje por radar.\end{tabular}                                        \\ \hline
Fibra óptica                                                                                 & 100-1000 THz        & 3-0.3 pm         & \begin{tabular}[c]{@{}c@{}}Telefonia muy alta capacidad, servicios de\\ banda ancha (SONET, SDH y ATM),\\ video conferencia, CATV por F.O.\end{tabular}      \\ \hline
\end{tabular}
}
\end{table}
\subsection{Ancho de banda y capacidad de información}\index{Ancho de banda y capacidad de información}
Las limitaciones más importantes para el funcionamiento de una sistema de comunicaciones son el \textbf{ruido} y el \textbf{ancho de banda}. El ancho de banda de un canal de comunicación es la diferencia entre la frecuencia máxima y mínima que puede pasar por el canal. El ancho de banda de un canal de comunicación debe ser igual o mayor que el ancho de banda de la información.
\begin{definition}[Ley de Hartley]
Es la medida de cuanta información se puede transferir a través de un sistema de comunicaciones en un determinado tiempo.
\begin{equation}
I\approx B\times t
\label{eq:hartley}
\end{equation}
Donde:
\begin{itemize}
\item \textbf{I}: Capacidad de información.
\item \textbf{B}: Ancho de banda. (Hz)
\item \textbf{t}: Tiempo de transmisión. (s)
\end{itemize}
\end{definition}
\begin{notation}
Se requieren \textbf{3 KHz} de ancho de banda para transmitir las señales telefónicas con calidad de voz.\\
Se asigna \textbf{200 KHz} para transmisión comercial de FM para música, con alta fidelidad.\\
Se requieren casi \textbf{6 MHz} de ancho de banda para emitir señales de televisión de alta calidad
\end{notation}
Otra medida que debemos saber es:
\begin{definition}[Capacidad de información de un canal digital]
Shannon relacionó la capacidad de información de un canal de comunicaciones, en bits por segundo (bps), con el ancho de banda y la relación señal a ruido: 
\begin{equation}
I=B\cdot\log_2(1+S/N)
\label{eq:shannon}
\end{equation}
Donde:
\begin{itemize}
\item I: Capacidad de información. (bps)
\item B: Ancho de banda. (Hz)
\item S/N: Relación señal a ruido.
\end{itemize}
\end{definition}
\subsection{Ruido}\index{Ruido}
Energía eléctrica \textbf{no deseable} presente en la banda útil del circuito de comunicación. Se puede clasificar el ruido en dos categorías:
\begin{enumerate}
\item \textbf{Correlacionado}: Solo existe \textbf{cuando hay una señal}. Es aquel que se relaciona mutuamente con la señal, y no puede estar en un circuito a menos que haya una señal de entrada.
Se produce por amplificación no lineal, e incluye la distorsión armónica (cuando se producen las armónicas no deseadas de una señal, debido a una amplificación no lineal) y de intermodulación (generación de frecuencias indeseables de suma o diferencia), ya que las dos son formas de distorsión no lineal.
\item \textbf{No Correlacionado}: Está presente \textbf{siempre}, haya o no señal. El ruido no correlacionado puede sub dividirse en dos categorías generales:
\begin{enumerate}
\item \textbf{El Ruido Externo }es el que se genera fuera del dispositivo o circuito. Hay tres causas principales de ruido Externo:
\begin{enumerate}
\item \textbf{Ruido atmosférico}: Perturbaciones eléctricas naturales. Electricidad estática (rayos)
\item \textbf{Ruido extraterrestre}: Señales eléctricas originadas fuera de la atmósfera terrestre (solar y cósmico)
\item \textbf{Ruido hecho por el hombre}: Su puente principal son mecanismos que producen chispas, ruido industrial (conmutadores, generadores, lámparas fluorescentes)
\end{enumerate}
\item \textbf{El Ruido Interno} es la interferencia eléctrica generada dentro de un dispositivo o circuito. Las causas principales son:
\begin{enumerate}
\item \textbf{Ruido Térmico}: Asociado con el movimiento rápido y aleatorio de electrones libre, producido por la agitación térmica.
\item \textbf{Ruido de Tiempo de Tránsito}: Variación irregular y aleatoria, producida por la modificación de una corriente de portadores, cuando pasa de la entrada a la salida de un dispositivo.
\item \textbf{Ruido de Disparo}: Se debe a la llegada aleatoria de portadoras al elemento de salida de un dispositivo electrónico (diodo, FET, transistor bipolar).
\end{enumerate}
\end{enumerate}
\end{enumerate}
\subsection{Ruido térmico}\index{Ruido térmico}
Es el movimiento \textbf{aleatorio} de los electrones libres dentro de un conductor, causado por la agitación térmica. Lamentablemente, este ruido está a lo largo de todo el espectro electromagnético.
Llamado también: Movimiento Browniano por su descubridor Robert Brown, Ruido de Johnson en honor a quien lo relacionó con el movimiento de los electrones y Ruido Blanco porque se produce en todas las frecuencias.
\begin{definition}[Potencia de ruido térmico]
\begin{equation}
P_{tn}=K\cdot T\cdot B
\label{eq:ruido termico}
\end{equation}
Donde:
\begin{itemize}
\item $P_{tn}$: Potencia del ruido  térmico\footnote{tn:thermal noise o ruido térmico.} (W)
\item \textbf{K:} Constante de Boltzmann= $1.38\times\basedec{-23} J/°K$.
\item \textbf{T}: Temperatura absoluta. (°K)
\item \textbf{B}: Ancho de banda. (Hz)
\end{itemize}
Alternativamente, el ruido térmico puede ser expresado en dBm, para ello debemos usar la siguiente expresión comparada con un 1mW:
\begin{equation}
P_{tn}(dBm)=10\cdot\log\left(\frac{K\cdot T\cdot B}{0.001}\right)
\label{eq:ruido termico log}
\end{equation}
En temperatura ambiente (17 C o 290 K), el ruido térmico ambiente:
\begin{equation}
P_{tn}(dBm)=-174 dBm+10\log(B)
\label{eq:ruido termico ambiente}
\end{equation}
\end{definition}
\begin{definition}[Voltaje ruido térmico]
\begin{equation}
V_{tn}=\sqrt{4\cdot R\cdot K\cdot T\cdot B}
\label{eq:voltaje ruido termico}
\end{equation}
Donde:
\begin{itemize}
\item \textbf{R}: Resistencia interna. ($\Omega$)
\item \textbf{$V_{tn}$}: Voltaje RMS del ruido. (V)
\end{itemize}
\end{definition}
\begin{notation}
Para la máxima potencia transferencia de potencia $R_L=R_I$.
\end{notation}
\begin{definition}[Relación señal a ruido-SNR]
Es la relación en decibelios entre la potencia de la señal(S) y la potencia del ruido(N):
\begin{equation}
\label{eq:snr}
SNR=10\log_{10}\left(\frac{S}{N}\right)=20\log_{10}\left(\frac{V_s}{V_n}\right)
\end{equation}
\end{definition}
\begin{remark}
El \textbf{factor a ruido} se define como el cociente entre la potencia SNR de entrada y potencia SNR de salida. Por consecuencia, la \textbf{cifra de ruido} es el factor de ruido expresado en dB.
\end{remark}
\begin{notation}
Para voltaje, 6dB indica que la salida es dos veces el valor de la entrada, es decir: Si la entrada es 1, la salida será 2. Para potencia, 3dB indica lo mismo: el valor de la salida es dos veces el valor de la entrada.
\end{notation}
\subsection{Ángulo crítico}\index{Angulo crítico}
Debemos recordar ecuaciones como Ley de Snell,
\begin{definition}[Ley de Snell]
\begin{align}
n_1\sin\theta_1&=n_2\sin\theta_2\\
\sqrt{\epsilon_1}\sin\theta_1&=\sqrt{\epsilon_2}\sin\theta_2
\label{eq:snell}
\end{align}
Donde:
\begin{itemize}
\item $n$: Indice de refracción.
\item $\theta$ Ángulo de refracción.
\item $\epsilon$: Permitividad relativa del material.
\end{itemize}
\end{definition}
\begin{figure}[H]
\centering
\includegraphics[width=0.6\linewidth]{Ant/Ant6.png}
\caption{Ley de Snell}
\end{figure}
dentro de ella una que usaremos es la del \textit{ángulo crítico}:
\begin{center}
\includegraphics[width=0.8\linewidth]{Ant/Ant1.png}
\end{center}
La forma de obtener el ángulo crítico es:
\begin{equation}
\sin\theta_c=\frac{n_2}{n_1}
\label{eq: angulo critico}
\end{equation}
\begin{notation}
El índice de refracción esta definido como el cociente de la \textbf{velocidad de la luz en el vacío} entre la \textbf{velocidad de la luz del medio donde se propaga}. Generalmente se utiliza la velocidad de la luz en el vacío (\textit{c}) como medio de referencia para cualquier materia, aunque durante la historia se han utilizado otras referencias, como la velocidad de la luz en el aire. En el caso de la luz, es igual a $n=\sqrt{\epsilon_r\cdot\mu_r}$. Para la mayoría de los materiales, la \textbf{permeabilidad magnética relativa} ($\mu_r$) es muy cercano a 1 en frecuencias ópticas, es decir, luz visible, por lo tanto, \textit{n} es aproximadamente $\sqrt{\epsilon_r}$
\end{notation}
\subsection{Repaso: Propagación de ondas electromagnéticas}
La propagación de OEM por el espacio libre se suele llamar propagación de radiofrecuencia o radio propagación. Las OEM, en el espacio libre se propagan en línea recta a la velocidad de la luz ($3\basedec{8}m/s$). 
\begin{definition}[Distancia máxima de línea]
\begin{subequations}
\begin{align}
d_{max}&=\sqrt{2\cdot h}
\label{eq:dist millas} \\
d_{max}&=\sqrt{17\cdot h}\label{eq:dist km}
\end{align}
\end{subequations}
Donde:
\begin{itemize}
\item $d_{max}$: Distancia máxima de línea de vista. (millas ó Km)
\item \textbf{h}: Altura (Pies ó m)\footnote{La distancia será en millas si se trabaja con la ecuación \ref{eq:dist millas}, donde la altura debe ser introducida en pies. Para la ecuación \ref{eq:dist km}, la distancia estará en metros y la altura en metros.}
\end{itemize}
\end{definition}
Hablemos también de las \textbf{pérdidas por trayectoria}. El modelo de pérdida por trayectoria en el espacio libre es usado para predecir la intensidad del nivel de recepción cuando el transmisor y receptor tienen una trayectoria de línea de vista clara, sin obstrucciones entre ellos.
\begin{definition}[Pérdidas por trayectoria]
\begin{subequations}
\begin{align}
&L_p=\left(\frac{4\pi\cdot d}{\lambda}\right)^2=\left(\frac{4\pi\cdot d\cdot f}{c}\right)^2
\label{eq:perdidas trayec adimensional} \\
&L_p=20\cdot\log\left(\frac{4\pi\cdot d\cdot f}{c}\right)=20\cdot\log\left(\frac{4\pi}{c}\right)+20\cdot\log(f)+20\cdot\log(d)\label{eq:perdidas trayec db}
\end{align}
\end{subequations}
Donde:
\begin{itemize}
\item \textbf{d}: Distancia. (m)
\item \textbf{\textit{f}}: Frecuencia. (Hz)
\item \textbf{\textit{c}}: Velocidad de la luz.
\item $\lambda$: Longitud de onda. (m)
\end{itemize}
Si la distancia se expresa en Km y la frecuencia en MHz:
\begin{equation}
L_p(dB)=32.4+20\cdot\log(f)+20\log(d)
\end{equation}
\end{definition}
Otro término usado es la \textbf{polarización}, recordando que una OEM contiene un campo eléctrico y un campo magnético que forman 90° entre sí. Por lo tanto, la polarización de una OEM plana, no es mas que la orientación del vector campo eléctrico respecto a la superficie de la tierra.\\
\textbf{Tipos de polarización:}
\begin{enumerate}
\item \textbf{Polarización lineal}: Si la polarización permanece constante. Las formas lineales son:(Fig. \ref{fig:pol lin})
\begin{enumerate}
\item \textbf{Polarización Horizontal}: Campo eléctrico paralelo a la superficie de la tierra.
\item \textbf{Polarización Vertical}: Campo eléctrico perpendicular a la superficie terrestre.
\end{enumerate}
\item \textbf{Polarización Circular}: Luz polarizada circularmente consta de dos ondas electromagnéticas planas perpendiculares con una diferencia de fase de 90º. (Fig. \ref{fig:pol circ})
\item \textbf{Polarización elíptica}: La luz polarizada elípticamente consiste de dos ondas perpendiculares de amplitudes desiguales y con una diferencia de fase de 90º. (Fig. \ref{fig:pol elip})
\end{enumerate}
\begin{figure}[]
\centering
\subfloat[El campo eléctrico transversal de la onda va acompañado de un campo magnético como el que se ilustra.]{\includegraphics[width=0.5\linewidth]{Ant/Ant2.png}
\label{fig:pol lin}}
\subfloat[El vector de polarización gira 360º a medida de que la onda recorre una longitud de onda en el espacio]{\includegraphics[width=0.5\linewidth]{Ant/Ant3.png}\label{fig:pol circ}}\\
\subfloat[Cuando la Intensidad de Campo varía con cambios en la polarización, se dice que es una Polarización Elíptica.]{\includegraphics[width=0.5\linewidth]{Ant/Ant4.png}\label{fig:pol elip}}
\subfloat[Tipos de polarizaciones.]{\includegraphics[width=0.5\linewidth]{Ant/Ant5.png}}
\caption{Polarización de ondas.}
\end{figure}
\begin{notation}
Si el vector gira en sentido de las manecillas del reloj, se dice que es Derecho, si es contrario se dice que es Izquierdo
\end{notation}
\subsubsection{Densidad de potencia radiada}
La \textbf{rapidez} con que la energía pasa a través de una \textbf{superficie} dada en el espacio libre se llama D\textbf{ensidad de Potencia}. La \textbf{Densidad de Potencia Radiada} se define como la potencia por unidad de superficie en una determinada dirección. Las unidades son \textbf{Watt por Metro Cuadrado} ($W/m^2$). Se puede calcular a partir de los valores eficaces de los campos eléctrico o magnéticos.
\begin{definition}[Potencia Isotrópica Radiada Equivalente]
\begin{equation}
PIRE=P_r\cdot D=P_A\cdot G
\label{eq:pire}
\end{equation}
Donde:
\begin{itemize}
\item PIRE:  Potencia Isotrópica Radiada Equivalente. (W)
\item $P_R$: Potencia de radiación.
\item $P_A$: Potencia suministrada a la antena. (W)
\item \textbf{G}: Ganancia.
\item \textbf{D}: Directividad de la antena.\footnote{Estos conceptos se detallan más adelante.}
\end{itemize}
\end{definition}
\begin{definition}[Intensidad de campo]
Es la intensidad de los campos eléctrico y magnético de una onda electromagnética que se propaga por el espacio libre.
\begin{equation}
\mathbb{P}=E\cdot H=\frac{PIRE}{4\pi r^2}=Z_s\times H^2=\frac{E^2}{Z_s}
\label{eq:int campo}
\end{equation}
Donde:
\begin{itemize}
\item $\mathbb{P}$: Densidad de potencia. ($W/m^2$)
\item \textbf{E}: Intensidad de campo eléctrico. (V/m)
\item \textbf{H}: Intensidad del campo magnético. (A/m)
\item \textbf{r}: Radio de la esfera. (m)
\item \textbf{\textit{PIRE}}: Potencia Isotrópica Radiada Equivalente. (W)
\item \textbf{$Z_S$}: Impedancia en el vacío. ($\Omega$)
\end{itemize}

\end{definition}
\begin{definition}[Impedancia en el espacio libre]
La relación entre el módulo del campo eléctrico y el módulo del campo magnético es la impedancia característica del medio. La impedancia característica de un medio de transmisión \textbf{sin pérdidas} en igual a la raíz cuadrada de la relación de su permeabilidad magnética entre su permitividad eléctrica:
\begin{equation}
Z_s=\sqrt{\frac{\mu_0}{\epsilon_0}}=377\Omega
\label{eq:imp carac espacio libre}
\end{equation}
Donde:
\begin{itemize}
\item $Z_s$: Impedancia en el espacio libre. ($\Omega$)
\item $\mu_0$: Permeabilidad magnética ($1.26\basedec{-6}$H/m ó $4\pi\cdot K N/A^2$ donde $K=\basedec{-7}$).
\item $\epsilon_0$: Permitividad eléctrica del vacío ($8.85\basedec{-12}$F/m).
\end{itemize}
\end{definition}
%\begin{definition}[Potencia total radiada]
%Se puede obtener como la integral de la Densidad de Potencia en una esfera que encierre a la antena.
%\begin{equation}
%P_{tr}=\frac{P_{rad}}{4\pi r^2}
%\end{equation}
%\end{definition}
La \textbf{Intensidad de Radiación} es la potencia radiada por unidad de ángulo sólido en una determinada dirección.
\subsubsection{Ley de cuadrado inverso}
La \textbf{densidad de potencia} es inversamente proporcional al cuadrado de la distancia de la fuente.
\begin{equation}
\frac{\mathbb{P}_2}{\mathbb{P}_1}=\left(\frac{r_1}{r_2}\right)^2
\label{eq:ley cuadrado inverso}
\end{equation}
Para que se cumpla esta ley, la velocidad de propagación en todas las direcciones debe ser uniforme (Medio Isotrópico)
\subsubsection*{Atenuación y absorción}
\textbf{Atenuación} es la reducción de la Densidad de Potencia con la distancia. La atenuación se debe al esparcimiento esférico de la onda, se le llama ``atenuación espacial'' de la onda. Se expresa generalmente en términos del logaritmo de la relación de  densidad de  potencia (pérdida en dB)\\
\textbf{Absorción} solo se presenta cuando los CEM se propagan por la atmósfera. Es la energía transferida de la OEM a los átomos y las moléculas de la atmósfera. La absorción de radiofrecuencias en una atmósfera normal, es relativamente insignificante a frecuencias por debajo de 10 GHz.
\section{Antenas}
Las dos funciones primordiales de la antena son:
\begin{enumerate}
\item Convertir la energía electromagnética, procedente del generados a través de la línea de transmisión, en energía electromagnética que se propaga libremente por el espacio.
\item Adapta la impedancia interna del generador a la impedancia del espacio.
\end{enumerate}
En las líneas de transmisión se propagan ondas electromagnéticas \textbf{guiadas}, es decir campos electromagnéticos variables entre cargas y corrientes. Las antenas convierten estos ondas electromagnéticas guiadas en \textbf{libres} y \textbf{viceversa}. Tanto las ondas guiada como las libres son señales de radio.\\
En el proceso de su propagación, las ondas de radio se \textbf{dispersan} más allá de las líneas de radio-comunicación y son absorbidas por el medio circundante. Si la dirección de radiocomunicación es conocida y limitada, las perdidas pueden reducirse concentrando las ondas emitidas en direcciones definidas.\\
Con lo definido sobre antenas y conocimientos en impedancias, podemos definir:
\begin{corollary}[Impedancia de antena]
Es la relación entre tensión y corriente en sus terminales de entrada (de la antena). Como todas, dicha impedancia es en general compleja: parte real ó resistencia de antena y la parte compleja ó reactancia de antena.
\end{corollary}
\subsection{Tipos de antenas}
A grandes rasgos existen dos tipos de antenas: antenas de \textbf{transmisión} y antenas de \textbf{recepción}.
\subsubsection*{Antenas de transmisión}
La antena de transmisión transforma energía de un campo electromagnético estacionario producido por la señal de radio, en energía de un campo electromagnético de radiación, añadiendo además que este último debe emitirse en unas direcciones dadas.
\begin{figure}[]
\centering
\includegraphics[width=0.8\linewidth]{Ant/Ant7.png}
\caption{Gráficas de antena.}
\end{figure}
\subsubsection*{Antenas de recepción}
La antena de recepción está destinada a la transformación de la energía de una radioseñal consistente en un campo de radiación que procede de una dirección dada, en energía de un campo estacionario de ondas electromagnéticas.
\begin{remark}
La antena de transmisión y recepción tienen procesos \textbf{recíprocos}. Esto quiere decir que existe la posibilidad de utilizar la misma antena en calidad de transmisora y de receptora, y de conservar invariables los parámetros principales de la antena.
\end{remark}
\subsection{Características y parámetros de las antenas transmisoras}
Los parámetros de una antena son los que permiten especificar el funcionamiento de las mismas, y por lo tanto son susceptibles de ser medidos. Se puede especificar la antena como un \textbf{conjunto de parámetros} conectados con los  requisitos de un sistema más amplio de radiocomunicaciones o radiodifusión.\\
\textbf{Parámetros de antena}:
\begin{itemize}
\item Diagrama de radiación
\item Densidad de potencia radiada
\item \textbf{Directividad}: Es la relación entre la densidad de potencia radiada en la dirección de máxima radiación, a una cierta distancia R, y la potencia total radiada dividida por el área de la esfera de radio R. La directividad se puede calcular a partir del diagrama de radiación. La ganancia de una antena es igual a la directividad multiplicada por la eficiencia. La relación entre la densidad de potencia radiada por la antena en la dirección útil y la que radia por el lóbulo trasero se conoce como relación delante/detrás (forward/backward) y es un importante parámetro de diseño de la antena en lo relativo a interferencias.\\
El ángulo que hace referencia al diagrama de radiación del lóbulo principal en el planohorizontal de la antena se denomina ``azimut'', que para el diagrama de radiación vertical se denomina ``ángulo de elevación'', que se diseña para concentrar el máximo de radiación para aquellos ángulos por debajo de la horizontal, que es donde se agrupan los usuarios, ya que las antenas se colocan en cotas elevadas para alcanzar una mayor cobertura.
\item \textbf{Ganancia}: Es la relación entre la densidad de potencia radiada en la dirección del máximo a una distancia R y la potencia total entregada a la antena dividida por el área de una esfera de radio R. La eficiencia es la relación entre la ganancia y la directividad, que coincide con la relación entre la potencia total radiada y la potencia entregada a la antena.
\item \textbf{Polarización}: La polarización electromagnética, en una determinada dirección, es la figura geométrica que traza el extremo del vector campo eléctrico a una cierta distancia de la antena, al variar el tiempo. La polarización puede ser lineal, circular y elíptica. La polarización lineal puede tomar distintas orientaciones (horizontal, vertical, $+45^\circ$, $-45^\circ$). Las polarizaciones circular o elíptica pueden ser a derechas o izquierdas (dextrógiras o levógiras), según el sentido de giro del campo (observado alejándose desde la antena). Se llama diagrama copolar al diagrama de radiación con la polarización deseada, y diagrama contrapolar (crosspolar, en inglés) al diagrama de radiación con la polarización contraria.
\item \textbf{Rendimiento}: El rendimiento de una antena transmisora es la relación entre la potencia de radiación y la potencia total aplicada a la antena, en la cual se toma en cuenta, además de la potencia de radiación, la potencia de pérdida.
\item \textbf{Impedancia}: Una antena se tendrá que conectar a un transmisor (o a un receptor) y deberá radiar (recibir) el máximo de potencia posible con un mínimo de perdidas. Se deberá adaptar el transmisor o receptor a la antena para una máxima transferencia de potencia, que se suele hacer a través de una línea de transmisión. Esta línea también influirá en la adaptación, debiéndose considerar entre otros, su impedancia característica y atenuación.\\
La impedancia característica ($Z_0$) es un parámetro que depende de parámetros primarios; de la relación longitud-diámetro del material del conductor y de la frecuencia de trabajo, mientras que la impedancia de entrada es el parámetro circuital de la antena (relación del voltaje de entrada a la corriente de entrada).
\item \textbf{Anchura de haz}:  Es un parámetro de radiación, ligado al diagrama de radiación. Se puede definir el ancho de haz a -3 dB, que es el intervalo angular en el que la densidad de potencia radiada es igual a la mitad de la máxima. También se puede definir el ancho de haz entre ceros, que es el intervalo angular del haz principal del diagrama de radiación, entre los dos ceros adyacentes al máximo.
\item Adaptación
\item Área y longitud efectiva
\end{itemize}
\begin{definition}[Potencia de radiación y resistencia de radiación]
Representa la característica de la antena para la emisión energía electromagnética.
\begin{equation}
R_r=\frac{P_r}{i^2}
\label{eq:resistencia radiacion}
\end{equation}
Donde:
\begin{itemize}
\item $R_r$: Resistencia de total de radiación. ($\Omega$)
\item $P_r$: Potencia de radiación. (W)
\item \textbf{i}: Valor eficaz de la corriente de la antena. (A)
\end{itemize}
\end{definition}
Cuantitativamente la resistencia de radiación de define como aquella resistencia pura en la que se libera una potencia numéricamente igual a la potencia de radiación, para una corriente en la resistencia igual ala corriente en la antena.
\begin{definition}[Potencia de Pérdidas y resistencia de pérdidas]
Potencia que se pierde por el calentamiento del conductor, en los aisladores, en la tierra y en los objetos situados cerca de la antena.
\begin{equation*}
R_p=\frac{P_p}{i^2}
\label{eq:resistencia perdidas}
\end{equation*}
Donde:
\begin{itemize}
\item $R_p$: Resistencia de pérdidas. ($\Omega$)
\item $P_p$: Potencia de pérdidas. (W)
\item \textbf{i}: Valor eficaz de la corriente de la antena. (A)
\end{itemize}
\end{definition}
\begin{definition}[Potencia de una antena y resistencia activa total o resistencia de antena]
Resistencia que corresponde a potencia suministrada a la antena. Potencia de antena es la suministrada a la antena por el transmisor, a través de la línea de transmisión, se obtiene con la suma de la potencia de \textbf{radiación} y la potencia de \textbf{pérdidas}.
\begin{equation}
P_A=P_r+P_p=i^2\left(R_r+R_p\right)
\label{eq:potencia antena}
\end{equation}
\begin{equation}
R_A=R_r+R_p
\end{equation}
Donde:
\begin{itemize}
\item $P_A$: Potencia suministrada a la antena. (W)
\item $P_r$: Potencia de radiación. (W)
\item $P_p$: Potencia de pérdidas. (W)
\item \textbf{i}: Valor eficaz de la corriente de la antena. (A)
\item $R_r$: Resistencia de radiación. ($\Omega$)
\item $R_p$: Resistencia de pérdidas. ($\Omega$)
\item $R_A$: Resistencia activa total. ($\Omega$)
\end{itemize}
\end{definition}
\begin{remark}
Ten en cuenta que cuando se habla de atenuación y la unidad son los \textbf{decibelios}, estos se \textbf{restan}. Por el contrario, si se habla de una atenuación en \textbf{watts}, este \textbf{divide} a la potencia.
\end{remark}
\begin{definition}[Rendimiento o Eficiencia de una antena]
Es la relación entre la Potencia de Radiación y la Potencia Suministrada a la Antena.
\begin{equation*}
\eta_A=\frac{P_r}{P_A}=\frac{R_r}{R_r+R_p}=\frac{R_r}{R_A}=\frac{G}{D}
\label{eq:rendimiento de antena}
\end{equation*}
\begin{displaymath}
0\leq\eta_A\leq 1
\end{displaymath}
Donde:
\begin{itemize}
\item $\eta_A$: Rendimiento de antena.
\item $P_A$: Potencia suministrada a la antena. (W)
\item $P_r$: Potencia de radiación. (W)
\item $R_r$: Resistencia de radiación. ($\Omega$)
\item $R_p$: Resistencia de pérdidas. ($\Omega$)
\item $R_A$: Resistencia activa total. ($\Omega$)
\item \textbf{G}: Ganancia.
\item \textbf{D}: Directividad.
\end{itemize}
\end{definition}
\subsection{Parámetros de acción directiva de antenas}\index{Parámetros de acción directiva de antenas}
\subsubsection{Directividad}
La característica de directividad de antena muestra la \textbf{dependencia} de la intensidad de campo de radiación respecto a la dirección, con la condición que este campo sea medido siempre a igual distancia de la antena. Para el estudio de la radiación de una antena, se supone que la antena está situada en el punto medio de una ``esfera'' y en el origen de un sistema de coordenadas espaciales.
En la superficie de la ``esfera'' se calcula E y H en cualquier punto de esta, alejado una distancia \textit{r} del centro del dipolo.\\
\textbf{Características:}
\begin{enumerate}
\item \textbf{Propiedad Directiva}: Todas las antenas reales tienden a concentrar los campos radiado en alguna dirección.
\item \textbf{Característica de Directividad}: Depende de la intensidad de campo de radiación, respecto a la dirección, medido siempre a igual distancia de la antena. 
\item \textbf{Función de Directividad}: Expresión matemática de la directividad.
\begin{displaymath}
f^2(\theta,\phi)
\end{displaymath}
\item \textbf{Diagrama de Directividad}: Representación gráfica de la función de directividad. Normalmente se expresa en proyecciones:
\begin{enumerate}
\item Plano Horizontal ( $\phi$ varía y $\theta$ = 90º)
\item Plano Vertical ( $\theta$ varía y $\phi$	 = 0º)
\end{enumerate}
\end{enumerate}
\begin{definition}[Factor de directividad]
Factor de Directividad es la relación entre la densidad del flujo de potencia emitido por la antena dada en una \textbf{determinada dirección}, y la densidad de flujo de potencia que emitiría una antena absolutamente \textbf{no direccional} en cualquier dirección, siendo iguales las potencias totales de radiación de ambas antenas y medido a igual distancia.
\begin{equation}
D=\frac{\mathbb{P}_{max}}{\mathbb{P}_{ref}}=\frac{E_{max}^2}{E_0^2}
\label{eq:directividad}
\end{equation}
Donde:
\begin{itemize}
\item \textbf{D}: Factor de directividad.
\item $\mathbb{P}_{max}$: Densidad de potencia en un punto, en la dirección de máxima radiación. ($W/m^2$)
\item $\mathbb{P}_{ref}$: Densidad de potencia en el mismo punto, con una antena no direccional o de referencia. ($W/m^2$)
\end{itemize}
\end{definition}
\begin{figure}[H]
\centering
\includegraphics[width=0.7\linewidth]{Ant/Ant8.png}
\caption{Representación tridimensional de la directividad.}
\end{figure}
\begin{definition}[Ganancia de potencia]
\begin{equation}
G=\eta_A\cdot D=\frac{\mathbb{P}_{max}}{\mathbb{P}_{ref}'}
\label{eq:ganancia}
\end{equation}
Donde:
\begin{itemize}
\item \textbf{G}: Ganancia.
\item $\eta_A$: Rendimiento de la antena.
\item \textbf{D}: Directividad de la antena.
\item $\mathbb{P}_{max}$: Densidad de potencia en un punto, en la dirección de máxima radiación. ($W/m^2$)
\item $\mathbb{P}_{ref}'$: Densidad de potencia en el mismo punto, con una antena no direccional o de referencia sin pérdidas. ($W/m^2$)
\end{itemize}
\end{definition}
\begin{definition}[Factor de calidad]
\begin{equation}
Q=\frac{f_r}{BW}
\label{eq:factor de calidad}
\end{equation}
Donde:
\begin{itemize}
\item Q: Factor de calidad.
\item $f_r$: Frecuencia de resonancia. (Hz)
\item $BW$: Ancho de banda. (Hz)
\end{itemize}
\end{definition}
\subsection{Antenas receptoras}
\begin{definition}[Área de captura]
Mientras que la \textbf{ganancia de potencia} es el parametro natural para describir la mayor densidad de potencia de una señal transmitida, por las propiedades direccionales de la antena transmisora, para describir las propiedades receptiras de una antena se usa una cantidad relacionada: \textbf{área de captura}
\begin{equation}
A_{cap}=\frac{G_r\cdot\lambda^2}{4\pi}
\label{eq:area capturada}
\end{equation}
Donde:
\begin{itemize}
\item $A_{cap}$: Área efectiva de captura. ($m^2$)
\item $G_r$: Ganancia del receptor.
\item $\lambda$: Longitud de onda de la señal recibida. (m)
\end{itemize}
\end{definition}
\begin{definition}[Potencia capturada]
Potencia disponible en las terminales de salida de la antena receptora. La potencia capturada es directamente proporcional a la densidad de potencia recibida y al área de captura de la antena receptora.
\begin{equation}
P_{cap}=\mathbb{P}\cdot A_{cap}\cdot G
\label{eq:potencia capturada}
\end{equation}
Donde:
\begin{itemize}
\item $P_{cap}$: Potencia capturada. (W)
\item $\mathbb{P}$: Densidad de potencia capturada. ($W/m^2$)
\item $A_{cap}$: Área capturada. ($m^2$)
\item \textbf{G}: Ganancia.
\end{itemize}
\end{definition}
\begin{remark}
Recuerda que en todas las ecuaciones se debe trabajar con unidades \textbf{lineales}. Esto significa que todos los decibelios que nos da los problemas deben ser transformadas a sus equivalentes lineales para poder usar las ecuaciones dadas a los largo de este capitulo.
\end{remark}
\section{Transformador $\lambda$/4}
El transformador lambda-cuartos (``\textit{Quarter-Wavelength}´´) es el adaptador de impedancias más sencillo y usado para conseguir la impedancia que queremos en una línea de transmisión. Se trata de una línea de transmisión con longitud $\lambda$/4 en la frecuencia de diseño. El ancho de banda de adaptación de un transformador lambda-cuartos suele ser estrecho, aunque se puede ensanchar utilizando múltiples secciones lambda-cuartos en lugar de solo una. Por lo tanto si solo queremos que el diseño este adaptado a la frecuencia de trabajo con solo una sección nos valdría, sin embargo, si necesitamos un ancho de banda mayor en el que nuestro circuito funcione y cumpla las especificaciones, necesitaremos utilizar múltiples secciones adaptadoras.\\
\begin{notation}
Hay una demostración detrás de toda la teoría y demostración de las ecuaciones, puedes buscarlas en libros o en internet. Usaremos las ecuaciones resultantes
\end{notation}
\begin{figure}[H]
\centering
\includegraphics[width=\linewidth]{Ant/ant21.png}
\caption{Transformador $\lambda/4$.}
\label{fig:trans lambda}
\end{figure}
\textbf{Pasos para calcular}:
\begin{enumerate}
\item Normalizamos la impedancia. Ubicamos en la carta.
\item Trazamos un círculo con entro en 1 y radio el punto que trazamos anteriormente.
\item Ubicamos el cruce con el eje R más cercano. Siguiente la dirección \textit{hacia el generador}. En la parte superior de la carta tomamos como punto de referencia resistencia infinita, en la parte negativa tomamos resistencia cero. Ese valor hay que desnormalizar para hallar $Z_b$.
\item Con el valor hallado calculamos $Z_a$.
\item Con el mismo criterio del paso 3, trazamos una recta desde el centro de la carta que pase por el punto de la impedancia de carga. Hallamos la intersección de la recta y el anillo exterior "hacia el generador". Hallamos la distancia desde ese punto hacia Z=0 o Z=$\infty$ para hallar la longitud.
\end{enumerate}
\begin{example}[Se tiene una $Z_L=32-\iu 40$ con una impedancia característica de $80\Omega$. Diseñe la adaptación por $\lambda/4$.]
Normalizamos nuestra impedancia, en este caso estamos trabajando con una impedancia característica de 80 y no de 50 como casi siempre se suele hacer:
\begin{displaymath}
\overline{Z_L}=\frac{32-\iu 40}{80}=0.4-\iu 0.5
\end{displaymath}
Ubicamos la impedancia normalizada en la carta de Smith.
\begin{center}
\begin{tikzpicture}
	\begin{smithchart}[show origin]
       % Resistance plot - violet
        \addplot+[mark=*,only marks,samples at={0, -.01,...,-.5}, mark options={solid},color={violet},mark size=.2,line width=1] (0.4,x);
        % Reactance plot - blue
        \addplot[domain=0:90, samples=600, color=blue] {-.5};
        %Puntos
        \addplot+ [black,mark=o,only marks,point meta=explicit symbolic,nodes near coords] coordinates {(0.4,-0.5) [$\overline{Z_L}$]};
	\end{smithchart}
\end{tikzpicture}
\end{center}
Trazamos un circulo con centro en 1 y como radio el punto $Z_L$:
\begin{center}
\begin{tikzpicture}
	\begin{smithchart}[show origin]
        % Circle 
        \path[draw=violet] (0pt,0pt) circle (1.5cm);
        %Puntos
        \addplot+ [black,mark=o,only marks,point meta=explicit symbolic,nodes near coords] coordinates {(0.4,-0.5) [$\overline{Z_L}$] (0.32,0) [$\overline{Z_b}$]};
	\end{smithchart}
\end{tikzpicture}
\end{center}
El círculo corta la carta de Smith en dos puntos, \textbf{elegimos el más cercano}\footnote{Si estamos en la parte negativa, buscamos a R=0, si estamos en la parte positiva buscamos R=$\infty$. Sigue la dirección del anillo que indica hacia el generador.}. Ese punto es la impedancia del punto B normalizada. Desnormalizando:
\begin{displaymath}
Z_b=0.32\cdot 80=25.6\Omega
\end{displaymath}
Esto significa que la carga compleja que tenemos, en los puntos de $Z_b$ presenta una impedancia real a una distancia $\ell$. La distancia $\ell$ es hallada trazando una línea \textcolor{red}{recta} desde el origen hacia el exterior de la carta, que pase por le punto $\overline{Z_L}$:
\begin{center}
\begin{tikzpicture}
	\begin{smithchart}[show origin]
        % Circle 
        \path[draw=violet] (0pt,0pt) circle (1.5cm);
		%Line
		\addplot+ [mark size=1,red,line width=1] coordinates{(1,0) (0,-0.57)};
        %Puntos
        \addplot+ [black,mark=o,only marks,point meta=explicit symbolic,nodes near coords] coordinates {(0.4,-0.5) [$\overline{Z_L}$] (0.32,0) [$\overline{Z_b}$]};
	\end{smithchart}
\end{tikzpicture}
\end{center}
En el anillo hacia el generador, leemos el valor que resulta: 0.417 y nos movemos con la dirección que indica el anillo \textit{wavelength toward generator} hacia el eje real más cercano (en este caso R=0\footnote{Hay situaciones en la cual el más cercano es R=$\infty$, también es valido.}). Si nos recordamos, la vuelta completa es de $0.5\lambda$, para hallar esa distancia restamos:
\begin{displaymath}
\ell=0.5-0.417=0.083\lambda
\end{displaymath}
Recuerda que $\lambda$ se calcula con la ecuación \ref{eq:longitud de onda}\\
Calculando la impedancia de $\lambda/4$:
\begin{displaymath}
Z_a=\sqrt{25.6\cdot 80}=45.25\Omega
\end{displaymath}
Con ambas impedancias, solo bastaría con conseguir un cable de longitud $\lambda/4$ de impedancia $Z_a=45.25$\\
Con el transformador podemos acoplar una carga compleja o desacoplada a nuestra impedancia característica, logrando así la máxima eficiencia, máximo rendimiento, máxima transferencia de potencia, mínimo coeficiente de reflexión.
\end{example}
\chapterimage{chapter_head_ANT.pdf} % Chapter heading image
\chapter{Antenas para HF, VHF, UHF}
\section{Isotrópica}
La antena isotrópica es una antena hipotética sin pérdida (se refiere a que el área física es cero y por lo tanto no hay pérdidas por disipación de calor) que tiene intensidad de radiación igual en todas direcciones. (IEEE Standard Dictionary of Electrical and Electronic Terms, 1979). \\
Sirve de base de referencia para evaluar la \textbf{directividad}. La antena isotrópica no es una antena, sino un concepto de referencia para evaluar a las antenas en su función de concentración de energía y a las pérdidas por propagación en el espacio libre en los enlaces de radiofrecuencia. Su patrón de radiación es una esfera.\\
Cada aplicación y cada banda de frecuencia presentan características peculiares que dan origen a
unos tipos de antenas especiales muy diversas. Los tipos más comunes de antenas son los que se
explican en los siguientes apartados.
\section{Dipolo simétrico y asimétrico}\index{Dipolo simétrico y asimétrico}
\subsection{Diseño de un dipolo}
Vamos a diseñar un dipolo simétrico, usaremos una frecuencia de 93.5 MHz. Como vimos en teoría de la antena y usando la ecuación \ref{eq:longitud de onda} calculamos la distancia de los brazos:
\begin{wrapfigure}{r}{0.3\linewidth}
	\centering
    \includegraphics[width=.9\linewidth]{Ant/Ant9}
  \caption{Longitudes del dipolo.}
  \label{fig:antena dipolo medidas}
\end{wrapfigure}
\begin{displaymath}
\lambda=\frac{3\basedec{8}m/s}{93.5\basedec{6}Hz}=3.20 m\simeq 3200mm
\end{displaymath}
Con $\lambda$ calculado, podemos hallar nuestra longitud de la antena:
\begin{displaymath}
\frac{\lambda}{2}=1600mm
\end{displaymath}
Y la longitud de cada brazo es:
\begin{displaymath}
\frac{\lambda}{4}=800mm
\end{displaymath}
Ahora calcularemos el GAP\footnote{Distancia entre los brazos del dipolo simétrico}, el valor del GAP esta dado por la ecuación:
\begin{equation}
GAP=0.01\lambda
\label{eq:gap}
\end{equation}
Usando la ecuación \ref{eq:gap} calculamos nuestro valor para el GAP:
\begin{displaymath}
GAP=0.01(3200mm)=32mm\approx 40mm
\end{displaymath}
Aunque el valor nos salga 32mm, se aproxima al \textbf{múltiplo} de diez cercano o un número par. Esto ya es a criterio del ingeniero. En nuestro diseño lo hicimos a 40mm.
Resumiendo los datos en la figura \ref{fig:antena dipolo medidas}.\\
También necesitamos la caja de radiación, para esto solo agregamos la longitud de $\lambda/4$ a todos los lados:
\begin{figure}[H]
\centering
\includegraphics[width=\linewidth]{Ant/ant10_1.png}
\caption{Caja de radiación.}
\end{figure}
\begin{notation}
Nota que en la figura \ref{eq:gap}, si sumamos el GAP y longitud de ambos brazos no sale $\lambda/2$, estos valores van a ser modificador luego, son tan solo una aproximación.
\end{notation}
Con esto ya calculado, llevamos la antena al ANSYS:\\
\begin{tabular}{k{0.5\linewidth}  j{0.5\linewidth}}
        \includegraphics[width=\linewidth]{ant/ant11.png} & \textbf{Creamos un cilindro} \newline 
        En un nuevo proyecto creamos un cilindro, no importa el tamaño porque luego lo editaremos. \\
        \includegraphics[width=\linewidth]{ant/ant12.png} & \textbf{Cambiamos las propiedades} \newline 
        En el panel de objetos doble click en \textit{cylinder1} y cambiamos el nombre, material y color. Aceptar.\\
        \includegraphics[width=\linewidth]{ant/ant13.png} & \textbf{Dimensiones} \newline 
        Desplegamos el otro menu bajo \textit{BrazoL} y clickamos en CreateCylinder. Modificamos las dimensiones de acuerdo a lo calculado arriba. \textit{Center Position} es un punto relativo al cual vamos a trabajar (a modo de anclaje), aquí colocaremos la mitad del GAP, \textit{Radius} 20, pues nuestro diámetro es 40 y \textit{Height} ponemos -800, recuerda que es un dipolo, si lo situamos en el origen un brazo ira hacia los positivos y el otro hacía el positivo.\\
        \includegraphics[width=\linewidth]{ant/ant14.png} & \textbf{Duplicamos el brazoL} \newline 
        Seleccionando nuestro cilindro, vamos a \textit{Edit>Duplicate>Around Axis}, configuramos como la imagen y OK.
    \end{tabular}
   
\begin{tabular}{k{0.5\linewidth}  j{0.5\linewidth}}
        \includegraphics[width=\linewidth]{ant/ant15.png} & \textbf{Creamos la caja de radiación} \newline 
        Creamos un cilindro que cubra nuestra antena. Si la orientación del cilindro causase problemas, asegurate que el plano de referencia en la barra de herramientas este en \textbf{YZ}. Cambiamos su nombre a ``aire'', material: vacuum, color celeste y transparencia 0.9.\\
        \includegraphics[width=\linewidth]{ant/ant16.png} & \textbf{Ajustamos la caja de radiación} \newline 
        Vamos a \textit{Create Cylinder}. \textit{Center Position}: Mitad del largo de la caja de radiación (1620); \textit{Radius}: mitad del diámetro (820) y \textit{Height}: Menos la longitud de la caja de radiación (-3240).\\
         \includegraphics[width=\linewidth]{ant/ant17.png} & \textbf{Creación de puerto} \newline 
        Creamos un rectángulo en el espacio libre que dejamos entre ambas brazos. Le cambiamos el nombre a ``Puerto'' y cambiamos el color.\\
        \includegraphics[width=\linewidth]{ant/ant18.png} & \textbf{Fronteras} \newline 
        Click derecho en ``aire''>Assign boundary>Radiation\\
        \includegraphics[width=\linewidth]{ant/ant19.png} & \textbf{Excitación} \newline 
        Click derecho en ``puerto''>Assign excitation>Port>Lumped port. Aceptamos la impedancia de 50 ohms. En la siguiente pantalla, en la columna \textit{integration line} seleccionamos \textit{new line}. Seleccionamos la mitad de nuestro puerto, asegurándonos que se muestre un triangulo como en la figura. Damos aceptar y finalizar.\\
\end{tabular}
\begin{tabular}{k{0.5\linewidth}  j{0.5\linewidth}}
        \includegraphics[width=\linewidth]{Ant/ant20.png} & \textbf{Análisis} \newline 
        En la ventana de \textit{project manager}>Analysis (click derecho) > Add solution setup>Advanced. Colocamos nuestra frecuencia de trabajo y aceptar. En la siguiente ventana damos aceptar. Validamos el proyecto\footnote{En la pestaña \textit{Simulation}} y \textit{Analyze all}. Esperamos que termine.
\end{tabular}
\begin{definition}[Antena armónica]
Si se dispone de un número entero de semiondas, el dipolo recibe el nombre de \textbf{antena armónica}. La longitud de la antena armónica es:
\begin{equation}
l=p\cdot \frac{\lambda}{2}
\label{eq:long ant armo}
\end{equation}
Donde:
\begin{itemize}
\item \textbf{l}: Longitud de antena. (m)
\item \textbf{p}: Número de armónicos.
\item $\lambda$: Longitud de onda. (m)
\end{itemize}
\end{definition}
\subsubsection{Campo de un dipolo simétrico}
\begin{figure}[H]
\centering
\includegraphics[width=0.7\linewidth]{ant/ant22.png}
\caption{Disposición mutua del dipolo simétrico y del punto M en el que se termina su campo de radiación.}
\end{figure}
\begin{definition}[Intensidad de campo eléctrico]
La intensidad del campo eléctrico del dipolo esta dada por:
\begin{equation}
E_{inst}=\frac{60\cdot I_m}{r}\cdot f(\theta)\cdot\sin\parentesis{\omega t-\beta r}
\label{eq:campo electrico dipolo}
\end{equation}
Donde $I_m$ es la amplitud de la corriente en el antinodo.
\end{definition}
En la ecuación \ref{eq:campo electrico dipolo}, el factor $\sin\parentesis{\omega t-\beta r}$ indica que el dipolo simétrico emite ondas progresivas. Dentro de este factor, al ángulo de fase $\omega t-\beta r$ depende de la distancia \textit{r}, pero no de $r_1$ ni de $r_2$. Esto indica que le punto medio O es el \textbf{punto equivalente de radiación} (centro de fase de todo el dipolo), y lo segundo significa que las ondas radiadas son \textbf{esféricas}.\\
La amplitud de la intensidad del campo eléctrico en la dirección del ángulo $\theta$:
\begin{equation}
E_m=\frac{60\cdot I_m}{r}f(\theta)
\end{equation}
Siendo $f(\theta)$ la función directividad del dipolo:
\begin{equation}
f(\theta)=\frac{\cos\parentesis{\frac{\pi l}{\lambda}\cos\theta}-\cos\parentesis{\frac{\pi l}{\lambda}}}{\sin\theta}
\label{eq:func theta}
\end{equation}
\begin{corollary}
La ecuación \ref{eq:func theta} es función del ángulo $\theta$, es decir, la amplitud de la intensidad de campo de un dipolo simétrico varía en el plano meridional a consecuencia de la interferencia de los campos de las secciones elementales del dipolo.
\end{corollary}


\section{Antena Marconi}
Conceptualmente, se trata de un conductor vertical de poco espesor, perpendicular a la Tierra. Puede imaginarse como un brazo de un dipolo, al cual la Tierra le sirve de espejo para "fabricar" la imagen del otro brazo del dipolo.
\begin{wrapfigure}{r}{0.45\linewidth}
  \begin{center}
    \includegraphics[width=0.8\linewidth]{ant/ant23.png}
  \end{center}
  \caption{Antena marconi}
\end{wrapfigure}
La \textbf{altura} de una antena Marconi es del orden de $\lambda/4$, con una impedancia caracteristica de $36\Omega$ y ganancia isotropica de 4.76dBi.
\begin{remark}
No confundir altura con longitud de onda de la antena. La altura es $\lambda/4$ y la longitud de onda de la antena es $\lambda/2$.
\end{remark}
Las perdidas del suelo afecta a la impedancia de la antena y el punto de eficiencia de la alimentación. Una antena de Marconi montado sobre un suelo perfectamente llevar a cabo tendría una entrada de impedancia que es la mitad de la impedancia de un dipolo, o como vimos anteriormente 36 ohms. Cuando se monta sobre un un fondo real , la impedancia de entrada puede variar de 38 ohm para una antena de radiodifusión de AM bien diseñado montado sobre un suelo preparado especialmente, a más de 100 ohm para una Marconi montada por encima, sin preparación de tierra pobre que no tiene radiales.\\
La perdida de suelo \textbf{reduce} la eficiencia de la antena, porque parte de la energía que es suministrada a la antena se disipa en el suelo en vez de ser radiada. La eficiencia puede ser calculada a partir del \textbf{valor medio de resistencia} de entrada utilizando la siguiente formula:
\begin{equation}
\eta=\frac{36}{R_{input}}
\end{equation}
\begin{remark}
El patrón de radiación de la Antena Marconi es una dona a la mitad. No existe radiación directamente hacia arriba en la dirección del cable.
\begin{center}
\includegraphics[width=0.7\linewidth]{ant/ant24.png}
\end{center}
\end{remark}
\subsubsection{Aplicaciones}
\begin{itemize}
\item Servicios de radio terrestre
\item Telecomunicaciones a bajas frecuencias
\end{itemize}
\section{Yagi-Uda}
\begin{figure}[H]
\centering
\includegraphics[width=0.6\linewidth]{ant/ant25.png}
\caption{Partes antena Yagi-Uda}
\end{figure}
\textbf{Elementos de antena}:
\begin{itemize}
\item \textbf{Excitación}: Pueden ser activos o excitados, estos se conectan directamente a la línea de transmisión y reciben potencia de la fuente.
\item \textbf{Parásitos}: No se conectan a la línea de transmisión y reciben la energía a través de la inducción mutua. Estos elementos se clasifican en reflectores y directores:
\begin{itemize}
\item \textbf{Reflector}: Elemento parásito más largo que el elemento de excitación. Reduce la intensidad de la señal que esta en su dirección e incrementa la que esta en dirección opuesta.
\item \textbf{Director}: Elementos parásitos más cortos que su elemento de excitación. Incrementa la intensidad de campo en su dirección y la reducen en la dirección opuesta.
\end{itemize}
\end{itemize}
En las antenas Yagi, la polarización depende de la posición de la antena, pudiendo ser horizontal y vertical tanto su posición como la polarización. Además en común ver un dispositivo que adapta impedancias de la línea de transmisión a la antena (más de 4 elementos).
\begin{figure}[H]
\centering
\includegraphics[width=0.7\linewidth]{ant/ant26.png}
\caption{Polarización horizontal y vertical: antena yagi.}
\end{figure}
También se puede mezclar ambas, para generar con la misma antena las dos polarizaciones a la vez. Aunque se necesitan 2 cables coaxiales (uno por antena).
\begin{figure}[H]
\centering
\includegraphics[width=0.7\linewidth]{Ant/ant27.png}
\caption{Doble antena Yagi.}
\end{figure}
Su \textbf{ganancia} esta dada por:
\begin{equation}
G=10\log n
\end{equation}
Donde \textit{n} es el número de elementos a considerar.\\
En cuanto a las distancias: el \textbf{reflector} es una barra de aluminio aproximadamente 5\% más larga que el dipolo, y el \textbf{director} se corta aproximadamente 5\% respecto al elemento de \textbf{excitación}. Es espacio \textbf{entre los elementos} por lo general está entre 0.1 y 0.2 longitudes de onda.\\
Directividad de una antena yagi es calculada con:
\begin{equation}
D=\frac{P_{max}}{\frac{W_t}{4\pi r2}}=\frac{4\pi r^2\cdot E_{max}^2}{I_1^2\cdot \Re{Z_{in}}\cdot\eta}
\end{equation}
\subsubsection{Aplicaciones}
\begin{itemize}
\item Las antenas Yagi Uda se emplean en la recepción de señales de TV ya que esta antena tiene una buena capacidad de recepción.
\item Utilizado en aplicaciones de defensa.
\item Empleado en el dominio astronómico.
\item También se utiliza en radioastronomía.
\end{itemize}
\section{Antenas helicoidales}
Antena con forma de helice a lo largo de un eje, trabaja con polarización circular y opera especialmente en el rango de 2 a 5 GHz (VHF y UHF), su diseño es muy fácil y práctico.\\
La antena helicoidal posee dos modos: \textbf{Normal} en el cual la antena de comporta similar a una antena dipolo con una radiación omnidireccional respecto al eje de la hélice; las dimensiones de la hélice son pequeñas en comparación con la longitud de onda. Luego esta la \textbf{axial}, en el cual la radiación se produce en el mismo sentido del eje de la hélice, es cuando las dimensiones de la hélice son comparables a una longitud de onda:
\begin{figure}[H]
\centering
\includegraphics[width=0.7\linewidth]{ant/ant28.png}
\caption{Modos de la antena helicoidal.}
\end{figure}
\begin{figure}[H]
\centering
\includegraphics[width=0.7\linewidth]{ant/ant29.png}
\caption{Medidas de una antena helicoidal.}
\end{figure}
La ganancia de potencia una antena helicoidal esta definida por:
\begin{equation}
A_p(dB)=10\log\corchetes{15\cdot \parentesis{\frac{D\pi}{\lambda}}^2\cdot\frac{NS}{\lambda}}
\end{equation}
Donde:
\begin{itemize}
\item $A_p(dB)$: Ganancia de la potencia de la antena. (dB)
\item \textbf{D}: Diámetro de la hélice. (m)
\item \textbf{N}: Número de vueltas.
\item \textbf{S}: Paso de la hélice. (m)
\item $\lambda$: Longitud de onda. (m)
\end{itemize}
Y el ancho de haz puede ser calculado con:
\begin{equation}
\theta=\frac{52}{\frac{\pi D}{\lambda}\sqrt{\frac{NS}{\lambda}}}
\end{equation}
\begin{itemize}
\item $\theta$: Ancho de haz. (sex.)
\item \textbf{D}: Diámetro de la hélice. (m)
\item \textbf{N}: Número de vueltas.
\item \textbf{S}: Paso de hélice. (m)
\item $\lambda$: Longitud de onda. (m)
\end{itemize}
\subsection{Aplicaciones}
\begin{itemize}
\item Radios portátiles en baja frecuencia (30-150 MHz).
\item Dispositivos inalambricos (2.425 GHz)
\item Dispositivos GSM, GPS bandas de 434 MHz, 868 MHz y 2400 MHz.
\item Radio empaquetado de alta velocidad. (S5-PSK, 1.288 Mbit/s)
\item Acceso a Internet inalámbrico de alta velocidad.
\end{itemize}
\begin{notation}
Algunas otras antenas son bocina que sirven como alimentador de parabólicas, ranuras para comunicaciones espaciales.
\end{notation}
\begin{example}[Cálculo del horizonte]
Calcule el horizonte de radio en kilómetros de una antena transmisora ubicada en lo alto de un cerro de 590 pies de altura sobre el nivel del mar en un mástil de 20 pies y la antena receptora que se halla sobre un mástil de 26 pies de altura sobre el nivel de mar.\\
\textbf{Solución:}
Tenemos dos antenas, por lo que debemos calcular el alcance de cada antena usando las ecuaciones \ref{eq:dist millas} o \ref{eq:dist km} \footnote{Dependiendo en que unidades trabajes, en este caso trabajaremos en millas y luego lo convertiremos a kilómetros.} , teniendo esas ecuaciones calculemos el alcance de cada antena:\\
\begin{align*}
d_1&=\sqrt{2(20+590)}\\
d_1&=34.93 millas\\
d_2&=\sqrt{2(26)}\\
d_2&=7.21 millas
\intertext{Sumamos ambas distancas para hallar la distancia máxima entre ellas}
d_T=34.93+7.21=42.14\simeq 67.82 Km
\end{align*}
\end{example}
\begin{example}[Cálculo de densidad de potencia]
Calcule la densidad de potencia a 32 Km de la estación transmisora, la potencia a la salida del transmisor es de 15 W, la atenuación en el conductor es 1,2 dB. La antena transmisora tiene 6.2 dB de factor de directividad y un rendimiento del 95\%.\\
\textbf{Solución:}\\
Antes de usar algún tipo de ecuación debemos analizar nuestra atenuación; nos menciona que tenemos 15 W a la salida del transmisor, luego tenemos una atenuación, es decir, se pierde energía por la línea de transmisión así que no todos los 15 W llegarán a la antena. Tenemos dos formas de formas de calcular la potencia que llegará a la antena. Primero la calculamos usando solo unidades lineales:
\begin{displaymath}
P[W]=\frac{P_{T_x}[W]}{At}
\end{displaymath}
Como estamos trabajando con unidades lineales, debemos convertir la atenuación que originalmente esta en decibelios a unidades lineales, para ello:
\begin{displaymath}
At=10^{\frac{1,2 dB}{10}}=1,32
\end{displaymath}
Ahora recién con este valor podemos hallar la potencia que llega a la antena tomando en cuenta la atenuación:
\begin{displaymath}
P[W]=\frac{15 W}{1,32}=11,36 W
\end{displaymath}
La \textbf{segunda manera} de hallar la potencia de la antena es hallando  el valor en unidades de decibelios; planteando la potencia en dB debemos de considerar que ahora las unidades no se deben dividir, se \textbf{deben restar}, esto se puede demostrar usando las propiedades de los logaritmos (esto no de desarrollará pues ya es bien sabido como). Sin perder la formalidad de las matemáticas, lo podemos expresar como:
\begin{displaymath}
P_A[dBW]=P_{T_x}[dBW]-  At[dB]
\end{displaymath}
Ahora debemos que usar nuestra potencia en unidades de decibelios, para ello:
\begin{displaymath}
P_{T_x}[dBW]=10\cdot\log 15=11,76 dBW
\end{displaymath}
Ahora que todos los datos están en las mismas unidades, efectuamos la operación:
\begin{displaymath}
P_A[dBW]=11,76 dBW - 1,2 dB=10,56 dBW\simeq 11,38 W
\end{displaymath}
Hay una pequeña diferencia entre ambas más mientras no afecte significativamente a los cálculos es aceptable.\\
Calcularemos la intensidad de campo usando la ecuación \ref{eq:int campo}, notaremos de acuerdo a la ecuación \ref{eq:pire}, PIRE puede ser calculado de dos maneras, nosotros tenemos potencia de antena, por lo que nos falta ganancia que puede ser calculada facilmente con la ecuación  \ref{eq:ganancia}:
\begin{displaymath}
G=0,95\cdot 10^{\frac{6,2}{10}}=3,96
\end{displaymath}
Teniendo en cuenta que debe ser en unidades lineales, por eso convertimos la directividad de decibelios a lineales.\\
Con la ganancia hallada y usando la ecuación \ref{eq:int campo}:
\begin{displaymath}
\mathbb{P}=\frac{11,36 W\cdot 3,96}{4\cdot\pi\cdot(32\basedec{3})^2}=3,5 mW/m^2
\end{displaymath}
\end{example}
\begin{example}[Intensidad de campo eléctrico]
Si la intensidad del campo eléctrico en la zona de recepción a 21 Km de la estación transmisora es de 1,1 mV/m, cuanto debe ser la ganancia de la antena transmisora expresada en dB, si la potencia a la entrada de la antena transmisora es de 1,5 W.\\
\textbf{Solución:}\\
Teniendo en cuenta las ecuaciones \ref{eq:pire} y \ref{eq:int campo} podemos despejar la siguiente expresión:
\begin{displaymath}
\frac{P_A\cdot G}{4\pi\cdot r^2}=\frac{E^2}{Z_S}
\end{displaymath}
Reemplazando los valores que tenemos en la expresión:
\begin{displaymath}
\frac{1.5 W\cdot G}{4\pi\cdot (12\basedec{3})^2}=\frac{(1.1\basedec{-3})^2}{377 \Omega}
\end{displaymath}
Despejando la ganancia de esta ecuación:
\begin{displaymath}
G=\frac{(1.1\basedec{10^{-3}})\cdot 4\pi\cdot (12\basedec{3})^2}{377\cdot 1,5}=3.87
\end{displaymath}
Sin embargo nos pides la respuesta en decibelios:
\begin{displaymath}
G[dB]=10\log(3.37)=5.88 dB
\end{displaymath}
\end{example}
\begin{corollary}[Características del diagrama de radiación de un dipolo horizontal próximo a un plano de tierra.]
\begin{itemize}
\item Un dipolo horizontal colocado sobre una pantalla irradia, igual que un dipolo aislado, ondas esféricas progresivas con centro de fases en el punto medio entre el dipolo y su imagen eléctrica.
\item Un dipolo horizontal colocado a cualquier altura \textit{h}, no irradia a lo largo de la superficie de la pantalla.
\item En los diagramas de directividad del dipolo horizontal sobre la pantalla conductora, se tiene igualdad de máximos y la presencia de mínimos nulos en los lóbulos del diagrama.
\end{itemize}
\end{corollary}
\begin{corollary}[Características constructivas de las antenas periodo logarítmicas]
Es una antena conformada por barios dipolos que resuenan a distintas frecuencias, que la hace como una antena de banda ancha. El esparcimiento entre dipolos es variable y la alimentación se hace desde el elemento más corto. 
\end{corollary}
\chapterimage{chapter_head_ANT.pdf} % Chapter heading image
\chapter{Radioenlace de microondas}
\section{Guías de onda}
\begin{wrapfigure}{r}{0.4\linewidth}
  \begin{center}
    \includegraphics[width=.9\linewidth]{Ant/ant30.png}
  \end{center}
  \caption{Propagación de OEM por la guía de onda.}
\end{wrapfigure}
En electromagnetismo y en telecomunicaciones, una guía de onda es cualquier estructura física que guía ondas electromagnéticas. Una guía de ondas o guíaondas, es un tubo conductor hueco, por lo general de corte transversal rectangular, a veces circular o elíptico. Son adecuadas para transmitir señales debido a sus \textbf{bajas pérdidas}, sin embargo cuentan con ancho de banda limitado y volumen mayor que líneas coaxiales.\\
Es importante que sepas que las guías de onda no conducen corriente en el sentido estricto, sino mas bien sirve como una frontera para confinar la energía electromagnética; las OEM se reflejan (su energía) en la superficie interior.
Para lograr una baja resistencia y por ende con pérdidas de transmisión a un nivel bajo, estas guías de onda deben ser fabricadas de cobre, aluminio o bronce.\\
\subsection{Conceptos generales}
Las ecuaciones de Maxwell tienen soluciones múltiples, o modos: $E=0$ y $H_n=0$. Los \textbf{modos de propagación} dependen de la longitud de onda, de la polarización y de las dimensiones de la guía. El \textbf{modo longitudinal} de una guía de onda es un tipo particular de onda estacionaria formado por ondas confinadas en la cavidad. \\
En la \textbf{línea de transmisión}, la velocidad de la onda \textbf{depende del dieléctrico}, mas no de la velocidad de la onda; en la \textbf{guía de onda}, la velocidad varía en \textbf{función de la frecuencia}.
\begin{definition}[Velocidad de fase]
\begin{equation}
V_{ph}=f\cdot\lambda
\label{eq:velfase}
\end{equation}
Donde:
\begin{itemize}
\item $V_{ph}$: Velocidad de fase. (m/s)
\item f: Frecuencia. (Hz)
\item $\lambda$: Longitud de onda. (m)
\end{itemize}
\end{definition}
\begin{definition}[Velocidad de grupo]
\begin{equation}
V_g=\frac{c^2}{V_{ph}}
\label{eq:velgrup}
\end{equation}
\begin{itemize}
\item $V_g$: Velocidad de grupo. (m/s)
\item c: Velocidad de la luz $3\times\basedec{8}$ (m/s)
\end{itemize}
La longitud de onda en la guía puede ser calculada con:
\begin{equation}
\lambda_g=\lambda_0\parentesis{\frac{V_{ph}}{c}}=\frac{c}{\sqrt{f^2-f_c^2}}=\frac{\lambda_0}{\sqrt{1-\parentesis{\frac{f_c}{f}}^2}}
\end{equation}
Donde:
\begin{itemize}
\item $\lambda_g$: Longitud de onda en la guía. (m)
\item $\lambda_0$: Longitud de onda en el espacio libre. (m)
\item $V_{ph}$: Velocidad de fase. (m/s)
\item c: Velocidad de la luz. (m/s)
\item f: Frecuencia de operación. (Hz)
\item $f_c$: Frecuencia de corte. (Hz)
\end{itemize}
\end{definition}
Combinando las ecuaciones anteriores se puede obtener la velocidad de fase como:
\begin{equation}
V_{ph}=\frac{c\cdot\lambda_g}{\lambda_0}=\frac{c}{\sqrt{1-\parentesis{\frac{f_c}{f}}^2}}
\end{equation}
\subsection{Inyección y extracción de señales}
Para realizar la \textbf{inyección} se una señal de microondas en una guía, en el extremo de la guía se introduce una sonda tipo de antena que produce una OEM, los campos eléctricos y magnéticos se combinan con la señal, rebotan en el interior de las paredes viajando por la guía de onda. En este medio \textbf{no hay perdidas por radiación}. La sonda que alimenta la guía de onda, es una antena vertical de \textbf{un cuarto de longitud de onda}. La posición de la sonda determina si la señal está polarizada de manera horizontal o vertical. Para la inyección de onda también se suelen usar \textbf{aros}; para la \textbf{extracción} se usan ambos modos: sondas o aros.
\subsection{Guía de onda rectangular}
\begin{figure}[H]
\centering
\includegraphics[width=0.7\linewidth]{Ant/ant31.png}
\caption{Guía de onda rectangular}
\end{figure}
La frecuencia de operación de una guía de onda la determina el ancho de esta (a), la dimensión de la guía se hace a lambda medios, un poco más bajo de la frecuencia de operación más baja o frecuencia de corte. Por debajo de esta \textbf{frecuencia de corte} la guía de onda no transmitirá energía.\\
La guía de onda es considerada como un \textbf{filtro pasa alto} debido a esta característica de funcionar solo a frecuencias por encima de la frecuencia de corte. La \textbf{frecuencia de corte} puede ser calculada con:
\begin{equation}
f_{c}=\frac{c}{2a}
\end{equation}
Teniendo en consideración que el ancho de la guía (a) debe estar en metros.\\
Generalmente el alto de la guía (b) es la mitad del ancho: b=a/2.\\
El corte se presenta a la frecuencia para la cual la dimensión transversal máxima de la guía, es la mitad de la longitud de la onda en el espacio libre ($\lambda_{co}$):
\begin{equation}
\lambda_{co}=2a
\end{equation}
\subsubsection{La propagación de la señal}
Cuando la sonda envía energía por la guía de onda, la señal se propaga usando el principio de reflexión de la ley de Snell. Una sonda vertical genera una \textbf{onda polarizada en forma vertical} con un campo eléctrico \textbf{vertical} y un campo magnético en \textbf{ángulo recto} con el campo eléctrico formando un ángulo recto, incluso con la dirección de propagación de la onda; por lo que se llama \textbf{campo eléctrico transverso (TE)}.\\
En caso de que el campo magnético es \textbf{perpendicular} a la dirección de propagación, se denomina \textbf{campo magnético transverso (TM)}.
\begin{figure}[H]
\centering
\includegraphics[width=0.8\linewidth]{Ant/ant32.png}
\caption{Modos de transmisión}
\end{figure}
En el modo TE, el campo eléctrico existe a través de la guía y no hay línea E que se extiendan en forma longitudinal a lo largo de la guía, los vectores E son \textbf{perpendiculares} a las paredes de la guía.
En el modelo TM, las líneas H forman aros en planos \textbf{perpendiculares} a las paredes de la guía y ninguna parte de una línea H se extiende de manera longitudinal a lo largo de la guía.
\begin{notation}
Los ángulos de incidencia y reflexión dependen de la frecuencia de operación. Conforme la frecuencia de operación decrece, el ángulo también lo hace, y la trayectoria entre los lados se vuelve más corta.
Cuando la frecuencia de operación alcanza la frecuencia de corte de la guía de onda, la señal solo bota de un lado a otro entre las paredes laterales de la guía de onda. No se propaga energía.
\end{notation}
La forma en que los campos eléctrico y magnético se propagan en el interior de la guía de onda depende de: 
\begin{itemize}
\item El método de acoplamiento de la energía  (sonda o aro).
\item De la frecuencia de operación.
\item Del tamaño de la guía de onda.
\end{itemize}
Con la designación de TE y TM se usan subíndices para describir mejor los patrones del campo E y H. Una designación típica es $TE_{m,n}$; el primer número entero (\textit{m}) indica la \textbf{cantidad de patrones} de $\lambda/2$ de líneas perpendiculares a la pared de la guía, que existen a lo largo de la dimensión mayor de la guía (sección \textit{a}) a través de la sección transversal. El segundo número (\textit{n}) indica la cantidad de patrones transversales de $\lambda/2$ que existen a lo largo de la dimensión menor de la guía (sección \textit{b}) a través de la sección transversal. Si no hay cambio en la intensidad del campo de una dimensión se usa Cero.\\
El modo $TE_{1,0}$ se llama ``modo dominante'' por ser el más natural. La guía de onda rectangular se suele trabajar dentro de los márgenes de frecuencia de $f_c$ a $2f_c$. En caso de trabajar en modos superiores de propagación, esta no se acopla bien con la carga, y se provocan reflexiones y creación de ondas estacionarias.
\subsection{Guía de onda circular}
Las guías de onda circulares son mas fácil de fabricar que las rectangulares, y mas fácil de empalmar. Las guías circulares tienen un área mucho mayor que una rectangular correspondiente, para llevar la misma señal. Otra desventaja de la guía circular es la variación de la polarización mientras la onda se propaga por ella.\\
Se usan en radar y en aplicaciones de microondas, cuando hay ventaja de propagar ondas polarizadas vertical y horizontalmente en la misma guía.
El comportamiento de las ondas electromagnéticas en las guías de onda circulares es igual que las rectangulares. El modo de propagación $TE_{1,1}$ es el dominante en las guías de onda circulares, para este modo la longitud de onda de corte es:
\begin{equation}
\lambda_0=1.7d
\label{eq:onda de corte circular}
\end{equation}
Donde \textit{d} esta en metros y es el diámetro de la guía de onda. La longitud de corte de la ecuación \ref{eq:onda de corte circular} es medida en metros.
\section{Antenas de bocina}
Las antenas de bocina son unas antenas que realizan la transición desde el medio guiado (guías de onda) al espacio libre. Las bocinas se utilizan en los satélites principalmente como alimentadores de los reflectores y en algunas ocasiones se utilizan como antenas simples cuando se requieren grandes anchos de haz. Las antenas de bocina se utilizan frecuentemente para conformar haces que den una cobertura terrestre. 
El ancho de haz necesario par dar cobertura a la tierra desde la órbita geoestacionaria es de 18 grados, que es fácilmente realizable con antenas de bocina.\\
Si una dimensión del tubo es mayor que media longitud de onda, entonces la onda puede propagar a través de la guía de onda con la perdida sumamente baja, y si al ser colocada el final de una guía de onda simplemente es dejado abierto, la onda irradiara al espacio.
\begin{figure}[H]
\centering
\includegraphics[width=.5\linewidth]{Ant/ant33.png}
\caption{Antena bocina circular.}
\end{figure}
\begin{equation}
D=\frac{4\pi}{\lambda^2}A_{ef}=\frac{4\pi}{\lambda^2}\parentesis{\pi a^2}\eta_{il}
\end{equation}
\begin{figure}[H]
\centering
\includegraphics[width=0.6\linewidth]{Ant/ant34.png}
\caption{Antena bocina rectangular.}
\end{figure}
\begin{equation}
D=\frac{4\pi}{\lambda^2}A_{ef}=\frac{4\pi}{\lambda^2}ab\eta_{il}
\end{equation}
\subsection{Tipos de bocina}
Existen dos tipos más usadas: rectangulares y cónica:
\subsubsection{Bocinas rectangulares}
Adecuada para sistemas de polarización lineal, ya que minimiza las perdidas y reduce la generación de modos de ordenes superiores que afecten al comportamiento de la eficiencia y de la polarización. Tiene la ventaja de transmitir ondas sin polarización cruzada, que junto con el hecho de que su ganancia se puede calcular exactamente a partir de sus dimensiones.
\subsubsection{Bocina cónica}
Son las que se utilizan fundamentalmente en antenas de satélites de haz global. Son las más adecuadas para utilizar polarizaciones circulares, aunque también pueda utilizar polarizaciones lineales, estas polarizaciones tienen un mejor comportamiento en las bocinas piramidales. Se pueden clasificar según el modo de propagación transmitido: bocinas de modo dominante, bocinas de modo dual y bocinas corrugadas. 
Las \textbf{bocinas de modo dual} se han desarrollado para obtener un ancho de haz igual en los planos E y H, con un bajo nivel de polarización cruzada. En este tipo de bocinas, los modos $TE_{11}$ y $TM_{11}$ son combinados con apropiadas relaciones de amplitud y diferencias de fase en su apertura.\\
Otra variante de las bocinas son las bocinas de \textbf{modos híbridos}. Corrugando la pared interior de una bocina cónica se consigue que no se propaguen ni los modos TE ni los TM, sino que se crean en la bocina unos modos híbridos. Estas antenas mejoran la polarización cruzada y el nivel de los lóbulos secundarios, otra característica remarcable es que consiguen anchos de haz simétricos respecto al eje de la bocina.\\
Bocinas de \textbf{modo dominante} (o de modo único): Se sintoniza al modo predominante de la guía de onda circular. Este es el más básico de los tres tipos.
Bocinas de modo dual (o multimodo): Se sintoniza al modo  de propagación de la onda que se propaga por la guía de onda.\\
Estas bocinas se utilizan extensamente en satélites comerciales. Pero la utilización más común de las bocinas es como un \textbf{elemento de radiación} para reflectores de antenas. La bocina se sitúa en el \textbf{foco} o en un lugar próximo a él de un reflector \textbf{parabólico} para iluminar su superficie tanto en la aplicación de transmisión como en recepción. La radiación electromagnética en la superficie del reflector produce corrientes eléctricas en la superficie. Y de estas corrientes se producen otros campos electromagnéticos que finalmente se convierten en un diagrama de radiación de campo lejano del sistema de antena total.
\section{Antenas de ranura}
Se llaman antenas de ranura aquellas en que la radiación y la recepción de las ondas electromagnéticas se realizan mediante unas o varias ranuras practicadas en una guía de ondas o un resonador volumétrico.
%Antenas de ranura
\section{Antenas reflector parabólico}
Cuando se desea la máxima \textbf{directividad} de una antena, la forma del reflector generalmente es parabólica, con la fuente primaria localizada en el foco y dirigida hacia el reflector. Las antenas con reflector parabólico, o simplemente antenas parabólicas se utilizan extensamente en sistemas de comunicaciones en las bandas de \textbf{UHF} a partir de unos 800 MHz y en las de SHF y EHF. Entre sus características principales se encuentran la sencillez de construcción y \textbf{elevada direccionalidad}. La forma más habitual del reflector es la de un paraboloide de revolución, excitado por un alimentador situado en el foco como se ilustra en la figura \ref{fig:ant par}.
\begin{figure}[H]
\centering
\includegraphics[width=0.6\linewidth]{ant/ant47.png}
\caption{Antena con reflector parabólico}
\label{fig:ant par}
\end{figure}
Otro tipo de antena, bastante utilizado en aplicaciones de radar es el cilindro parabólico que tiene la forma mostrada en la figura \ref{fig:ref cilin} y fue la primera antena con reflector utilizada por Hertz en sus experimentos. El alimentador, o fuente de energía es una antena lineal o un alineamiento de éstas, colocada en la línea focal y la reflexión en la superficie parabólica transforma el frente de onda de cilíndrico en plano.
\begin{figure}[H]
\centering
\includegraphics[width=0.6\linewidth]{ant/ant48.png}
\caption{Cilindro parabólico}
\label{fig:ref cilin}
\end{figure}
En las antenas parabólicas se aplican las propiedades ópticas de las ondas electromagnéticas. Las propiedades geométricas de la parábola son tales que las ondas emitidas por el alimentador en el foco se reflejan por la parábola en un haz de rayos paralelos al eje de la parábola, de modo que la longitud del trayecto del foco al reflector parabólico y, después, hasta la superficie de la abertura que pasa por los bordes de la parábola, es la misma para cualquier ángulo. Por consecuencia en la abertura de la antena se tiene una superficie equifase y, teóricamente, el haz radiado es cilíndrico, si bien en la práctica esto no es completamente cierto, ya que parte de la energía se dispersa en los bordes del reflector. En la figura \ref{fig:geo para} se ilustra la geometría de la antena parabólica.
\begin{figure}[H]
\centering
\includegraphics[width=0.6\linewidth]{ant/ant49.png}
\caption{Geometría de la parábola}
\label{fig:geo para}
\end{figure}
Un defecto de las antenas parabólicas con el alimentador en el foco lo constituye el hecho de que el alimentador obstruye los rayos reflejados produciendo un región de baja intensidad o sombra en el centro de la apertura. El efecto en el patrón de radiación puede estimarse aproximadamente tomando la diferencia de la radiación de la abertura y del área de sombra localizada en la dirección del alimentador. El efecto neto es una alteración del patrón de radiación en que se rellenan los nulos entre lóbulos como se ilustra en la figura \ref{fig:sombra rad} en coordenadas rectangulares.
\begin{figure}[H]
\centering
\includegraphics[width=0.6\linewidth]{ant/ant50.png}
\caption{Efecto de la sombra en el patrón de radiación}
\label{fig:sombra rad}
\end{figure}
Otro efecto que se produce cuando el alimentador está en la trayectoria de la onda reflejada es que algo de la energía de ésta regresa al sistema alimentador y produce un desacoplamiento de impedancia. El valor absoluto de la impedancia es prácticamente constante en función de la frecuencia o de la posición del alimentador, pero su fase puede variar rápidamente debido al viaje de ida y vuelta del alimentador al reflector y de regreso a éste. Un método para evitar este problema de impedancia, así como la sombra producida por el alimentador es  desplazar éste como se ilustra en la figura \ref{fig:offset}. En este tipo de antena para todos los fines prácticos, el alimentador queda fuera de la onda reflejada.
\begin{figure}[H]
\centering
\includegraphics[width=0.6\linewidth]{ant/ant51.png}
\caption{Antena parabólica con foco desplazado: \textit{offset}.}
\label{fig:offset}
\end{figure}
\begin{definition}[Ganancia de una antena parabólica]
La ganancia teórica de una antena parabólica de abertura circular excitada uniformemente está dada por:
\begin{equation}
G=\eta\parentesis{\frac{\pi\cdot D}{\lambda}}^2
\label{eq:ganancia para}
\end{equation}
Donde:
\begin{itemize}
\item \textbf{G}:Ganancia
\item \textbf{D}: Diámetro de la antena. (m)
\item $\lambda$: Longitud de onda. (m)
\item $\eta$: Eficiencia
\end{itemize}
\end{definition}
En la eficiencia de cualquier antena y, en particular de las antenas parabólicas intervienen diversos factores como los mencionados en párrafos anteriores y otros como las pérdidas óhmicas, la dispersión en los bordes, la obstrucción y dispersión por los diversos elementos estructurales que soportan el alimentador o el subrreflector en el caso de antenas Cassegrain que se tratarán más adelante, rugosidad de la superficie reflectora, etc.
\begin{definition}[Directividad]
\begin{equation}
D=4\pi\parentesis{\frac{L}{\lambda}}^2
\label{eq: direc para}
\end{equation}
Donde:
\begin{itemize}
\item \textbf{D}: Directividad.
\item \textbf{L}: Abertura de lado L. (m)
\item $\lambda$: Longitud de onda. (m)
\end{itemize}
\end{definition}
\begin{notation}
No confundir la letra D de las ecuaciones \ref{eq:ganancia para} t \ref{eq: direc para}, en la primera representa distancia y en la segunda representa directividad.
\end{notation}
\subsection{Antenas \textit{offset}}
Antenas en las que la estructura de soporte del alimentador no presenta obstrucción significativa al haz reflejado por el paraboloide como se ilustra en la figura \ref{fig:geo antena offset}. Aunque hay cierta ambigüedad en el uso del término offset en la ingeniería de antenas, aquí entenderemos que una antena offset es aquella que no es simétrica respecto al eje de revolución, ya que se descarta la porción de la superficie reflectora situada a un lado del eje. Como el alimentador debe estar localizado sobre el eje o muy cerca de él, en la antena offset se desplaza al alimentador de la región de máxima abertura, reduciendo o eliminando el bloqueo. Desde luego, el eje del alimentador debe desplazarse verticalmente de modo que el haz transmitido por él incida sobre la superficie de la porción reflectora del paraboloide, ya que otro modo se produce un desborde excesivo en los bordes del reflector.
\begin{figure}[H]
\centering
\includegraphics[width=0.6\linewidth]{ant/ant52.png}
\caption{Geometría de una antena offset.}
\label{fig:geo antena offset}
\end{figure}
\begin{definition}[BDF]
La relación entre el ángulo del haz y el del alimentador se designa como factor de desviación del haz (BDF) y se puede calcular como:
\begin{equation}
BDF=\frac{\sin^{-1}\corchetes{\frac{d}{f}\cdot\frac{1+k\parentesis{\frac{D}{4f}}^2}{1+\parentesis{\frac{D}{4f}}^2}}}{\tan^{-1}\parentesis{\frac{d}{f}}}
\end{equation}
Donde:
\begin{itemize}
\item \textbf{d}: Distancia entre la posición del alimentador en el eje horizontal perpendicular al del paraboloide. (m)
\item \textbf{D}: Diámetro del paraboloide. (m)
\item \textbf{f}: Distancia focal. (m)
\item \textbf{k}: Constante menor que 1.
\end{itemize}
\end{definition}
\subsection{Antenas con doble reflector: Cassegrain}
\subsubsection{Antenas doble reflector}
Las antenas con doble reflector están constituidas por dos reflectores, uno principal parabólico y otro secundario, en la forma que se ilustra esquemáticamente en la figura \ref{fig:doble reflec}.
\begin{figure}[H]
\centering
\includegraphics[width=0.6\linewidth]{ant/ant53.png}
\caption{Geometría básica de una antena de doble reflector.}
\label{fig:doble reflec}
\end{figure}
El subreflector suele ser hiperbólico en cuyo caso la antena se designa como Cassegrain16 o bien elíptico y la antena se designa como gregoriana17. En la primera, el hiperboloide suele presentar la parte convexa hacia el reflector principal como en el caso de la figura \ref{fig:ant par} y, en la gregoriana, el elipsoide reflector suele presentar la parte cóncava. En algunos casos se emplean también subreflectores planos o esféricos. Estas antenas se utilizan extensamente en comunicaciones espaciales y radioastronomía, además de comunicaciones terrestres. Este
tipo de antenas ofrece algunas ventajas sobre las antenas de un solo reflector y, aunque pueden tener diseños diferentes, comparten un conjunto de aspectos básicos comunes. Una de
las ventajas es que el alimentador de la antena no requiere de una línea de transmisión larga y se conecta casi directamente a la salida del transmisor o a la entrada del receptor reduciendo considerablemente las pérdidas. Si bien el bloqueo por la estructura de soporte no puede eliminarse en el caso de la geometría de la figura \ref{fig:ant par}, la eficiencia de las antenas de doble reflector en general es superior a la de las de reflector simple llegando aproximadamente al
70\% o más. Su ganancia se calcula de la misma manera que la una antena parabólica simple, utilizando la ecuación \ref{eq:ganancia para}.
\subsubsection{Antena Cassegrain}
Un telescopio Cassegrain consiste de dos espejos y un instrumento óptico de observación. El espejo primario es grande y cóncavo y refleja la luz incidente hacia un espejo secundario convexo y más pequeño, frente al espejo primario. Este espejo secundario refleja a su vez la luz hacia el centro del espejo primario en el que sitúa el observador o, en al caso de una antena, el receptor, como se ilustra esquemáticamente en la figura \ref{fig:doble reflec}.\\
El subreflector debe ser lo suficientemente grande como para interceptar la porción útil de la radiación del alimentador y refleja esta onda sobre el reflector primario de acuerdo a las leyes de la óptica. En la geometría clásica de la antena Cassegrain se emplea un paraboloide como reflector primario o principal y un hiperboloide para el reflector secundario en que uno de los dos focos de la hipérbola es el punto focal real del sistema y está localizado en el centro del alimentador. El otro es un foco virtual que se localiza en el foco de la parábola. Como resultado, todas las partes de la onda originada en el foco real y luego que luego son reflejadas por ambas superficies, viajan distancias iguales hasta el plano de la abertura frente a la antena.
\begin{figure}[H]
\centering
\includegraphics[width=0.4\linewidth]{ant/ant54.png}
\caption{Geometría antena Cassegrain.}
\end{figure}
\section{Antenas Parche}
Microstrip o antenas de parches son cada vez más útiles porque se pueden imprimir directamente sobre una placa de circuito. Ellos son cada vez más generalizada en el mercado de la telefonía móvil. Son de bajo coste, tienen un perfil bajo y se fabrican fácilmente. La tendencia a la miniaturización al lograr dispositivos cada vez mas pequeños, de fácil producción en masa (costos reducidos), fáciles de adaptar con circuitos integrados de microondas, versátiles en términos de impedancia, patrón, polarización y frecuencia de resonancia. Dentro de las desventajas podemos encontrar la baja potencia de radiación (por su estructura no se puede soportar grandes potencias en su estructura), baja eficiencia, ancho de banda angosto para ciertos diseños, considerables perdidas y son fácilmente afectadas por el factor térmico (sobre todo si se trabaja en sustratos flexibles).
\subsection{Características}
\begin{figure}[H]
\centering
\subfloat[Vista superior]{\includegraphics[width=0.4\linewidth]{ant/ant35.png}}
\subfloat[Vista lateral]{\includegraphics[width=0.4\linewidth]{ant/ant36.png}}
\caption{Geometría de microstrip.}
\end{figure}
Las antenas tipo parche poseen una tira conductora de largo \textit{L}, ancho \textit{W} y grosor \textit{T}. La tira conductora esta en la parte superior del substrato dieléctrico el cual tiene un ancho \textit{H}. En la parte inferior del substrato dieléctrico esta la tierra, como se ve en la figura.
donde:
\begin{itemize}
\item T y H z<<$\lambda$  ($\lambda$ es la longitud de onda en el espacio libre).
\item $\lambda$/3 <L< $\lambda$/2
\item 2.2 $\leq\epsilon_r\leq$ 12 (Se debe buscar la menor permitividad posible para lograr una mejor eficiencia de la antena).
\end{itemize}
Dependiendo de los requerimientos específicos para los cuales se construye una antena de microstrip de un solo elemento, se pueden recurrir a varios tipos de configuraciones los mas típicos son:
\begin{itemize}
\item Dipolo (tanto en su forma de media onda como de onda completa)
\item Cuadrada, rectangular, pentagonal, triangular, circular, disco con ranura, sector de disco, anillo, semi-disco, anillo elíptico, espiral.
\item Otro tipo particular de antena parche que ha surgido en años recientes es la llamada ``Antena  f  invertida plana'' muy utilizada para unidades móviles (es la mitad de una antena parche cuadrada).
\item Se pueden usar arreglos de antenas con el fin de lograr características deseadas 
\item El patrón de ondas de una antena parche es omnidireccional aunque la potencia radiada es emitida  solamente hacia la parte superior en su forma ideal.
\end{itemize}
\begin{figure}[]
\centering
\subfloat[Patrón de radiación 2D]{\includegraphics[width=0.4\linewidth]{ant/ant37.png}}
\subfloat[Patrón de radiación 3D]{\includegraphics[width=0.4\linewidth]{ant/ant38.png}}
\caption{Patrón de radiación de una antena microstrip.}
\end{figure}
\subsection{Métodos de análisis}
Dependiendo de la sencillez y la precisión que se busque se puede seleccionar el método que mas se ajuste a las necesidades. Entre los diversos métodos para el análisis de antenas tipo parche se pueden mencionar:
\begin{itemize}
\item Modelos empíricos.
\item Modelos semi-empíricos.
\item Modelos de onda completa.
\end{itemize}
\subsubsection{Modelos empíricos}
Son los menos precisos a la hora de diseñar, su método de análisis se basa en la suposición de conceptos. Estos modelos pueden tener buen nivel de precisión cuando se trabajan en rango de frecuencias menores a los de las ondas milimétricas (f < 30GHz) pero si se salen de este rango el modelo comienza a tener graves imprecisiones por lo que es necesario usar otros modelos. Aquí se encuentra el modelo de línea de transmisión.\\
\textbf{Modelo líneas de transmisión}\\
Este modelo es el mas sencillo pero el mas impreciso además de que solamente puede ser utilizado para el diseño de antenas rectangulares y circulares. Este modelo considera los bordes de la antena como dos aperturas (slots) que radian.
Las aperturas a su vez son consideradas como admitancias complejas compuestas de una conductancia G y una susceptancia B, la siguiente figura muestra el circuito equivalente para una antena rectangular en el modelo de línea de transmisión.
\begin{figure}[H]
\centering
\includegraphics[width=0.6\linewidth]{ant/ant39.png}
\caption{Circuito antena rectangular.}
\end{figure}
\begin{definition}[Fringing Effects]
El efecto de franja (fringing effects) se debe a las líneas de campo eléctrico que hacen que el tamaño de la antena sea más ancho después de la excitación. La causa principal del efecto de franja se debe al ancho y la posición de la antena de alimentación. Los campos producidos por los fringing effects ayudan en la radiación y dan un patrón de radiación más directivo, pero la frecuencia de resonancia se desplaza de la frecuencia deseada.
\begin{center}
\includegraphics[width=0.7\linewidth]{ant/ant40.png}
\end{center}
\end{definition}
El modelo de línea de transmisión se resumen en los siguientes pasos:
\begin{enumerate}
\item Se especifica la frecuencia de operación y el sustrato a utilizar.
\item Se obtiene el ancho efectivo de la antena de parche rectangular mediante la formula:
\begin{displaymath}
W=\frac{1}{2f_r\sqrt{\mu_0\epsilon_0}}\sqrt{\frac{2}{\epsilon_r+1}}=\frac{c}{2f_r}\sqrt{\frac{2}{\epsilon_r+1}}
\end{displaymath}
\item Se obtiene la permitividad eléctrica efectiva mediante:
\begin{displaymath}
\epsilon_{ref}=\frac{\epsilon_r+1}{2}+\frac{\epsilon_r+1}{2}\corchetes{1+12\frac{h}{W}}^{-\frac{1}{2}}
\end{displaymath}
\item Se obtiene la extensión $\Delta L$ mediante la ecuación que se derivará en la obtención de la longitud real de la antena considerando la longitud efectiva.
\begin{displaymath}
\Delta L=0.412\cdot h\frac{\parentesis{\epsilon_{ref}+0.3}\parentesis{\frac{W}{h}+0.264}}{\parentesis{\epsilon_{ref}-0.258}\parentesis{\frac{W}{h}+0.8}}
\end{displaymath}
\item Longitud será:
\begin{displaymath}
L=\frac{1}{2\cdot f_r\sqrt{\epsilon_{ref}}\sqrt{\mu_0\epsilon_0}}-2\Delta L
\end{displaymath}
\end{enumerate}
\subsubsection{Modelo semi-empíricos}
Es un modelo mas complejo que el empírico pero menos preciso que el modelo de onda compleja. Ejemplos de este modelo se pueden nombrar los siguientes:
\begin{itemize}
\item Modelo de corriente superficial eléctrica.
\item Técnica de la transformada de Hankel.
\item Método de reciprocidad.
\item Enfoque de ecuación integral dual, etc.
\end{itemize}

\subsubsection{Modelos de onda completa}
Estos modelos se presentan como los mas precisos a la hora de diseñar, sin embargo también son los mas complicados de realizar y se requieren de herramientas computacionales avanzadas para llevar a cabo.
Como ejemplos de modelo de onda completa se tiene:
\begin{itemize}
\item Método de momentos en el dominio del tiempo.
\item Método de momentos en el dominio espectral.
\item Método de estados finitos.
\item Técnicas de transformada rápida de Fourier, etc.
\end{itemize}
\begin{notation}
Ejemplos de programas que usan el modelo de onda completa son el programa Sonnet y el programa HFSS. El software Sonnet usa el método de momentos, mientras que el software HFSS usa un método de elementos finitos modificado.
\end{notation}
\subsection{Alimentación}
Existen diferentes métodos para alimentar una antena microstrip de forma que radie lo mas eficientemente posible.
Por ejemplo esta el método de \textbf{alimentación por sonda coaxial} donde el pin de cable coaxial alimenta directamente al radiador, mientras que la parte negativa de este se conecta a la de tierra de la antena de microstrip.\\
Otro método de alimentación es el llamado \textbf{alimentación con Substrato de Guía de Onda Integrado} (Substrate Integrated Waveguide), con la que se tiene una antena (elemento radiador), en la parte superior de un substrato dieléctrico que recibe la alimentación de una guía de onda integrada en otro substrato dieléctrico localizada en la parte inferior de la antena. El acoplamiento electromagnético es através de una abertura localizada en la parte superior de la guía de onda y cuyas dimensiones y posicionamiento generan diferentes impedancias características y por lo tanto diferente acoplamiento como se muestra en la siguiente vista superior de la estructura para alimentar a la antena:
\begin{figure}[H]
\centering
\includegraphics[width=0.7\linewidth]{ant/ant41.png}
\caption{Formas de alimentación.}
\end{figure}
\section{Arreglo o arrays de antenas}
Un array es una antena compuesta por un número de radiadores idénticos ordenados regularmente y alimentados para obtener un diagrama de radiación predefinido.\\
Hay diferentes tipos de arrays. 
\begin{itemize}
\item Los arrays lineales tienen los elementos dispuestos sobre una línea.
\item Los arrays planos son agrupaciones bidimensionales cuyos elementos están sobre un plano.
\item Los arrays conformados tienen las antenas sobre una superficie
curva, como por ejemplo el fuselaje de un avión
\end{itemize}
Los arrays tienen la ventaja de que se puede controlar la amplitud de las corrientes y la fase de cada elemento, modificando la forma del diagrama de radiación. Además se puede conseguir que los parámetros de antena dependan de la señal recibida a través de circuitos asociados a los elementos radiantes, como en el caso de las agrupaciones adaptativas.\\
\begin{notation}
El factor de array es el diagrama de radiación de una agrupación de
elementos isotrópicos.
\end{notation}
Cuando los diagramas de radiación de cada elemento del array son iguales y los elementos están orientados en la misma dirección del espacio, el diagrama de radiación de la agrupación se puede obtener como el producto del factor de array por el diagrama de radiación del elemento.\\
Hay 5 métodos para controlar el diagrama de radiación de un array:
\begin{enumerate}
\item Mediante la configuración geométrica (linear, rectangular, circular…)
\item La situación relativa de los elementos
\item La amplitud de la excitación de cada elemento
\item La fase de la excitación de cada elemento
\item El diagrama de radiación de cada elemento
\end{enumerate}
\subsection{Arrays lineales}
Dos radiadores iguales, alimentados con misma amplitud y fase. Ambos radiadores producen ondas esféricas que se sumarán de forma
constructiva en determinadas direcciones y cancelación en otras
%imagen
En el caso de que estemos suficientemente lejos de las fuentes, y suponiendo que el primer radiador se encuentra en el origen de coordenadas, la diferencia de caminos recorrida por ambas ondas será:
\begin{equation}
R_1-R_2=d\cdot\cos\theta
\end{equation}
Se podrá escribir la amplitud total de la señal como el producto de una onda esférica por un factor de interferencia.
\begin{equation}
\frac{e^{-jkr}}{4\pi\cdot r}\parentesis{1+e^{jkd\cos\theta}}
\end{equation}
Se producirá \textbf{interferencia constructiva} cuando la diferencia de caminos sea un \textbf{múltiplo entero de longitudes de onda}, siendo la amplitud de la señal el doble. Cuando la diferencia de fase sea un\textbf{ múltiplo impar de $\pi$}, la interferencia será destructiva.
\begin{tabular}{k{0.5\linewidth}  j{0.5\linewidth}}
        \includegraphics[width=0.7\linewidth]{ant/ant42.png} & \textbf{d=0}\newline Si la separación entre los dos
radiadores es cero, no existirá
ningún tipo de desfase, por lo que
la señal se radiará isotrópicamente
en todas las direcciones del
espacio. \\
        \includegraphics[width=0.7\linewidth]{ant/ant43.png} & \textbf{d=$\lambda/2$}\newline Si los dos elementos están separados una semilongitud de onda, se
producirá un máximo en la dirección perpendicular a la recta que
une sus posiciones, obteniendo un nulo de radiación en la dirección
de dicho eje, ya que las señales se sumarán en oposición de fase.\\
        \includegraphics[width=0.7\linewidth]{ant/ant44.png} & \textbf{d=$\lambda$}\newline Si la separación entre los dos radiadores es de una longitud de onda
se producirán máximos de radiación en las direcciones del eje y
perpendiculares a él, produciéndose cancelación para un ángulo en
el que ambas señales esté en oposición de fase, lo que sucede para la
dirección que forma un ángulo de 60° con el eje de la agrupación. \\
        \includegraphics[width=0.7\linewidth]{ant/ant45.png} & \textbf{d=$\lambda/4$}\newline Cuando la diferencia de fases es de $-\pi/4$ y la separación es $\lambda/4$, las
ondas se suman en la dirección del eje z en fase, y en –z en oposición
de fase. 
\end{tabular}
\begin{figure}[H]
\centering
\includegraphics[width=\linewidth]{ant/ant46.png}
\caption{Diagramas tridimensionales de dos radiadores isotrópicos de la misma amplitud, en función de
la separación \textit{d} y de la diferencia de fases $\alpha$.}
\end{figure}
%----------------------------------------------------------------------------------------
%	NEW CHAPTER
%----------------------------------------------------------------------------------------
\part{Internetworking 2}
\chapterimage{chapter_head_IT2.pdf} % Chapter heading image

\chapter{Unidad I}
\section{Protocolo spanning tree}\index{Protocolo spanning tree}
Las redes conmutadas, por lo general, tienen rutas  redundantes y enlaces redundantes incluso entre los  mismos dos dispositivos. Las rutas redundantes eliminan un punto de falla único  para mejorar la confiabilidad y disponibilidad. Las rutas redundantes pueden causar bucles de capa 2  físicos y lógicos.
\begin{figure}[H]
\centering
\includegraphics[width=0.6\linewidth]{IN2/IN1.png}
\caption{Camino redundante de PC1 a PC4 en caso falle una ruta.}
\end{figure}
El algoritmo \textbf{Spanning Tree} (árbol de expansión) se utiliza en los switches para prevenir los bucles lógicos que pueden aparecer en una red. Los bucles se producen cuando existen varios caminos distintos entre dos puntos de la red y su efecto es que las tramas pueden circular de forma indefinida atrapadas en un bucle sin conseguir alcanzar su destino, lo que además afectará negativamente al rendimiento de la red. El algoritmo \textit{Spanning Tree} ayuda a los switches a elegir el camino más idóneo y, por tanto, elimina los bucles.
\begin{notation}
El protocolo \textit{spanning tree esta detallado especificado en el estándar \textbf{IEEE 802.1D}. Existe su variante con funcionamiento optimizado: \textbf{spanning tree rápido}-IEEE 802.1w}
\end{notation}
\subsection{Problemas con la redundancia de capa 1}
Los problemas con los bucles son:
\begin{itemize}
\item \textbf{Inestabilidad de la base de datos MAC}: Las tramas de ethernet no poseen Tiempo de duración, como el encabezado IP de capa 3. Esto significa que Ethernet no tiene ni un mecanismo para descartar las tramas que se propagan innecesariamente. EL bucle empieza de la siguiente manera; observando la imagen  \ref{fig:inestabilidad spanning tree}:
\begin{enumerate}
\item PC1 envía un marco de transmisión a S2. S2 recibe el marco de transmisión en F0/11. Cuando S2 recibe la trama de transmisión, actualiza su tabla de direcciones MAC para registrar que la PC1 está disponible en el puerto F0/11.
\item Debido a que es una trama de transmisión, S2 reenvía la trama por todos los puertos, incluidos Trunk1 y Trunk2. Cuando la trama de transmisión llega a S3 y S1, los switches actualizan sus tablas de direcciones MAC para indicar que PC1 está disponible en el puerto F0/1 en S1 y en el puerto F0/2 en S3.
\item Debido a que es una trama de transmisión, S3 y S1 reenvían la trama por todos los puertos excepto el puerto de entrada. S3 envía el marco de transmisión de PC1 a S1. S1 envía el marco de transmisión de PC1 a S3. Cada conmutador actualiza su tabla de direcciones MAC con el puerto incorrecto para PC1.
\item Cada conmutador reenvía la trama de difusión por todos sus puertos excepto el puerto de entrada, lo que hace que ambos conmutadores reenvíen la trama a S2.
\item Cuando S2 recibe las tramas de transmisión de S3 y S1, la tabla de direcciones MAC se actualiza con la última entrada recibida de los otros dos conmutadores.
\item S2 reenvía la trama de transmisión por todos los puertos excepto el último puerto recibido. El ciclo comienza de nuevo.
\end{enumerate}
\begin{figure}[H]
\centering
\includegraphics[width=0.6\linewidth]{IN2/IN2.png}
\caption{Momento 6 del bucle.}
\label{fig:inestabilidad spanning tree}
\end{figure}
Debido a los cambios constantes en la tabla de direcciones MAC, los switches S3  y S1 no saben a qué puerto reenviar las tramas.
\item \textbf{Tormenta de difusión}: Si no controlamos el problema anterior, las demás computadoras también deben mandar primero sus marcos de transmisión, y el problema de inestabilidad se acrecentará, esto se refleja en el sistema como varios tramas de transmisión \textbf{atrapados} en la red, \textbf{consumiendo ancho de banda} y hasta saturándola. Haciendo que sea imposible el envío de datos legítimos de comunicación. Si no hay ancho de banda útil para el envío de información, la red no estará disponible causando una denegación de servicio efectiva (DoS).\\
Hay otras consecuencias de las tormentas de transmisión. Debido a que el tráfico de difusión se reenvía a todos los puertos de un conmutador, todos los dispositivos conectados tienen que procesar todo el tráfico de difusión que se inunda sin cesar alrededor de la red en bucle. Esto puede hacer que el dispositivo final no funcione correctamente debido a los requisitos de procesamiento necesarios para soportar una carga de tráfico tan alta en la NIC.\\
Secuencia de eventos:
\begin{enumerate}
\item PC1 envía una trama de transmisión a la red en bucle (Inestabilidad en la base de datos MAC).
\item La trama de transmisión se repite entre todos los conmutadores interconectados en la red.
\item La PC4 también envía una trama de transmisión a la red en bucle.
\item La trama de transmisión PC4 queda atrapada en el bucle entre todos los conmutadores interconectados, al igual que la trama de transmisión PC1.
\item A medida que más dispositivos envían transmisiones a través de la red, más tráfico queda atrapado en el bucle y consume recursos. Esto eventualmente crea una tormenta de transmisión que hace que la red falle.
\item Cuando la red está totalmente saturada con el tráfico de difusión que circula entre los conmutadores, el conmutador descarta el tráfico nuevo porque no puede procesarlo. La Figura \ref{fig:tormenta de difusión} muestra la tormenta de difusión resultante.
\end{enumerate}
\begin{figure}[H]
\centering
\includegraphics[width=0.6\linewidth]{IN2/IN3.png}
\caption{Tormenta de difusión}
\label{fig:tormenta de difusión}
\end{figure}
Una tormenta de transmisión puede desarrollarse en segundos porque los dispositivos conectados a una red envían regularmente tramas de transmisión, como solicitudes ARP. Como resultado, cuando se crea un bucle, la red conmutada se desconecta rápidamente.
\item \textbf{Tramas de unidifusión duplicadas}: Una trama de unidifusión desconocida se produce  cuando el switch no tiene la dirección MAC de destino en la tabla de direcciones MAC y debe difundir la trama a todos los puertos, excepto el  puerto en que se recibió la trama (puerto de ingreso). Si se envían tramas de unidifusión desconocidas a  una red con bucles, se puede producir la llegada de  tramas duplicadas al dispositivo de destino.\\
Secuencia de eventos:
\begin{enumerate}
\item PC1 envía una trama de unidifusión destinada a PC4.
\item S2 no tiene una entrada para PC4 en su tabla MAC. En un intento por encontrar la PC4, inunda la trama de unidifusión desconocida de todos los puertos del switch excepto el puerto que recibió el tráfico.
\item La trama llega a los conmutadores S1 y S3.
\item S1 tiene una entrada de dirección MAC para PC4, por lo que reenvía la trama a PC4.
\item S3 tiene una entrada en su tabla de direcciones MAC para PC4, por lo que reenvía la trama de unidifusión de Trunk3 a S1.
\item S1 recibe la trama duplicada y la reenvía a la PC4.
\item PC4 ahora ha recibido el mismo marco dos veces.
\end{enumerate}
\begin{figure}[H]
\centering
\includegraphics[width=0.6\linewidth]{IN2/IN4.png}
\caption{S1 y S3 envian una trama duplicada a la PC4.}
\end{figure}
\end{itemize}
\section{Tecnología EtherChannel}
\subsection{Funcionamiento}
Hay escenarios en los que se necesita más ancho de banda o redundancia entre  dispositivos que lo que puede proporcionar un único enlace. Se pueden conectar varios  enlaces entre dispositivos para aumentar el ancho de banda. Sin embargo, el protocolo de  árbol de expansión (STP), que está habilitado en dispositivos de \textbf{capa 2} como switches. Cisco de forma predeterminada, bloqueará enlaces redundantes para evitar bucles de  conmutación. Se necesita una tecnología de agregación de enlaces que permita \textbf{vínculos redundantes} entre dispositivos que \textbf{no serán bloqueados por STP}. Esa tecnología se conoce como EtherChannel.\\
EtherChannel es una tecnología de agregación de enlaces que \textbf{agrupa} varios enlaces físicos Ethernet en un único enlace lógico. Se utiliza para proporcionar tolerancia a fallos, uso compartido de carga, mayor ancho de banda y redundancia entre switches, routers y servidores. La tecnología de EtherChannel hace posible combinar la cantidad de enlaces físicos entre los switches para aumentar la velocidad general de la comunicación switch a switch.\\
En los inicios, Cisco desarrolló la  tecnología EtherChannel como una técnica switch a switch LAN  para \textbf{agrupar} varios puertos Fast  Ethernet o gigabit Ethernet en un  único canal lógico. Cuando se configura un EtherChannel, la interfaz virtual  resultante se denomina ``canal de  puertos'' (\textit{port channel}). Las interfaces físicas se agrupan en una interfaz de canal  de puertos, como se muestra en la figura \ref{fig:tec ether}.
\begin{figure}[H]
\centering
\includegraphics[width=0.6\linewidth]{IN2/IN5.png}
\caption{Tecnología EtherChannel}
\label{fig:tec ether}
\end{figure}
\subsection{Ventajas de EtherChannel}
\begin{itemize}
\item La mayoría de las tareas de configuración se pueden realizar en la interfaz EtherChannel en lugar  de en cada puerto individual, lo que asegura la coherencia de configuración en todos los enlaces.
\item EtherChannel depende de los puertos de switch existentes. No es necesario actualizar el enlace a  una conexión más rápida y más costosa para tener más ancho de banda.
\item El equilibrio de carga ocurre entre los enlaces que forman parte del mismo EtherChannel.
\item EtherChannel crea una agregación que se ve como un único enlace lógico. Cuando existen varios  grupos EtherChannel entre dos switches, STP puede bloquear uno de los grupos para evitar los  bucles de switching. Cuando STP bloquea uno de los enlaces redundantes, bloquea el  EtherChannel completo. Esto bloquea todos los puertos que pertenecen a ese enlace  EtherChannel. Donde solo existe un único enlace EtherChannel, todos los enlaces físicos en el  EtherChannel están activos, ya que STP solo ve un único enlace (lógico).
\item EtherChannel proporciona redundancia, ya que el enlace general se ve como una única conexión  lógica. Además, la pérdida de un enlace físico dentro del canal no crea ningún cambio en la  topología.
\end{itemize}
\subsection{Restricciones}
\begin{itemize}
\item No pueden mezclarse los tipos de interfaz. Por ejemplo, Fast Ethernet y Gigabit Ethernet no se pueden mezclar dentro de un único EtherChannel.
\item En la actualidad, cada EtherChannel puede constar de hasta ocho puertos Ethernet configurados  de manera compatible. El EtherChannel proporciona un ancho de banda full-duplex de hasta 800  Mbps (Fast EtherChannel) u 8 Gbps (Gigabit EtherChannel) entre un switch y otro switch o host.
\item El switch Cisco Catalyst 2960 Layer 2 soporta actualmente hasta seis EtherChannels.
\item La configuración de los puertos individuales que forman parte del grupo EtherChannel debe ser  coherente en ambos dispositivos. Si los puertos físicos de un lado se configuran como enlaces  troncales, los puertos físicos del otro lado también se deben configurar como enlaces troncales  dentro de la misma VLAN nativa. Además, todos los puertos en cada enlace EtherChannel se  deben configurar como puertos de capa 2.
\item Cada EtherChannel tiene una interfaz de canal de puertos lógica, la configuración aplicada a la  interfaz de canal de puertos afecta a todas las interfaces físicas que se asignan a esa interfaz.
\end{itemize}
\subsection*{Protocolos de negociación automática}
Los EtherChannels se pueden formar por medio de una negociación con uno de dos  protocolos: Port Aggregation Protocol (PAgP) o Link Aggregation Control Protocol (LACP).  Estos protocolos permiten que los puertos con características similares formen un canal  mediante una negociación dinámica con los switches adyacentes.
\subsection{Funcionamiento del PAgP}
PAgP (pronunciado ``Pag - P'') es un protocolo patentado por Cisco que ayuda en la creación  automática de enlaces EtherChannel. Cuando se configura un enlace EtherChannel mediante PAgP,  se envían paquetes PAgP entre los puertos aptos para EtherChannel para negociar la formación de un  canal. Cuando PAgP identifica enlaces Ethernet compatibles, agrupa los enlaces en un EtherChannel.  El EtherChannel después se agrega al árbol de expansión como un único puerto.\\
Cuando se habilita, PAgP también administra el EtherChannel. Los paquetes PAgP se envían cada 30  segundos. PAgP revisa la coherencia de la configuración y administra los enlaces que se agregan, así  como las fallas entre dos switches. Cuando se crea un EtherChannel, asegura que todos los puertos  tengan el mismo tipo de configuración.
\begin{notation}
En EtherChannel, es obligatorio que todos los puertos tengan la misma velocidad, la misma  configuración de dúplex y la misma información de VLAN. Cualquier modificación de los puertos  después de la creación del canal también modifica a los demás puertos del canal.
\end{notation}
\subsection{PAgP}\index{PAgP}
PAgP ayuda a crear el enlace EtherChannel al detectar la configuración de cada lado y asegurarse de que los enlaces sean  compatibles, de modo que se pueda habilitar el enlace EtherChannel cuando sea necesario. Los modos de PAgP de la siguiente  manera:
\begin{itemize}
\item \textbf{Encendido}: Este modo obliga a la interfaz a proporcionar un canal sin PAgP. Las interfaces configuradas en el modo encendido no intercambian paquetes PAgP.
\item \textbf{PAgP desirable}: Este modo PAgP coloca una interfaz en un estado de negociación activa en el que la interfaz inicia negociaciones con otras interfaces al enviar paquetes PAgP.
\item \textbf{PAgP auto}: Este modo PAgP coloca una interfaz en un estado de negociación pasiva en el que la interfaz responde a los paquetes PAgP que recibe, pero no inicia la negociación PAgP.
\end{itemize}
Los modos deben ser compatibles en cada lado. Si se configura un lado en modo automático, se coloca en estado pasivo, a la  espera de que el otro lado inicie la negociación del EtherChannel. Si el otro lado se establece en modo automático, la  negociación nunca se inicia y no se forma el canal EtherChannel. Si se deshabilitan todos los modos mediante el comando \textit{\textbf{no}} o  si no se configura ningún modo, entonces se deshabilita el EtherChannel. El modo \textit{encendido} coloca manualmente la interfaz en  un EtherChannel, sin ninguna negociación. Funciona solo si el otro lado también se establece en modo encendido. Si el otro lado  se establece para negociar los parámetros a través de PAgP, no se forma ningún EtherChannel, ya que el lado que se establece  en modo \textit{encendido} no negocia. El hecho de que no haya negociación entre los dos switches significa que no hay un control, para asegurarse de que todos los enlaces en el EtherChannel terminen del otro lado o de que haya compatibilidad con PAgP en  el otro switch.
\begin{table}[H]
\centering
\begin{tabular}{|c|c|c|}
\hline
\rowcolor[HTML]{9698ED} 
S1        & S2             & Establecmiento del canal \\ \hline
On        & On             & Sí                       \\ \hline
On        & Desirable/Auto & No                       \\ \hline
Desirable & Desirable      & Sí                       \\ \hline
Desirable & Auto           & Sí                       \\ \hline
Auto      & Desirable      & Sí                       \\ \hline
Auto      & Auto           & No                       \\ \hline
\end{tabular}
\caption{Combinacion PAgP y el resultado del establecimiento.}
\end{table}
\subsection{LACP}\index{LACP}
LACP forma parte de una especificación \textbf{IEEE (802.3ad)} que permite agrupar varios puertos físicos  para formar un único canal lógico. LACP permite que un switch negocie un grupo automático mediante  el envío de paquetes LACP al otro switch. Realiza una función similar a PAgP con EtherChannel de  Cisco. Debido a que LACP es un estándar IEEE, se puede usar para facilitar los EtherChannels en entornos de varios proveedores. En los dispositivos de Cisco, se admiten ambos protocolos.\\
LACP proporciona los mismos beneficios de negociación que PAgP. LACP ayuda a crear el enlace  EtherChannel al detectar la configuración de cada lado y al asegurarse de que sean compatibles, de  modo que se pueda habilitar el enlace EtherChannel cuando sea necesario. Los modos para LACP  son los siguientes:
\begin{itemize}
\item \textbf{On}: Este modo obliga a la interfaz a proporcionar un canal sin LACP. Las interfaces
configuradas en el modo encendido no intercambian paquetes LACP.
\item \textbf{LACP active}: Este modo de LACP coloca un puerto en estado de negociación activa. En este
estado, el puerto inicia negociaciones con otros puertos mediante el envío de paquetes LACP.
\item \textbf{LACP passive}: Este modo de LACP coloca un puerto en estado de negociación pasiva. En este  estado, el puerto responde a los paquetes LACP que recibe, pero no inicia la negociación de  paquetes LACP.
\end{itemize}
\begin{table}[H]
\centering
\begin{tabular}{|c|c|c|}
\hline
\rowcolor[HTML]{9698ED} 
S1      & S2             & Establecmiento del canal \\ \hline
On      & On             & Sí                       \\ \hline
On      & Active/Passive & No                       \\ \hline
Active  & Active         & Sí                       \\ \hline
Active  & Passive        & Sí                       \\ \hline
Passive & Active         & Sí                       \\ \hline
Passive & Passive        & No                       \\ \hline
\end{tabular}
\caption{Diversas combinaciones de modos LACP en S1 y S2 y el resultado resultante del establecimiento de canales.}
\end{table}
\subsection{Pautas y restricciones para la configuración}
\begin{itemize}
\item \textbf{EtherChannel support}: Todas las interfaces Ethernet deben admitir EtherChannel,
sin necesidad de que las interfaces sean físicamente contiguas
\item \textbf{Speed and duplex}: Configure todas las interfaces en un EtherChannel para que  funcionen a la misma velocidad y en el mismo modo dúplex.
\item \textbf{VLAN match}: Todas las interfaces en el grupo EtherChannel se deben asignar a la  misma VLAN o se deben configurar como enlace troncal (mostrado en la figura).
\item \textbf{Rango de VLAN}: Un EtherChannel admite el mismo rango permitido de VLAN en  todas las interfaces de un EtherChannel de enlace troncal. Si el rango permitido de  VLAN no es el mismo, las interfaces no forman un EtherChannel, incluso si se  establecen en modo auto o desirable .
\end{itemize}
\section{Conceptos de routing}
El routing es la capacidad de buscar la ruta correcta para mover o transferir paquetes de información entre una o varias redes de Internet. Para hacer routing se necesitan de routers, estos son computadoras especializadas que poseen componentes como:
\begin{itemize}
\item Unidad central de procesamiento (CPU)
\item Sistema operativo (OS): los routers utilizan IOS de Cisco
\item Memoria y almacenamiento (RAM, ROM, NVRAM, flash, disco duro)
\item Tarjetas de interfaz de red
\end{itemize}
Estos se pueden enlazar mediante puertos para interconectarse con otras redes.
\begin{figure}[H]
\centering
\includegraphics[width=0.6\linewidth]{IN2/IN6.png}
\caption{Memorias del router}
\end{figure}
Más adelante se estudiará las ruta estática y los protocolos de routing dinámico para descubrir redes remotas y crear sus tablas de routing. La tabla de routing ayuda a determinar la mejor ruta para enviar paquetes. Los paquetes son encapsulados y se envían en cada interfaz indicada en la tabla de routing.
\begin{figure}
\centering
\includegraphics[width=\linewidth]{IN2/IN7.png}
\caption{Encapsulación de paquetes.}
\end{figure}
\subsection{Funciones de routing}
\begin{itemize}
\item \textbf{Conectar redes y administrar el tráfico de ellas}: El router es el responsable del routing, capacidad de reenviar paquetes IP desde una red a otra, del tráfico entre redes. Los routers encapsulan los paquetes y los reenvían a la interfaz indicada en la tabla de routing. Los métodos de reenvío de paquetes son:
\begin{itemize}
	\item Switching de procesos: es  un mecanismo de reenvío de  paquetes más antiguo que  todavía está disponible para  routers Cisco.
	\item Switching rápido: es un  mecanismo común de reenvío  de paquetes que usa una  memoria caché de switching  rápido para almacenar la información de siguiente salto.
	\item Cisco Express Forwarding: Es el mecanismo más reciente, más rápido y más utilizado de CISCO.
\end{itemize}
\item \textbf{Aceleración de datos entre redes}: Los routers son capaces de utilizar las tablas de routing para determinar cuál es la mejor ruta para enviar paquetes. Pueden emplear rutas estáticas o dinámicas para descubrir redes remotas.
\item \textbf{Creación de tabla de routing}: La tabla de routing es el listado de todas las posibles rutas de la red. Cuando un router recibe paquetes IP que deben enviarse a otro lugar de la red, mira la dirección IP de destino del paquete y, luego, busca la información de routing en la tabla de routing.
\item \textbf{Seguridad de red}: El router sirve como un firewall, lo que restringe el acceso de cualquier persona que no esté autorizada. Esto se hace con claves de seguridad similares a las contraseñas, sólo los dispositivos precargados con la clave de seguridad o usadas por usuarios que pueden ingresar la contraseña correcta, tienen acceso a la red
\end{itemize}
\subsection{Conexión de dispositivos}
Para conectar los dispositivos en los routers se debe hacer uso de:
\begin{itemize}
\item \textbf{Gateways predeterminadas}: Para habilitar el acceso a  la red, los dispositivos deben estar configurados con la siguiente información de  direcciones IP.
\begin{itemize}
\item \textbf{Dirección IP}: Identifica a un  host único en una red local.
\item \textbf{Máscara de subred}: Identifica a la subred de la  red del host.
\item \textbf{Gateway predeterminado}: Identifica al router al que se  envía un paquete cuando el  destino no está en la misma  subred de la red local.
\end{itemize}
\item \textbf{Habilitar IP en un host}: La dirección IP puede ser asignada en forma estática o dinámica.
\begin{itemize}
\item Estática: Manualmente. Al host se le asigna  manualmente una dirección IP, una máscara de subred y un gateway  predeterminado. También se puede asignar la dirección IP de un  servidor DNS. Se utiliza para identificar recursos de red específicos, como servidores de red e impresoras. Se puede utilizar en redes muy pequeñas con pocos hosts.
\item Dinámica: Un servidor asigna en forma dinámica la información de la dirección IP utilizando el \textbf{ protocolo de  configuración dinámica de hosts} (DHCP). La mayoría de los hosts obtienen la información de su dirección IP  mediante DHCP. Los routers Cisco pueden proporcionar servicios DHCP.
\end{itemize}
\item \textbf{Acceso a la consola}: 
\begin{itemize}
\item Puerto de computadora: Puerto serie y puerto USB tipo A
\item Cable requerido: Cable de consola, Adaptador de puerto serie USB a RS232 y Cable USB tipo A a USB tipo B.
\item Puerto en el ISR: Puerto de consola RJ-45 y Puerto de consola USB Tipo B
\item Emulación del terminal: Tera Term o PuTTY.
\end{itemize}
\item \textbf{Habilitar IP en in switch}
\end{itemize}
\subsection{Verificación de la conectividad de redes conectadas directamente}
Existen diferentes métodos para comprobar la conectividad entre dos redes conectadas:
\subsubsection{Prueba de stack}
\begin{enumerate}
\item \textbf{PING}: El comando ping es una manera efectiva de probar la conectividad. La prueba se denomina prueba de stack de protocolos, porque el comando ping se mueve desde la Capa 3 del Modelo OSI hasta la Capa 2 y luego hacia a la Capa. El ping utiliza el protocolo ICMP (Protocolo de mensajes de control de Internet) para comprobar la conectividad.
\item \textbf{Indicadores de ping IOS}: Un ping de IOS cederá a una de varias indicaciones para cada eco ICMP enviado. Los indicadores más comunes son:
! - indica la recepción de una respuesta de eco ICMP
. - indica un límite de tiempo cuando se espera una respuesta
U - se recibió un mensaje ICMP inalcanzable
\item \textbf{Prueba de loopback}: A modo de primer paso en la secuencia de prueba, se utiliza el comando ping para verificar la configuración IP interna en el host local. Recuerde que esta prueba se cumple con el comando ping en una dirección reservada denominada loopback (127.0.0.1). Esto verifica la correcta operación del stack de protocolos desde la capa de Red a la capa Física, y viceversa, sin colocar realmente una señal en el medio.
\end{enumerate}
\subsubsection{Asignación de interfaz}
Del mismo modo que se usan comandos y utilidades para verificar la configuración de un host, se deben aprender los comandos para verificar las interfaces de dispositivos intermediarios. El IOS provee comandos para verificar la operación de interfaces de router y switch.
\begin{enumerate}
\item \textbf{Verificación de interfaz de router}: Uno de los comandos más utilizados es el comando show ip interface brief. Este proporciona un resultado más abreviado que el comando show ip interface. Ofrece además un resumen de la información clave de todas las interfaces.
\item \textbf{Prueba red local}: La siguiente prueba de la secuencia corresponde a los hosts en la LAN local. Al hacer ping a los hosts remotos satisfactoriamente se verifica que tanto el host local (en este caso, el router) como el host remoto estén configurados correctamente. Esta prueba se realiza al hacer ping a cada host en forma individual en la LAN. Si un host responde con el mensaje "Destination Unreachable" (destino inalcanzable), observe qué dirección no fue satisfactoria y continúe haciendo ping a los otros hosts de la LAN. Otro mensaje de falla es "Request Timed Out" (la petición ha expirado). Indica que no hubo respuesta al intento del ping en el período de tiempo predeterminado, lo cual indica que el problema puede estar en la latencia de red.
\end{enumerate}
\subsection{Determinación de ruta}
\subsubsection{Función determinación de ruta}
La función de determinación de ruta es el proceso según el cual el router determina qué ruta usar cuando envía un paquete. Para determinar la mejor ruta, el router \textbf{busca} en su tabla de enrutamiento una dirección de red que coincida con la dirección IP de destino del paquete. El resultado de esta búsqueda es una de tres determinaciones de ruta:
\begin{itemize}
\item \textbf{Red conectada directamente}: si la dirección IP de destino del paquete pertenece a un dispositivo en una red que está directamente conectado a una de las interfaces del router, ese paquete se envía directamente a ese dispositivo. Esto significa que la dirección IP de destino del paquete es una dirección host en la misma red que la interfaz de este router.
\item \textbf{Red remota}: si la dirección IP de destino del paquete pertenece a una red remota, entonces el paquete se envía a otro router. Las redes remotas sólo se pueden alcanzar mediante el envío de paquetes a otro router.
\item \textbf{Sin determinación de ruta}: si la dirección IP de destino del paquete no pertenece ya sea a una red conectada o remota, y si el router no tiene una ruta por defecto, entonces el paquete se descarta. El router envía un mensaje ICMP de destino inalcanzable a la dirección IP de origen del paquete. En los primeros dos resultados, el router vuelve a encapsular el paquete IP en el formato de la trama de enlace de datos de Capa 2 de la interfaz de salida. El tipo de interfaz determina el tipo de encapsulación de Capa 2. Por ejemplo, si la interfaz de salida es FastEthernet, el paquete se encapsula en una trama de Ethernet. Si la interfaz de salida es una interfaz serial configurada para PPP, el paquete IP se encapsula en una trama PPP.
\end{itemize}
\subsubsection{Función de conmutación}
La función de conmutación es el proceso utilizado por un router para \textbf{aceptar} un paquete en una interfaz y \textbf{enviarlo} desde otra interfaz. Una responsabilidad clave de la función de conmutación es la de \textbf{encapsular} los paquetes en el tipo de trama de enlace de datos correcto para el enlace de datos de salida.
\section{Routing estático}
\begin{figure}[H]
\centering
\includegraphics[width=0.7\linewidth]{IN2/IN8.png}
\caption{Sintaxis comando \textit{IP route}.}
\end{figure}
El enrutamiento estático proporciona un método que otorga a los ingenieros de redes \textbf{control absoluto} sobre las rutas por las que se transmiten los datos en una internetwork. Para adquirir este control, en lugar de configurar protocolos de enrutamiento dinámico para que creen las tablas de enrutamiento, se crean manualmente. Es importante entender las ventajas y desventajas de la implementación de rutas estáticas, porque se utilizan extensamente en internetworks pequeñas y para establecer la conectividad con proveedores de servicios. Es posible que se crea que el enrutamiento estático es sólo un método antiguo de enrutamiento y que el enrutamiento dinámico es el único método usado en la actualidad. Esto no es así, además, se destaca que escribir una ruta estática en un router no es más que especificar una ruta y un destino en la tabla de enrutamiento, y que los protocolos de enrutamiento hacen lo mismo, sólo que de manera automática. Sólo hay dos maneras de completar una tabla de enrutamiento: manualmente (el administrador agrega rutas estáticas) y automáticamente (por medio de protocolos de enrutamiento dinámico). 
\begin{itemize}
\item Facilita el mantenimiento de la tabla de enrutamiento en redes más pequeñas en las cuales no está previsto que crezcan significativamente.
\item Proporciona routing hacia las redes de rutas internas y desde estas. Una red de rutas internas es aquella a la cual se accede a través de una única ruta y cuyo router tiene solo un vecino.
\item Utiliza una única ruta predeterminada para representar una ruta hacia cualquier red que no tenga una coincidencia más específica con otra ruta en la tabla de routing.
\end{itemize}
\subsection{Tipos de rutas estáticas}
Las rutas estáticas suelen usarse con más frecuencia para conectarse a una red específica o para proporcionar un gateway de último recurso para una red de rutas internas. También pueden utilizarse para lo siguiente:
\begin{itemize}
\item Para reducir el número de rutas anunciadas mediante el resumen de varias redes contiguas como una sola ruta estática
\item Para crear una ruta de respaldo en caso de que falle un enlace de la ruta principal
\end{itemize}
\subsubsection*{Ruta estática predeterminada}
IPv4 e IPv6 admiten la configuración de rutas estáticas. Las rutas estáticas son útiles para conectarse a una red remota específica.
\subsubsection*{Ruta estática predeterminada}
Una ruta por defecto o predeterminada, es una ruta que coincide con \textbf{todos los paquetes} y es utilizada por el router si un paquete no coincide con ninguna otra ruta más específica en la tabla de routing. Además, puede ser aprendida de forma dinámica o configurada de manera estática. Una ruta estática predeterminada es simplemente una ruta estática con \textbf{0.0.0.0/0} como dirección IPv4 de destino. Al configurar una ruta estática predeterminada, se crea un gateway de último recurso. Las rutas estáticas predeterminadas se utilizan en los siguientes casos:
\begin{itemize}
\item Cuando ninguna otra ruta de la tabla de routing coincide con la dirección IP destino del paquete. En otras palabras, cuando no existe una coincidencia más específica. Se utilizan comúnmente cuando se conecta un router periférico de una compañía a la red ISP.
\item Cuando un router tiene otro router único al que está conectado. En esta situación, se conoce al router como router de rutas internas.
\end{itemize}
\subsubsection*{Rutas estática resumida}
Para reducir el número de entradas en la tabla de routing, se pueden resumir varias rutas estáticas en una única ruta estática si se presentan las siguientes condiciones:
\begin{itemize}
\item Las redes de destino son contiguas y se pueden resumir en una única dirección de red.
\item Todas las rutas estáticas utilizan la misma interfaz de salida o la dirección IP del siguiente salto.
\end{itemize}
\subsubsection*{Ruta estática flotante}
Las rutas estáticas flotantes son rutas estáticas que se utilizan para proporcionar una ruta de \textbf{respaldo} a una ruta estática o dinámica principal, en el caso de una falla del enlace. La ruta estática flotante se utiliza únicamente cuando la ruta principal \textbf{no está disponible}. Para lograrlo, la ruta estática flotante se configura con una distancia administrativa mayor que la ruta principal. La distancia administrativa representa la \textbf{confiabilidad} de una ruta. Si existen varias rutas al destino, el router elegirá la que tenga una menor distancia administrativa. Por ejemplo, suponga que un administrador desea crear una ruta estática flotante como respaldo de una ruta descubierta por EIGRP. La ruta estática flotante se debe configurar con una \textbf{distancia administrativa mayor} que el EIGRP. El EIGRP tiene una distancia administrativa de 90. Si la ruta estática flotante se configura con una distancia administrativa de 95, se prefiere la ruta dinámica descubierta por el EIGRP a la ruta estática flotante. Si se pierde la ruta descubierta por el EIGRP, en su lugar se utiliza la ruta estática flotante.
\begin{figure}[H]
\centering
\includegraphics[width=0.8\linewidth]{IN2/IN9.png}
\caption{Ruta estática flotante.}
\end{figure}
\section{Routing dinámico}
Los protocolos de enrutamiento mantienen tablas de enrutamiento dinámicas por medio de mensajes de actualización del enrutamiento, que contienen información acerca de los cambios sufridos en la red, y que indican al software del router que actualice la tabla de enrutamiento en consecuencia. Intentar utilizar el enrutamiento dinámico sobre situaciones que no lo requieren es una pérdida de ancho de banda, esfuerzo, y en consecuencia de dinero. 
\subsection{Comparación entre routing dinámico y estático}
\textbf{Routing estático}
\begin{enumerate}
\item Facilita el mantenimiento de la tabla de routing en redes más pequeñas
\item Realiza routing desde y hacia una red de rutas internas
\item Permite acceder a una única ruta predeterminada.
\item Fácil de implementar en redes pequeñas
\item Adecuado solamente para topologías simples o para fines específicos, como una ruta estática predeterminada.
\item Muy seguro.
\item No se envían anuncios, a diferencia del caso de los protocolos de routing dinámico.
\item La complejidad de la configuración aumenta notablemente a medida que crece la red.
\item La ruta hacia el destino siempre es la misma.
\item Se requiere intervención manual para volver a enrutar el tráfico.
\item Dado que no se requieren algoritmos de routing ni mecanismos de actualización, no se necesitan recursos adicionales (CPU o RAM). 
\end{enumerate}
\textbf{Routing dinámico}
\begin{enumerate}
\item Ayudan al administrador de red a administrar el proceso riguroso y lento de configuración y mantenimiento de rutas estáticas.
\item Adecuado en todas las topologías donde se requieren varios routers.
\item La implementación puede ser más compleja.
\item Por lo general, es independiente del tamaño de la red.
\item Menos seguro.
\item Se requieren opciones de configuración adicionales para proporcionarle protección.
\item Si es posible, adapta automáticamente la topología para volver a enrutar el tráfico.
\item La ruta depende de la topología actual.
\item Requiere CPU, RAM y ancho de banda de enlace adicionales.
\end{enumerate}
En general, las operaciones de un protocolo de enrutamiento dinámico pueden
describirse de la siguiente manera:
\begin{enumerate}
\item El router envía y recibe mensajes de enrutamiento en sus interfaces.
\item El router comparte mensajes de enrutamiento e información de enrutamiento con otros routers que están usando el mismo protocolo de enrutamiento.
\item Los routers intercambian información de enrutamiento para obtener información sobre redes remotas.
\item Cuando un router detecta un cambio de topología, el protocolo de enrutamiento puede anunciar este cambio a otros routers.
\end{enumerate}
\subsection{Routing dinámico vector distancia}
“Vector distancia” significa que las rutas se anuncian proporcionando dos
características:
\begin{itemize}
\item \textbf{Distancia}: identifica la distancia hasta la red de destino. Se basa en una métrica como el conteo de saltos, el costo, el ancho de banda y el retraso, entre otros.
\item \textbf{Vector}: especifica el sentido en que se encuentra el router de siguiente salto o la interfaz de salida para llegar al destino.
\end{itemize}
Su métrica se basa en lo que se le llama en redes “Número de Saltos”, es decir la cantidad de routers por los que tiene que pasar el paquete para llegar a la red destino, la ruta que tenga el menor número de saltos es la más óptima y la que se publicará.
\begin{itemize}
\item Visualiza la red desde la perspectiva de los vecinos
\item Actualizaciones periódicas
\item Transmitirá copias completas o parciales de las tablas de enrutamiento
\item Convergencia lenta
\item Incrementa las métricas a través de las actualizaciones
\end{itemize}
\subsection{Estado de enlace}
A diferencia de la operación del protocolo de routing vector distancia, un router configurado con un protocolo de routing de estado de enlace puede crear una ``vista completa'' o una topología de la red al reunir información proveniente de todos los demás routers. Un router de estado de enlace usa la información de estado de enlace para crear un mapa de la topología y seleccionar la mejor ruta hacia todas las redes de destino en la topología. Los routers con RIP (Protocolo de información de encaminamiento) habilitado envían actualizaciones periódicas de su información de routing a sus vecinos (esto hace al RIP que se limite a pequeñas redes). Los protocolos de enrutamiento de link-state no usan actualizaciones periódicas. Una vez que se produjo la convergencia de la red, la actualización del estado de enlace solo se envía cuando se produce un cambio en la topología. Cuando se produce un fallo en la red el router que detecta el error utiliza una dirección multicast para enviar una tabla LSA (anuncion de estado de enlace), cada router recibe y la reenvía a sus vecinos. La métrica utilizada se basa en el coste, que surge a partir del algoritmo de Dijkstra y se basa en la velocidad del enlace.
\subsubsection*{Ventajas}
\begin{enumerate}
\item Los protocolos de estado de enlace solo envían actualizaciones cuando
hay cambios en la topología.
\item Las actualizaciones periódicas son menos frecuentes que en los protocolos por vector de distancia.
\item Las redes que ejecutan protocolos de enrutamiento por estado de enlace pueden ser segmentadas en distintas áreas jerárquicamente organizadas, limitando así el alcance de los cambios de rutas.
\item Las redes que ejecutan protocolos de enrutamiento por estado de enlace soporta direccionamiento sin clase.
\item Las redes con protocolos de enrutamiento por estado de enlace soportan resúmenes de ruta.
\end{enumerate}
\section{NAT para IPv4}
El término NAT es la abreviatura de \textit{Network Address Transalation} o traducción de direcciones de red. NAT es el proceso mediante el cual una o más direcciones locales (\textit{local address}) se traducen en una más direcciones globales (\textit{global IP address}) y viceversa, con el fin de proporcionar acceso a internet a los host locales.\\
Además de esto el NAT es el encargado de sustituir la red privada o \textit{private network} por la IP pública del router, con el objetivo de facilitar el acceso a internet de los nodos en direcciones privadas, haciendo que los direccionamientos sean compatibles. Este router mantiene una tabla de conexiones para saber luego a quién se debe enviar los datos.\\
Este puede ser el caso de tu red doméstica y la red de Internet, siendo necesario el modo NAT para que puedas llegar a conectar a la red global y así recibir o enviar información a ella. Si lo has pensado, la mayoría de usuarios tenemos las mismas direcciones IP dentro de nuestra casa, las típicas 192.168.1.xxx, o las que nosotros decidamos configurar en el router en un momento dado. Entonces te preguntas: si son las mismas IP ¿por qué el paquete de datos no le llega al vecino en lugar de a mí?\\
Podemos decir que en vez de tener que asignar una dirección IP diferente para cada de los dispositivos conectados, el NAT lo que hace es dar una única para todos. Puede ser cualquiera entre 192.168.0.0 y 192.168.255.255. Los paquetes de datos que proceden de Internet contienen la dirección IPv4 externa en su encabezado. Según el tipo de datos, el NAT lo reenvía a los dispositivos privados o internos para que los datos puedan procesarse según sea necesario.\\
Pues aquí es donde actúa el NAT, ya que oculta todo el espacio de direcciones IP privadas detrás de una sola dirección IP, o unas cuantas en función del modo NAT que se utilice. Esta sería la dirección IP pública del router, la que realmente conecta nuestra red interna a la red externa. Cuando el paquete de información llega a nuestro router, realmente lo ha hecho al tener información sobre esa IP pública. Será por tanto el router el que analice el paquete y compruebe si efectivamente hay un destinatario dentro de su red que está pidiendo esta información y la dejará entrar.
\subsection{Tipos de NAT}\index{Tipos de NAT}
\begin{enumerate}
\item \textbf{NAT estática}: Este tipo de NAT, normalmente es utilizada cuando la dirección local es convertida a su vez en una dirección pública, lo que esto quiere decir es que, habrá una dirección IP asociada a nuestro router o dispositivo NAT que será consistente.
\item \textbf{NAT dinámica}: En este caso la dirección IP privada se traduce a una IP pública de una lista que tenga el router. De esta forma cuando una determinada IP privada quiere acceder al exterior, el router comprueba en su lista propia cuál es la IP pública que está libre para asignársela. Esto añade mayor seguridad para los hosts que pertenezcan a esa red privada, ya que permite enmascarar la configuración interna de la red la ser IP asignadas aleatoriamente. Lo que sí debe asegurarse en este modo NAT es que todos los host, al menos puedan tener una IP pública asociada en el caso de que todos ellos se conecten a la vez.
\item \textbf{NAT por sobrecarga}: El último quizás sea el más importante o útil en lo que se refiere a nuestra red privada doméstica, ya que es el modo NAT que usa nuestro router. Sin duda el más utilizado por utilizar también puertos, además de IP para la traducción. Por ello recibe también el nombre de Port Address Translation o PAT.\\
En este caso, nuestro router solamente tiene asignada una dirección pública a la vez, y esta incluso puede ser dinámica al asignarse por DHCP cuando el router se arranca. En teoría, un router podría coger una IP pública distinta cada vez que se encendiera. Pero en la práctica, se asignará casi siempre la misma IP por estar ya asociada al router con anterioridad mediante la MAC. Es exactamente lo que ocurre cuando un router le asigna la misma IP a nuestro PC cada vez que se enciende.
\end{enumerate}
Vamos a explicarlo a la siguiente manera\footnote{Ejemplo sacado de \textit{Profesional review} \url{https://www.profesionalreview.com/2020/08/22/modo-nat-que-es/Que\_es\_el\_modo\_NAT}}
\begin{figure}[H]
\centering
\includegraphics[width=0.7\linewidth]{IN2/IN10.png}
\caption{Red privada a pública}
\end{figure}
Este modo NAT de sobrecarga permite comunicarnos con el exterior \textbf{sin desvelar} la IP privada que tiene nuestro equipo. Esto lo hace a través de una \textbf{IP pública} que tiene asignada el router para salir a Internet, es todo bastante intuitivo.\\
Imaginemos que subimos una foto a Instagram desde nuestro equipo, para ello nos tenemos que conectar a la IP del servidor a donde se guardará la foto. Nuestro PC tiene la \textbf{dirección privada 10.0.0.3}, y se pone en contacto con el router para enviar el paquete.\\
Este paquete además tiene asociado un número de puerto de origen, por ejemplo el 80 si estamos en nuestro navegador. A continuación, el router detecta que nuestro equipo desea comunicarse con una remota perteneciente a Internet, así que coge el paquete y le reasigna la dirección IP pública de nuestra conexión y un puerto escogido al azar de entre los 65.536 que hay disponibles y que no esté en uso.\\
Finalmente el paquete se envía a la red de Internet hasta llegar a la IP pública del otro extremo. Dependiendo del modo NAT que utilice, hará un proceso de traducción similar de IP y de puerto para almacenar la foto en uno de los servidores (con su IP privada) que haya detrás de la IP pública de Instagram.
\begin{figure}[H]
\centering
\includegraphics[width=0.7\linewidth]{IN2/IN11.png}
\caption{Red privada a pública}
\end{figure}
El proceso contrario será el mismo, cuando un paquete llega a nuestro router a través de una dirección IP y un puerto al azar, este analiza si el paquete va destinado a algún nodo de su red interna mediante la información de la cabecera. Si es así, cambiará de nuevo al puerto al que corresponda, por ejemplo el 80 si debe verse en el navegador, y la dirección IP de destino a la IP interna de nuestro PC. De esta forma, el proceso concluye.
\subsection{Ventajas y desventajas}
\textbf{Ventajas}:
\begin{itemize}
\item Conserva los esquemas y rangos de direccionamiento registrados legalmente, permitiendo así la privatización de intranets, como sabemos, el surgimiento del NAT fue para ahorrar en uso de direcciones IP públicas, ya que estas tenían un número limitado.
\item Conserva las direcciones a través de la multiplexación de aplicaciones a nivel de puertos, esto se utiliza en la mayoría de casos en los que se utiliza PAT (Traducción de la dirección del puerto, surge de la sobrecarga del NAT) por sus siglas en inglés, y en el cual podemos realizar el manejo de múltiples equipos y/o aplicaciones utilizando una sola dirección IP gracias a la utilización de los puertos TCP y UDP.
\item Aumenta la flexibilidad de las conexiones a las redes públicas, esto quiere decir que podemos mantener nuestro direccionamiento privado IPv4 y al mismo tiempo permite mantener los cambios a nuevas direcciones públicas.
\item Permite ocultar las direcciones IPv4 privadas de los usuarios y otros dispositivos, esto significa que, al momento de la traducción de direcciones, el usuario tendrá una dirección privada y a su vez una dirección pública diferente a esa dirección privada evitando que esta sea conocida.
\end{itemize}
\textbf{Desventajas}:
\begin{itemize}
\item Se deteriora el rendimiento de la red, en especial, en el caso de los protocolos en tiempo real como VoIP. NAT aumenta los retrasos de reenvió porque la traducción de cada dirección IPv4 dentro de los encabezados de los paquetes lleva tiempo.
\item Se pierde el direccionamiento de extremo a extremo. Muchos protocolos y aplicaciones de Internet dependen del direccionamiento de extremo a extremo desde el origen hasta el destino. Algunas aplicaciones no funcionan con NAT. Por ejemplo, algunas aplicaciones de seguridad, como las firmas digitales, fallan porque la dirección IPv4 de origen cambia antes de llegar a destino. Las aplicaciones que utilizan direcciones físicas, en lugar de un nombre de dominio calificado, no llegan a los destinos que se traducen a través del router NAT. En ocasiones, este problema se puede evitar al implementar las asignaciones de NAT estática.
\item Se reduce el seguimiento IPv4 de extremo a extremo. El seguimiento de los paquetes que pasan por varios cambios de dirección a través de varios saltos de NAT se torna mucho más difícil y, en consecuencia, dificulta la resolución de problemas.
\item Genera complicaciones en la utilización de protocolos de tunneling, como IPsec, porque NAT modifica valores en los encabezados, lo que hace fallar las comprobaciones de integridad.
\item El inicio de las conexiones TCP puede interrumpirse. A menos que el router NAT esté configurado para admitir dichos protocolos, los paquetes entrantes no pueden llegar a su destino. Algunos protocolos pueden admitir una instancia de NAT entre los hosts participantes (por ejemplo, FTP de modo pasivo), pero fallan cuando NAT separa a ambos sistemas de Internet.
\end{itemize}
\subsection{Configurar}\index{Configurar NAT}
\subsubsection{NAT estática}
La configuración estática de una NAT se divide en una relación de 1 a 1, una dirección IP privada a una dirección IP pública. Por lo que existe una traducción en la tabla de direcciones de la NAT tan pronto como se configura con los comandos NAT, siendo posible borrarlos únicamente con los mismos comandos.
Para configurar la NAT estática en el software Packet Tracer de cisco, accedemos al modo de configuración global del Router e introducimos el siguiente comando:
\begin{lstlisting}[numbers=none]
R1(config)# ip nat inside source static [direccion privada] [direccion global]
\end{lstlisting}
Luego configuramos las interfaces del router que tienen la direccion privada y global
\begin{lstlisting}[numbers=none]
R1(config)# interface gi 0/0/0
R1(config-int)# ip nat inside
R1(config-int)# interface gi 0/0/1
R1(config-int)# ip nat outside
\end{lstlisting}
\subsubsection{NAT dinámica}
Por otro lado tenemos la configuración de la NAT dinámica, la cual se encarga de que las direcciones IP privadas pertenecientes a una red tengan acceso a otras redes por medio de un conjunto de direcciones de IP pública, las cuales se asignan a cada host de forma aleatoria.\\
Para configurar la NAT dinámica en CPT, primeramente ingresamos al modo de configuración global del router. Luego creamos una conjunto de direcciones IP públicas por las cuales accederemos a otras redes
\begin{lstlisting}[numbers=none]
R1(config)# ip nat pool [nombre] [conjunto de direcciones publicas] netmask [mascara de la red]
\end{lstlisting}
Posteriormente configuramos una lista de acceso, la cual contendrá las direcciones de nuestra red privada
\begin{lstlisting}[numbers=none]
R1(config)# access-list [N#] permit [direccion de red] [mascara invertida]
\end{lstlisting}
Finalmente, configuramos las interfaces del router que contendrán la red privada y la red global.
\begin{lstlisting}[numbers=none]
ip nat inside source static {local-ip} {local-port} {global-ip} {global port}
\end{lstlisting}
Para traducir 192.168.1.1 a 192.168.2.200
\begin{lstlisting}[numbers=none]
R1(config)#ip nat inside source static 192.168.1.1 192.168.2.200
\end{lstlisting}
\chapterimage{chapter_head_IT2.pdf} % Chapter heading image
\chapter{Unidad II}
\section{Tecnología WAN}
Una red de área extensa (WAN) es una red privada de telecomunicaciones geográficamente distribuida que interconecta múltiples redes de área local (LANs). En una empresa, una WAN puede consistir en conexiones a la sede de una empresa, sucursales, instalaciones de colocación, servicios en la nube y otras instalaciones. Normalmente, se utiliza un enrutador u otro dispositivo multifunción para conectar una LAN a una WAN. Las WAN de empresa permiten a los usuarios compartir el acceso a aplicaciones, servicios y otros recursos ubicados centralmente. Esto elimina la necesidad de instalar el mismo servidor de aplicaciones, cortafuegos u otro recurso en varias ubicaciones.\\
Una WAN se describe mejor como una red de datos que cubre una distancia geográfica relativamente amplia. A diferencia de una LAN, que normalmente se localiza en un área relativamente pequeña, como una oficina, un edificio o un campus pequeño, una WAN suele abarcar distancias de unos pocos a miles de kilómetros.\\
Con el fin de interconectar lugares geográficamente dispersos, los \textbf{proveedores de servicios}, generalmente alquilan enlaces sobre una base mensual. La velocidad y el costo de estos enlaces pueden variar considerablemente, dependiendo de los requisitos de ancho de banda, las distancias a recorrer y las tecnologías disponibles.
\subsection{Operaciones}
Las redes WAN trabajan sobre el modelo TCP/IP, más específico, en la capa 1 y 2. Los protocolos de capa 1 describen la manera de  proporcionar \textbf{conexiones} eléctricas, mecánicas, operativas  y funcionales a los servicios de un proveedor de servicios  de comunicación. Los protocolos de capa 2 definen la forma en que se  \textbf{encapsulan} los datos y los mecanismos para transferir las  tramas resultantes.\\
Se muestra la terminología que normalmente se usa para describir las conexiones WAN, entre otros:
\begin{itemize}
\item \textbf{Equipo local del cliente (CPE)}: cables internos y dispositivos ubicados en el perímetro empresarial que se conectan a un enlace de una prestadora de servicios. El suscriptor es dueño del CPE o lo alquila al proveedor de servicios. En este contexto, un suscriptor es una empresa que obtiene los servicios WAN de un proveedor de servicios.
\item \textbf{Equipo de comunicación de datos (DCE)}: también llamado “equipo de terminación de circuito de datos”, el DCE consta de dispositivos que colocan los datos en el bucle local. Principalmente, el DCE proporciona una interfaz para conectar a los suscriptores a un enlace de comunicación en la nube WAN.
\item \textbf{Equipo terminal de datos (DTE)}: dispositivos del cliente que transmiten los datos desde un equipo host o la red de un cliente para la transmisión a través de la WAN. El DTE se conecta al bucle local a través del DCE.
\item \textbf{Punto de demarcación}: un punto establecido en un edificio o un complejo para separar el equipo del cliente del equipo del proveedor de servicios. En términos físicos, el punto de demarcación es la caja de conexiones del cableado, ubicada en las instalaciones del cliente, que conecta los cables del CPE al bucle local. Por lo general, se coloca de modo que sea de fácil acceso para un técnico. El punto de demarcación es el lugar donde la responsabilidad de la conexión pasa del usuario al proveedor de servicios. Cuando surgen problemas, es necesario determinar si el usuario o el proveedor de servicios es responsable de la resolución o la reparación.
\item \textbf{Bucle local}: cable de cobre o fibra propiamente dicho que conecta el CPE a la CO del proveedor de servicios. A veces, el bucle local también se denomina “última milla”.
\item \textbf{Oficina central (CO)}: la CO es la instalación o el edificio del proveedor de servicios local que conecta el CPE a la red del proveedor.
\item \textbf{Red interurbana}: consta de líneas de comunicación, switches, routers y otros equipos digitales, de largo alcance y de fibra óptica dentro de la red del proveedor de servicios WAN.
\end{itemize}
\subsection{Opciones de acceso}
Existen diversas opciones de conexión de acceso  WAN que los ISP (Proveedor de servicios de internet) pueden utilizar para conectar el  bucle local al perímetro empresarial. Cada opción tiene distintas de ventajas y  desventajas, así como diferencias con respecto a la  tecnología, la velocidad y el costo.
\subsubsection{Pública}
El proveedor de servicios puede ofrecer acceso a Internet de banda ancha mediante una línea de suscriptor digital (DSL), cable y acceso satelital. Las opciones de conexión de banda ancha normalmente se usan para conectar oficinas pequeñas y trabajadores a distancia a un sitio corporativo a través de Internet. Los datos que se transmiten entre sitios corporativos a través de la infraestructura WAN pública se deben proteger mediante VPN.\\
Las conexiones de infraestructura pública incluyen DSL, cable, tecnología inalámbrica y datos móviles 3G/4G. En las conexiones mediante infraestructura pública, se puede proporcionar seguridad con redes virtuales privadas (VPN) de acceso remoto o de sitio a sitio. En esta sección se comparan las diferentes tecnologías WAN públicas: DSL, cable, tecnología inalámbrica y datos móviles 3G/4G. Además de la seguridad con redes virtuales privadas (VPN).
\subsubsection{Privada}
Los proveedores de servicios pueden ofrecer líneas arrendadas punto a punto dedicadas, enlaces de conmutación de circuitos, como PSTN o ISDN, y enlaces de conmutación de paquetes, como WAN Ethernet, ATM o Frame Relay.\\
El uso de líneas alquiladas o arrendadas proporciona conexiones punto a punto dedicadas permanentes. El acceso por dial-up, si bien es lento, aún es viable para las áreas remotas con opciones de WAN limitadas. Otras opciones de conexión privada incluyen ISDN, Frame Relay, ATM, WAN Ethernet, MPLS y VSAT. En esta sección se comparan las diferentes tecnologías WAN privadas: Líneas arrendadas, dial-up, ISDN, Frame Relay, ATM, WAN Ethernet, MPLS y VSAT.
\begin{definition}[Ethernet WAN]
Proporciona ancho de banda dedicado y de alta velocidad entre dos sitios, tráfico convergente de alta velocidad, incluyendo voz, video y datos en la misma infraestructura de red.
\end{definition}
\begin{definition}[Líneas arrendada]
Una línea arrendada, también conocida como línea dedicada, conecta dos ubicaciones para servicios privados de telecomunicaciones de voz y/o datos. Una línea arrendada no es un cable dedicado; es un circuito reservado entre dos puntos. La línea arrendada está siempre activa y disponible por una cuota mensual fija. Este término también hace referencia a:
\begin{itemize}
\item Circuitos arrendados
\item Enlace serial
\item Línea serial
\item Enlace \textbf{punto a punto}
\item Líneas T1/E1 o T3/E3
\end{itemize}
\end{definition}
\begin{definition}[Dial Up]
La tecnología Dial-Up le permite acceder al servicio Internet a través de una línea telefónica analógica y un MODEM. El módem modula los datos binarios en una  señal analógica en el origen y demodula la señal  analógica en datos binarios en el destino. Las características físicas del bucle local y su  conexión a la PSTN limitan la velocidad de señal  a menos de 56 kb/s.
\end{definition}
\begin{definition}[PSTN]
PSTN son las siglas correspondientes a Red Telefónica Pública Conmutada. Se trata de una red tradicional de teléfono que logra que se puedan realizar las llamadas locales a larga distancia en tiempo real y de forma fluida. El objetivo de esta red es lograr una transmisión efectiva de voz de emisor a receptor a través de un auricular.
\end{definition}
\begin{definition}[ISDN]
Red digital de servicios integrados o \textbf{RDSI} es una red que procede de la evolución de la \textbf{Red telefónica básica} (RTB) o  \textbf{Red telefónica conmutada} (RTC) que facilita conexiones digitales extremo a extremo entre terminales conectadas a ella.\\
ISDN cambia las conexiones internas de la  PSTN (Red Telefónica Pública Conmutada) para que transporte señales digitales  multiplexadas por división de tiempo (TDM) en  vez de señales analógicas. TDM permite que se transfieran dos o más  señales, o flujos de bits, como subcanales en  un canal de comunicación. La conexión ISDN puede requerir un adaptador  de terminal (TA), que es un dispositivo utilizado  para conectar las conexiones de la interfaz de  velocidad básica (BRI) de ISDN a un router.
\end{definition}
\begin{definition}[Frame relay]
Frame Relay es una tecnología de protocolo de red de conmutación de paquetes digital de capa de enlace de datos diseñada para conectar redes de área local (LAN) y transferir datos a través de redes de área amplia (WAN). Frame Relay es una tecnología de \textbf{paquetes o celdas} rápida, lo que significa que el protocolo no intenta corregir errores. Requiere una conexión dedicada durante el período de transmisión y \textbf{no es ideal} para voz o video, que requieren un flujo constante de transmisiones.
\end{definition}
\begin{definition}[ATM]
La tecnología del modo de transferencia  asíncrona (ATM) puede transferir voz, video y  datos a través de redes privadas y públicas.\\
Una línea ATM típica necesita casi un 20\% más  de ancho de banda que Frame Relay para  transportar el mismo volumen de tráfico de red.
Cuando la celda transporta los paquetes de  capa de red segmentados, la sobrecarga es  mayor debido a que el ATM debe poder rearmar  los paquetes en el destino.
\end{definition}
\begin{definition}[MPLS]
El switching por etiquetas multiprotocolo  (Multiprotocol Label Switching, MPLS) es una  tecnología WAN de alto rendimiento multiprotocolo  que dirige los datos desde un router al siguiente. 
MPLS se basa en etiquetas de ruta cortas en lugar de direcciones de red IP. Se llama multiprotocolo porque tiene la capacidad de  transportar cualquier contenido, incluido tráfico IPv4,  IPv6, Ethernet, ATM, DSL y Frame Relay. Usa etiquetas que le indican al router qué hacer con un paquete.
\end{definition}
\begin{definition}[VSAT]
Una VSAT es una pequeña antena satelital que se  utiliza para crear una WAN privada que proporciona  conectividad a ubicaciones remotas. El satélite se encuentra en órbita geosincrónica en el espacio. Las señales deben recorrer alrededor de  35 786 km hasta el satélite y regresar.
\end{definition}
\begin{definition}[DSL]
DSL (línea de suscriptor digital) es una familia de tecnologías que proporcionan el acceso a Internet mediante la transmisión de datos digitales a través del par trenzado de hilos de cobre convencionales de la red telefónica básica o conmutada, constituida por las líneas de abonado: ADSL, ADSL2, ADSL2+, SDSL, IDSL, HDSL, SHDSL, VDSL y VDSL2. Esta tecnología de banda ancha que proporciona transmisión de información a alta velocidad (hasta 7.1 Mbps) a través de una línea telefónica existente. Las velocidades son hasta 50 veces más rápidas que las de un módem por dial-up estándar de 28.8 Kbps. Además, el servicio DSL es instantáneo - con sólo un clic tendrás una conexión a Internet superrápida. Hace uso de \textbf{DSLAM} (Digital Subscriber Line Access Multiplexer o Multiplexor de acceso de línea de abonado digital). Es el equipamiento que permite dotar a tu línea telefónica análoga, de servicios basados en ADSL (ya sea ADSL, ADSL+, VDSL). Básicamente, es una máquina con muchos filtros (uno por línea), que separa la frecuencia de voz (300 a 3.400 Hz) de la utilizada para datos (25 kHz a 1,104 MHz). CMTS son las siglas de Cable Modem Termination System. Es un equipo que se encuentra normalmente en la cabecera de la compañía de cable y se utiliza para proporcionar servicios de datos de alta velocidad, como Internet por cable o Voz sobre IP, a los abonados.
\end{definition}
\begin{definition}[VPN]
VPN significa "Virtual Private Network" (Red privada virtual) y describe la oportunidad de establecer una conexión protegida al utilizar redes públicas. Las VPN cifran su tráfico en internet y disfrazan su identidad en línea. Esto le dificulta a terceros el seguimiento de sus actividades en línea y el robo de datos. El cifrado se hace en tiempo real.
\end{definition}
\begin{definition}[SONET]
La red óptica síncrona (SONET) es un protocolo de comunicación digital estandarizado que se utiliza para transmitir un gran volumen de datos a distancias relativamente largas utilizando un medio de fibra óptica. Con SONET, se transfieren múltiples flujos de datos digitales al mismo tiempo a través de fibra óptica utilizando LED y rayos láser. Los sistemas \textbf{SDH} son ampliamente usados, excepto en EE. UU. y Canadá, dónde se usa SONET. El SDH es un dispositivo digital que trabaja realizando multiplexación por división el tiempo. Toma pequeñas ranuras de tiempo y las ubica en forma ordenada en una ranura de tiempo más grande. La sucesión de ranuras en de tiempo se denomina “Trama”. Usa la tecnología \textbf{DWDM} es una técnica de transmisión, que realiza la multiplexación por división en longitudes de onda, es decir, que utiliza varias longitudes de onda de luz, descompuesta en colores, para enviar datos desde dos o más colores de luz que pueden viajar sobre una sola fibra óptica.
\end{definition}
\begin{definition}[CSU/DSU]
Una \textbf{unidad de servicio de canal} (CSU) es un dispositivo que conecta un terminal a una línea digital. Una \textbf{unidad de servicio de datos} (DSU) es un dispositivo que lleva a cabo funciones de protección y de diagnóstico en una línea de telecomunicaciones. En general, los dos dispositivos se entregan formando una sola unidad, CSU/DSU.
\end{definition}
\begin{definition}[WiMAX]
WiMAX es una tecnología de comunicaciones de red inalámbrica de próxima generación. La tecnología es similar a la Wi-Fi, pero proporciona acceso de banda ancha de alta velocidad en un área más grande con menos interferencias. WiMAX permite el acceso móvil a Internet. Para que lo entiendas de una manera sencilla, podemos referirnos a WiMAX como una alternativa al cable a la hora de llevar Internet a tu casa mediante conexión inalámbrica basada en los estándares de comunicación IEEE 802.16, y que permite llevarte Internet con un alcance que puede llegar a los 70 kilómetros.
\end{definition}
\begin{definition}[BGP]
El protocolo de puerta de enlace de frontera o BGP es un protocolo mediante el cual se intercambia información de encaminamiento entre sistemas autónomos, por ejemplos distintos servicios de Internet.
\end{definition}
\section{EIGRP}
El protocolo de routing de gateway interior mejorado (EIGRP – Enhanced Interior Gateway Routing Protocol) es un protocolo de routing vector distancia avanzado desarrollado por Cisco Systems. Como lo sugiere el nombre, EIGRP es una mejora de otro protocolo de routing de Cisco: el protocolo de routing de gateway interior (IGRP).\\
EIGRP incluye características propias de los protocolos de routing de estado de enlace. EIGRP es apto para numerosas topologías y medios diferentes. En una red bien diseñada, EIGRP puede escalar para incluir varias topologías y puede proporcionar tiempos de convergencia extremadamente rápidos con un mínimo tráfico de red.
\subsection{Características}
\begin{itemize}
\item \textbf{Algoritmo de actualización difusa}: El algoritmo de actualización por difusión (DUAL), que es el motor de cómputo detrás del EIGRP, constituye el centro del protocolo de routing. DUAL garantiza rutas de respaldo y sin bucles en todo el dominio de routing. Al usar DUAL, EIGRP almacena todas las rutas de respaldo disponibles a los destinos, de manera que se puede adaptar rápidamente a rutas alternativas si es necesario.
\item \textbf{Establecimiento de adyacencias de vecinos}: EIGRP establece relaciones con routers conectados directamente que también están habilitados para EIGRP. Las adyacencias de vecinos se usan para rastrear el estado de esos vecinos.
\item \textbf{Protocolo de transporte confiable}: El protocolo de transporte confiable (RTP) es exclusivo de EIGRP y se encarga de la entrega de los paquetes EIGRP a los vecinos. RTP y el rastreo de las adyacencias de vecinos establecen el marco para DUAL.
\item \textbf{Actualizaciones parciales y limitadas}: En lo que respecta a sus actualizaciones, en EIGRP se utilizan los términos “parcial” y “limitada”. A diferencia de RIP, EIGRP no envía actualizaciones periódicas, y las entradas de ruta no vencen. El término “parcial” significa que la actualización solo incluye información acerca de cambios de ruta, como un nuevo enlace o un enlace que deja de estar disponible. El término “limitada” se refiere a la propagación de las actualizaciones parciales que se envían solo a aquellos routers que se ven afectados por el cambio. Esto minimiza el ancho de banda que se requiere para enviar actualizaciones de EIGRP.
\item \textbf{Balanceo de carga de mismo costo o con distinto costo}: EIGRP admite balanceo de carga de mismo costo y balanceo de carga con distinto costo, lo que permite a los administradores distribuir mejor el flujo de tráfico en sus redes.
\end{itemize}
\subsection{Detección de la ruta inicial del EIGRP}
\begin{figure}[H]
\centering
\includegraphics[width=0.8\linewidth]{in2/in12.png}
\caption{Adyacencia de vecinos}
\end{figure}
\begin{enumerate}
\item El router R1 comienza incorporando el dominio  de routing del EIGRP y envía un paquete de  saludo del EIGRP a todas las interfaces  habilitadas para el EIGRP.
\item El router R2 recibe el paquete de saludo y
agrega al R1 a su tabla de vecinos.
\begin{itemize}
\item El R2 envía un paquete de actualización que  contiene todas las rutas que conoce.
\item El R2 envía un paquete de saludo del EIGRP al R1.
\end{itemize}
\item El R1 actualiza su tabla de vecinos con el R2.
\end{enumerate}
Una vez que ambos routers intercambian saludos, se establece la adyacencia de vecino
\begin{figure}[H]
\centering
\includegraphics[width=0.8\linewidth]{in2/in13.png}
\caption{Tabla de topología de EIGRP}
\end{figure}
\begin{enumerate}
\item El R1 recibe la actualización de EIGRP del vecino R2 e incluye información acerca de las rutas que anuncia el vecino, incluida la métrica a cada destino. El R1 agrega todas las entradas de actualización a su tabla de topología. La tabla de topología incluye todos los destinos anunciados por los routers vecinos (adyacentes) y el costo (métrica) para llegar a cada red.
\item Los paquetes de actualización EIGRP utilizan entrega confiable; por lo tanto, el R1 responde con un paquete de acuse de recibo EIGRP que informa al R2 que recibió la actualización.
\item El R1 envía una actualización de EIGRP al R2 en la que anuncia las redes que conoce, excepto aquellas descubiertas del R2 (horizonte dividido).
\item El R2 recibe la actualización de EIGRP del vecino R1 y agrega esta información a su propia tabla de topología.
\item El R2 responde al paquete de actualización EIGRP del R1 con un acuse de recibo EIGRP.
\end{enumerate}
\begin{figure}[H]
\centering
\includegraphics[width=0.8\linewidth]{in2/in14.png}
\caption{Convergencia del EIGRP}
\end{figure}
\begin{enumerate}
\item Después de recibir los paquetes de actualización EIGRP del R2, el R1 utiliza la información en la tabla de topología para actualizar su tabla de routing IP con la mejor ruta a cada destino, incluidos la métrica y el router del siguiente salto.
\item De la misma manera que el R1, el R2 actualiza su tabla de routing IP con las mejores rutas a cada red.
\end{enumerate}
Llegado a este punto, se considera que EIGRP está en estado convergente en ambos routers.
\subsection{Métricas}
De manera predeterminada, EIGRP utiliza los siguientes valores en su métrica compuesta para calcular la ruta preferida a una red:
\begin{itemize}
\item \textbf{Ancho de banda}: el ancho de banda más lento entre todas las interfaces de salida, a lo largo de la ruta de origen a destino.
\item \textbf{Retraso}: la acumulación (suma) de todos los retrasos de las interfaces a lo largo de la ruta (en decenas de microsegundos).
\end{itemize}
Se pueden utilizar los valores siguientes, pero no se recomienda, porque generalmente dan como resultado recálculos frecuentes de la tabla de topología:
\begin{itemize}
\item \textbf{Confiabilidad}: representa la peor confiabilidad entre origen y destino, que se basa en keepalives.
\item \textbf{Carga}: representa la peor carga en un enlace entre origen y destino, que se calcula sobre la base de la velocidad de paquetes y el ancho de banda configurado de la interfaz.
\end{itemize}
%----------------------------------------------------------------------------------------
%	NEW CHAPTER
%----------------------------------------------------------------------------------------
\part{Control Adaptativo Moderno}
\chapterimage{chapter_head_CAM.pdf} % Chapter heading image

\chapter{Unidad I}
\section{CADe Simu}
CADe SIMU es un programa de CAD electrotécnico que permite insertar los distintos símbolos organizados en librerías y trazar un esquema eléctrico de una forma fácil y rápida para posteriormente realizar la simulación.\\
El programa en modo simulación visualiza el estado de cada componente eléctrico cuando esta activado al igual que resalta los conductores eléctricos sometidos al paso de una corriente eléctrica.
\subsection{Paneles}
\begin{center}
Barra de herramientas
\includegraphics[width=\linewidth]{cam/cam1.png}
Librerías
\includegraphics[width=\linewidth]{cam/cam2.png}
Alimentaciones
\includegraphics[width=\linewidth]{cam/cam3.png}
Fusibles, seccionadores
\includegraphics[width=\linewidth]{cam/cam4.png}
Automáticos, disyuntores
\includegraphics[width=\linewidth]{cam/cam5.png}
Contactores, interruptores
\includegraphics[width=\linewidth]{cam/cam6.png}
Motores
\includegraphics[width=\linewidth]{cam/cam7.png}
Potencia
\includegraphics[width=\linewidth]{cam/cam8.png}
Contactos
\includegraphics[width=\linewidth]{cam/cam9.png}
Accionamiento
\includegraphics[width=\linewidth]{cam/cam10.png}
Detectores
\includegraphics[width=\linewidth]{cam/cam11.png}
Bobinas, señalizaciones
\includegraphics[width=\linewidth]{cam/cam12.png}
Relés electrónicos
\includegraphics[width=\linewidth]{cam/cam13.png}
Lógica
\includegraphics[width=\linewidth]{cam/cam14.png}
Ladder
\includegraphics[width=\linewidth]{cam/cam15.png}
Grafcet
\includegraphics[width=\linewidth]{cam/cam16.png}
Entrada/Salida
\includegraphics[width=\linewidth]{cam/cam17.png}
Electroneumática
\includegraphics[width=\linewidth]{cam/cam18.png}
Símbolos imagen 2D
\includegraphics[width=\linewidth]{cam/cam19.png}
Símbolos imagen 3D
\includegraphics[width=\linewidth]{cam/cam20.png}
Cables y conexiones
\includegraphics[width=\linewidth]{cam/cam21.png}
\end{center}
\begin{notation}
La clave del CADe Simu es 4962 y del PC Simu es 9966.
\end{notation}
\subsection{Simulaciones}
Para poder realizar circuitos en CADe Simu, necesitamos tener en cuenta los siguientes pasos:
\begin{enumerate}
\item Realizar el esquema insertando los componentes eléctricos.
\item Los distintos componentes siempre se tendrán que conectar a través del distinto cableado, NO se pueden conectar los distintos componentes de forma directa sin el cableado.
\item En un circuito eléctrico se tiene que partir siempre de una alimentación pudiendo ser de corriente continua o corriente alterna. La librería de alimentaciones permite seleccionar una gran variedad de símbolos de alimentación.
\item Durante la simulación se realiza una comprobación de la existencia tanto de cortocircuitos y de conexiones a masa. Si se produce uno de estos errores la simulación se para y se nos advierte con el correspondiente mensaje.
\item En un esquema los distintos símbolos de un mismo componente tienen que tener el mismo nombre y no se puede repetir con símbolos de otros componentes.
\end{enumerate}

\begin{example}[Arranque de motor trifásico con sistema de parada de emergencia]
Nos piden diseñar un arranque de motor trífasico con botones de Start y Stop, además que tenga un sistema de protección si el motor se sobre carga.\\
\textbf{Solución}\\
\textbf{Circuito de mando:}\\
Empezamos con dos relés térmicos: NA (normalmente abierto) y NC (normalmente cerrado). Estos van a actuar al mismo tiempo, por eso les ponemos el mismo nombre: \textit{-F2}. Con esta configuración podemos desviar la corriente de la rama principal, en la cual irá el circuito de arranque, hacia la rama de emergencia.
\begin{center}
\includegraphics[width=0.5\linewidth]{cam/cam22.png}
\end{center}
En la \textbf{rama principal} pondremos un pulsador NC con el nombre \textit{-S1}. Colocamos un NC pues inicialmente la corriente debe fluir hacia debajo de este. Si accionamos el pulsador \textit{-S1}, este se abrirá y hará que el circuito por debajo de este deje de energizarse: Esto nos servirá como interruptor de apagado. Seguidamente pondremos en paralelo un pulsador NA con el nombre \textit{-S2} y un contacto NA con el nombre \textit{-KM1}. Este contacto se activará al pulsar \textit{-S2}. Lo sé, no tienen el mismo nombre pero haremos que funcione con una ``memoria''. La ``memoria'' va ser reemplazada por una bobina que se encuentra en serie del pulsador \textit{-S2}. Añadiremos un piloto señalizador verde que nos indique que el motor esta funcionando.
\begin{center}
\includegraphics[width=0.3\linewidth]{cam/cam23.png}
\end{center}
Resumiendo hasta ahora: el térmico \textit{-F2} y el pulsador \textit{-S1} harán pasar la corriente hasta \textit{-S2}. Si este último se presiona, cargará la bobina y este a su vez mantendrá activado el contacto \textit{-KM1}, además se encenderá el piloto \textit{-H1}.\\
En la \textbf{rama de emergencia}, el térmico NA \textit{-F2} se cerrará cuando el térmico NC \textit{-F2} se abra, en otras palabras cuando exista una emergencia, la corriente se desviará completamente de la rama principal a la rama de emergencia. En esta rama pondremos solo un piloto rojo a manera de indicar que ha sucedido un problema: piloto rojo \textit{-H2}.\\
\textbf{Circuito de potencia}:
Poniendo de antemano la alimentación en fase correspondiente pondremos un contactor III con el nombre \textit{-KM1}. Cuando se active la bobina, este mantendrá el piloto \textit{-H1} activado y también el motor. Seguidamente pondremos un relé térmico \textit{-F2}. Esté será el encargado de activar los térmicos \textit{-F2} de emergencia. Finalmente colocamos el motor.
\begin{center}
\includegraphics[width=0.6 \linewidth]{cam/cam24.png}
\end{center}
\end{example}
\begin{remark}
Recuerda, para alimentación Fase necesitas \textcolor{red}{cable fase} y alimentación Neutro necesitas \textcolor{blue}{cable neutro}.
\end{remark}
\begin{example}[Realizar dos fajas transportadoras con modo Manual y Automático]
Se requiere dos fajas transportadoras que funcionen de dos modos:
\begin{itemize}
\item \textbf{Manual}: Botón de encendido que empiece la marcha de las fajas y motor de detención para detener todo.
\item \textbf{Automático}: Si se detecta un objeto al inicio de la faja, que empiece a funcionar.
\end{itemize}
\textbf{Solución}:\\
\textbf{Circuito de Mando}:\\
Tomaremos parte del circuito del anterior ejemplo. Empecemos con el modo \textbf{manual}, habrá un botón para poder entrar a este modo al igual que con el modo automático. Al ser un ``menú'', tenemos que usar pulsadores NA (solo se activarán cuando nosotros lo presionemos). Cuando se presione este pulsador \textit{-S1}, vamos a guardar su estado usando la bobina \textit{-K5}, a su vez esta bobina \textit{-K5} mantendrá activo el contacto NA \textit{-K5}. Estamos usando también un contacto NC \textit{-K6} para apagar este modo manual. Más tarde veremos como lo usaremos:\\
\begin{center}
\includegraphics[width=0.4\linewidth]{cam/cam25.png}
\end{center}
Cuando la energía llegué a \textit{-K5} indica que el modo manual ha sido activado, es por esta razón que procedemos a activar los motores. Al igual que el ejemplo anterior, tendremos una rama (principal) de control y la otra de emergencia (para este paso no vamos a explicar pues es el mismo funcionamiento), la única diferencia es que tanto el rama de control como la de emergencia esta siendo activo solo si el contacto NA \textit{-K5} se cierra, y eso sucede cuando el seleccionamos modo manual. En este ejercicio el pulsador \textit{-S3} representa \textbf{stop} y el pulsador \textit{-S4} representa \textbf{start}:
\begin{center}
\includegraphics[width=0.4\linewidth]{cam/cam26.png}
\end{center}
Finalmente, la bobina \textit{-K7} encenderá los contactos NA \textit{-K7}, que a su vez energizarán las bobinas de los motores.
\begin{center}
\includegraphics[width=0.4\linewidth]{cam/cam28.png}
\end{center}
\textbf{Modo Automático}\\
Para la selección de este modo es muy similar al anterior, nota que aquí estamos haciendo uso del contacto NC \textit{-K6}, este contacto nos ayuda a apagar el otro modo, es decir, si se selecciona el modo manual, el modo automático se apaga y viceversa.
\begin{center}
\includegraphics[width=0.2\linewidth]{cam/cam27.png}
\end{center}
Cuando se seleccione el modo auto, el contacto \textit{-K6} se encenderá ayudado por la bobina \textit{-K6}. Este contacto \textit{-K6} activará los detectores inductivos NA \textit{-B1} y \textit{-B2}. 
\begin{center}
\includegraphics[width=0.35\linewidth]{cam/cam29.png}
\end{center}
Esto quiere decir que los detectores no dejarán pasar energía a no ser que detecten algo, es ahí cuando se cierran y se activarán los temporizadores, el temporizador a su vez cierra el contacto desconexión NA \textit{-K1} energizando la bobina del motor \textit{-K3}. 
\begin{center}
\includegraphics[width=0.3\linewidth]{cam/cam28.png}
\end{center}
\textit{-K1} estará encendido durante 3 seg (se configura en el temporizador de desconexión), pasado ese tiempo \textit{-K1} se desconecta apagando el motor. El proceso es exactamente lo mismo para el detector inductor \textit{-B2}.
La parte de los motores es la misma que el ejemplo anterior.\\
\begin{center}
Circuito en CADe Simu\\
\includegraphics[width=0.85\linewidth]{cam/cam31.png}
\end{center}
\begin{center}
Simulación en PC Simu\\
\includegraphics[width=0.8\linewidth]{cam/cam32.png}
\end{center}
\end{example}
\subsection{Ladder}\index{Ladder}
Ladder es escalera en inglés. El nombre por lo tanto recuerda que este lenguaje de programación se programa mediante símbolos gráficos y en diferentes segmentos. Como en las escaleras, en cada segmento (o escalón), programamos las diferentes sentencias de la lógica. Al lenguaje Ladder también se le conoce como diagrama de contactos, puesto que realmente programamos mediante contactos eléctricos que, unidos, terminan formando una sentencia lógica.\\
Ladder es uno de los diferentes lenguajes de programación para los controladores lógicos programables (PLCs) estandarizados con IEC 61131-3. En Ladder, la energía se desplaza de izquierda a derecha en lugar de arriba hacia abajo como en los esquemas eléctricos. En un circuito típico aparecen los contactos en la parte izquierda y una bobina en la parte derecha. La lógica de control que representa dicho circuito puede verse como una inferencia lógica que tiene como antecedente la lógica de los contactos y como concluyente la bobina.\\
Para programar un autómata con Ladder, además de estar familiarizado con las reglas de los circuitos de conmutación, (también denominada Lógica de Contactos), es necesario conocer cada uno de los elementos de que consta este lenguaje. A continuación se describen de modo general los más comunes (Fig. \ref{fig:basicos ladder})
\begin{itemize}
\item \textbf{Contacto normalmente abierto (E1)}: si la variable asociada E1 vale ‘0’, el contacto permanece abierto, y si vale ‘1’ se cierra.
\item \textbf{Contacto normalmente cerrado (E2)}: si la variable asociada E1 vale ‘1’, el contacto permanece abierto, y si vale ‘0’ se cierra.
\item \textbf{Salida, bobina o relé (S1)}: la variable asociada S1 tomará el valor de la variable (o combinación de variables) que esté a su entrada (punto de conexión del lado izquierdo). También se puede enclavar o desenclavar, indicándolo con una S o R como se indica en los casos de S2 y S3.
\end{itemize}
\begin{figure}[]
\centering
\includegraphics[width=0.5\linewidth]{CAM/cam33.png}
\caption{Elementos básicos de Ladder.}
\label{fig:basicos ladder}
\end{figure}
\begin{notation}
Una bobina normal puede verse como una asignación del valor lógico conectado a su izquierda. Por contra, una bobina de enclavamiento (S / R) se activa de la misma manera que la bobina anterior, pero retiene el valor (‘1’ / ‘0’) aunque el valor lógico conectado a su izquierda pase a ‘0’.
\end{notation}
\subsubsection{Compuertas lógicas}
Se pueden implementar funciones lógicas de forma sencilla. Por ejemplo, en la figura siguiente (Fig. \ref{fig:compuertas ladder}) se implementa un función AND y una OR.
\begin{figure}[H]
\centering
\includegraphics[width=0.5\linewidth]{CAM/cam34.png}
\caption{Elementos básicos de Ladder.}
\label{fig:compuertas ladder}
\end{figure}
A manera de explicar el lenguaje Ladder, vamos a arrancar un motor trifásico usando para ello un PLC LOGO\footnote{Disponible en la pestaña \textit{Entrada/Salida}>Módulo Lógico}; vamos a tener 3 botones de control: \textit{Start}, \textit{Stop} y uno auxiliar (puede ser de emergencia). En el PLC hacemos la conexión \textbf{Alimentación Fase} y \textbf{Alimentación Neutro}. Del positivo debemos tener las siguientes conexiones:
\begin{itemize}
\item P1: Conexión directa.
\item I1: Conexión por un pulsador NA.
\item I2: Conexión por un pulsador NA.
\item I3: Conexión por un pulsador NA.
\item Q1-1: Conexión directa.
\item Q2-1: Conexión directa.
\end{itemize}
El diagrama nos debe quedar como la figura \ref{fig:logo fase}\\
\begin{figure}[H]
\centering
\subfloat[Fase: parte superior]{\includegraphics[width=0.5\linewidth]{CAM/cam35.png}}
\subfloat[Fase: parte inferior]{\includegraphics[width=0.5\linewidth]{CAM/cam36.png}}
\caption{Conexión fase.}
\label{fig:logo fase}
\end{figure}
Usando la alimentación neutra conectaremos la siguiente manera:
\begin{itemize}
\item P2: Conexión directa.
\item Q1-2: A la salida de una bobina, esta bobina debe tener el mismo nombres que los contactores III del motor.
\end{itemize}
El resultado del PLC debe ser como la figura \ref{fig:conexion plc}.
\begin{figure}[H]
\centering
\includegraphics[width=0.4\linewidth]{CAM/cam37.png}
\caption{Conexión PLC}
\label{fig:conexion plc}
\end{figure}
Recordando la Fig. \ref{fig:compuertas ladder}, más en específico usaremos la compuerta OR: es necesario que una de las entradas se active para que las energice lo demás. Empezamos colocando \textbf{Alimentación +} y \textbf{Alimentación -}. 
Configuramos la compuerta OR: la primera entrada/contacto que se cierre con el pulsador \textit{START} ó que se cierre con \textit{Q1} (Q1 va ser configurada luego como una retro-activación). Después de la compuerta OR vamos a tener contactos NC para que dejen pasar la energía inicialmente; estos serán \textit{PE} y \textit{STOP}. Al final de esta rama colocaremos una bobina \textit{Q1}, la que servirá como retroalimentación. (Fig. \ref{fig:ladder motor})
\begin{figure}[]
\centering
\includegraphics[width=\linewidth]{CAM/cam38.png}
\caption{Diagrama Ladder.}
\label{fig:ladder motor}
\end{figure}
\textbf{Funcionamiento}: Para que el motor pueda arrancar, necesitamos activar el pulsador \textit{START} para que la bobina \textit{Q1} se active y a su vez active el contacto \textit{Q1} para que se quede ``enganchado''. Esto se quedará enganchado hasta que los contactos NC se abran, esto solo pasará cuando los pulsadores \textit{STOP} y \textit{PE}se active, esto ocasionará que los contactos se abran y se interrumpa el circuito. El circuito final se gráfica en la figura \ref{fig:motor y logo}.
\begin{figure}[]
\centering
\includegraphics[width=\linewidth]{CAM/cam39.png}
\caption{Resultado final motor y PLC LOGO.}
\label{fig:motor y logo}
\end{figure}
\begin{remark}
Para PLC necesitas cable fase y cable neutro. Para ladder necesitas cable negativo y cable negativo. En PLC LOGO, P1 es puerto positivo y P2 es puerto negativo (en la parte superior: Dispositivos de entrada) y sirven para dar energía al PLC, por lo tanto hay que energizarlas. De igual manera con la parte inferior (Dispositivos de salida), se debe energizar positivamente las entradas 1 y los dispositivos deben estarn en las salidas 2 hacia negativo.
\end{remark}
\begin{example}[Diseñe una faja transportadora con el PLC LOGO, sensores y tenga apagado automático]
Para este proyecto se usará parte de como  encender un motor con ladder explicado anteriormente.\\
\textbf{PLC:}\\
Empezamos cambiando las entradas del PLC, esta vez no contaremos con pulsadores, al ser automático usaremos \textbf{sensores inductivos NA} (etiquetados con \textit{S1}, \textit{S2} y \textit{S3}) y un pulsador de detención \textit{STOP}.
\begin{center}
\includegraphics[width=0.6\linewidth]{cam/cam40.png}
\textbf{No te olvides de energizar el PLC.}
\end{center}
Por la parte inferior del PLC, conectaremos dos bobinas para cada motor, pues son dos motores, y en paralelo colocaremos pilotos. No es recomendable colocarlos en serie.
\begin{center}
\includegraphics[width=0.6\linewidth]{cam/cam41.png}
\end{center}
\textbf{Ladder}:\\
Para ladder vamos a usar el mismo mecanismo de accionamiento del motor visto antes, en este ejemplo usaremos dos ``memorias'' (\textit{M2} y \textit{M3}) NC.
\begin{center}
\includegraphics[width=0.6\linewidth]{cam/cam42.png}
\end{center}
¿Por qué las memorias? Necesitamos que tenga un apagado automático, es decir, cuando el sensor de inicio de carrera (\textit{S1} y \textit{S2}) de ambas fajas no detecte material y el sensor de fin carrera (\textit{S3}) detecte que es último material, debe pasar cierto tiempo y que se apague ambas bajas.\\
Esto lo haremos estableciendo el sensor \textit{S3} como normalmente abierto, y en la misma rama contactos NC de \textit{S1} y \textit{S2}, de esta manera cuando el sensor \textit{S3} detecte material y los sensores \textit{S1} y \textit{S2} NO detecten nada, activará el \textbf{temporizador de desconexión} \textit{-T1} (5 seg). El temporizador mantendrá la energía hasta pasado un tiempo, luego de ello se apagará. Si se activa \textit{M1} y el sensor \textit{S1} no detecta nada, esta última rama activará el \textbf{temporizador de conexión}, que pasado un tiempo (15 seg), activará la bobina \textit{M2}, si esta se activa apagará el motor 1. Lo mismo pasará con la última rama. Si no detectan nada en sus respectivos sensores y \textit{M1} ha sido activado confirmando que todos los sensores no detectan nada, procede a abrir los contactos \textit{M2} y \textit{M3} para apagar motores.
\begin{center}
\includegraphics[width=0.6\linewidth]{cam/cam43.png}
\end{center}
\end{example}
%----------------------------------------------------------------------------------------
%	NEW CHAPTER
%----------------------------------------------------------------------------------------
\part{Software de telecomunicaciones}
\chapterimage{chapter_head_SFT.pdf} % Chapter heading image

\chapter{Unidad I}
\section{XDR}
\section{RPC y RMI}
Llamada a procedimiento remoto o RPC es una técnica que utiliza  el modelo cliente-servidor para ejecutar  tareas es un proceso diferente como podría ser en una computadora remota. A veces solamente se le llama como a una función o subrutina remota.
\subsection*{RPC}
\begin{enumerate}
\item El cliente hace la llamada al procedimiento remoto mediante un mensaje a través de la red. Este se detiene ya que es un proceso síncrono (se envía el mensaje y se queda ``callado'' hasta que reciba una respuesta), es decir, necesita una respuesta del servidor para poder continua su ejecución.
\item El servidor recibe la petición y desempaqueta el mensaje para extraer la información necesaria para realizar la tarea.
\item El servidor ejecuta la tarea. Con esto nos referimos que como cliente, enviamos la información a otra computadora (servidor) para que nos haga el ``trabajo sucio''.
\item El servidor crea un mensaje de respuesta para el cliente que incluye el resultado de la tarea que este le pidió realizar.
\item El cliente recibe y desempaqueta el mensaje de respuesta del servidor. Continua con su ejecución normal.
\end{enumerate}
\begin{notation}
El servidor siempre va estar activo, su función es atender a varios clientes a la vez.
\end{notation}
\subsection*{STUB}
Es la pieza de código que le permite al servidor ejecutar la tarea que se le asignó. Se encarga de proveer la información necesaria para que el servidor convierta las direcciones de los parámetros que el cliente envió en direcciones de memoria válidos dentro del ambiente del servidor. Parte del código que se encarga de ``traducir'' los entornos de ambos, en otras palabras le explica al servidor (o viceversa) como debe interpretar los datos. A manera de ejemplo (solo para explicar): El cliente solicita una suma de 15+83 y manda 1853 al servidor; el servidor no sabrá que significa ese número, el \textit{stub} le dirá: ``El primer y tercer dígito es el primer sumando, además el segundo y cuarto es otro número'', 
con esto el servidor sabrá cuales son los datos. El servidor hace la suma (89) con los números y devuelve : 0089. Este resultado viaja al cliente, sin no antes pasar por el \textit{stub} que le dice: ``El resultado de la suma es el número conformado por el cuarto dígito y el tercer dígito''. En conclusión, el \textit{stub} acomoda los datos para que los datos puedan ser procesados por cada lado.
La representación de datos en cliente y servidor (\textit{big-endian} o \textit{little-endian}) podría discrepar, el \textit{stub} también provee la información necesaria para solucionar esta situación.\\
Una inconveniente con esto es que RPC es dependiente del lenguaje de programación que se utilice.
\begin{definition}[Socket]
Los \textbf{sockets} sirven para la comunicación entre programas (en una primera medida), y para comenzar una conexión debemos crear un \textit{socket}; para dicha conexión se necesita una dirección IP y un puerto para realizar la conexión.
\end{definition}
\subsection{RMI}
Es un paquete de JAVA que permite manejar objetos (y sus respectivos métodos) de manera remota, para utilizar los recursos de un servidor de manera transparente para el usuario local. Es el analogo a RMS pero en JAVA, de igual manera al \textit{stub} aquí se le conoce como \textbf{Skeletons}. La diferencia más notable es que RPC es una solicitud remota mientras que RMI genera una \textbf{máquina virtual}, cuando mando una solicitud al servidor, es simular como si se estuviera ejecutando en el mismo lugar (servidor).

\begin{enumerate}
\item El servidor proporciona un servicio RMI y el cliente llama a métodos del objeto ofrecido por el servicio.
\item El servicio RMI se debe registrar en un servicio de consulta para permitir a los clientes encontrar el servicio.
\item El J2SE incluye una aplicación llamada \textit{rmiregistry}, que lanza un proceso que permite registrar servicios RMI mediante un nombre. Este nombre identifica al servicio, esto se hace as´ı ya que en una maquina puede haber diferentes servicios.
\item Una vez que el servicio se ha registrado, el servidor esperará a que lleguen peticiones RMI desde
los clientes.
\item El cliente solicita el servicio mediante el nombre con el que fue registrado y obtiene una referencia al objeto remoto
\item El formato utilizado por RMI para representar una referencia al objeto remoto es el siguiente \url{rmi://hostname:puerto/nombreServicio}.
\item Una vez obtenida la referencia remota (ya sea mediante \textit{rmiregistry}, leyendo el URL de un archivo,...) los clientes pueden enviar mensajes como si se tratase de objetos ejecutándose en la misma máquina virtual.
\item Los detalles de red de las peticiones y las repuestas son transparentes para el desarrollador. Esto se consigue mediante el uso de \textit{stub} (a partir de la versión 1.2, ya que en la versión 1.1 se requería generar un skeleton).
\end{enumerate}
\begin{definition}[Skeleton]
La clase skeleton es una clase generada por RMI. Esta clase es la responsable de comunicarse con el \textit{stub} durante la comunicación RMI. Debe reconstruir los parámetros para formar los tipos primitivos y objetos, lo que es conocido como \textit{unmarshalling}.
\end{definition}
\begin{figure}[H]
\centering
\includegraphics[width=0.5\linewidth]{soft/soft1.png}
\caption{Arquitectura RMI}
\end{figure}
\section{CORBA}
CORBA (\textit{Common Object Request Broker Architectur}e) es una arquitectura abierta desarrollada por los miembros del OMG (\textit{Object Management Group}). Desde 1989 la misión del OMG ha sido la especificación de una arquitectura para un \textbf{bus software abierto}, o \textit{Object Request Broker} (ORB), sobre el que diversos componentes de objetos escritos por diferentes vendedores puedan interoperar a través de la red y de los sistemas operativos. Este estándar permite que los objetos CORBA realicen llamadas entre ellos sin conocer dónde residen los objetos a los que acceden o en qué lenguaje están implementados estos últimos. OMG define un lenguaje (\textit{Interface Definition Language}: IDL) usado para definir las interfaces de los objetos CORBA.
\subsubsection*{Características}:
\begin{itemize}
\item Los objetos CORBA pueden estar en cualquier sitio de la red
\item Los objetos CORBA pueden interoperar con objetos de otras plataformas
\item Los objetos CORBA pueden escribirse en cualquier lenguaje de programación para el que exista un "mapeado" (correspondencia) desde el OMG IDL a dicho lenguaje. Actualmente las correspondencias definidas incluyen Java, C++, C, Smalltalk, COBOL, y Ada).
\end{itemize}
La plataforma Java 2 SE 1.4 proporciona un ORB que cumple con la especificación de CORBA 2.3.1 y dos modelos de programación CORBA que utilizan el ORB CORBA para Java y el Internet InterORB Protocol (IIOP). Los dos modelos de programación son el \textit{modelo de programación RMI} (o RMI-IIOP), y el \textit{modelo de programación IDL} (o Java IDL).
\subsection{CORBA y EAI}
La integraciones de aplicaciones para empresa (EAI( se puede conseguir en cuatro fases:
\begin{itemize}
\item Nivel de datos
\item Nivel de interfaz de aplicación
\item Nivel de lógica de negocio.
\item Nivel de presentación
\end{itemize}
Normalmente se comienza con la integración a nivel de \textbf{datos}, y se continúa con la integración a nivel de \textbf{interfaz}, y posteriormente a nivel de \textbf{lógica del negocio}. En ambas fases (interfaz y negocio) el objetivo es el \textbf{reusar} la funcionalidad de las aplicaciones existentes.\\
El uso de aplicaciones existentes se conseguirá por:
\begin{itemize}
\item \textbf{Modificando el código fuente}: Debemos definir una o más \textbf{APIs} con las operaciones que necesitamos para acceder de forma externa, y conectar dichas operaciones con el código existente. CORBA es una tecnología adecuada para realizar esto. Para cada interfaz que queramos añadir a la aplicación deberemos definir un \textbf{nuevo objeto CORBA distribuido}, declararemos las operaciones de la interfaz y conectaremos cada operación con el código fuente de la aplicación existente.
\item \textbf{Mediante técnicas como screeen scraping o emulación de terminales}: Si el código fuente no está disponible, o si no disponemos de las herramientas necesarias para recompilar la apliación a partir del código fuente, o simplemente no queremos modificar el código fuente, podemos utilizar estas técnicas. \textbf{Sreen scraping} y emulación de terminales son apropiadas principalmente para aplicaciones basadas en caracteres, los wrappers simulan lo que teclea el cliente para realizar las funciones de las aplicaciones existentes y leer la pantalla para extraer los resultados. En este caso el uso de CORBA vuelve a resultar adecuado.
\end{itemize}
\subsection{Arquitectura CORBA}
CORBA consiste en la especificación de un modelo de objetos distribuido idependiente del lenguaje, sistema operativo y plataforma. La parte principal de CORBA es el ORB (Object Request Broker). ORB actúa como un bus de mensajes que transporta peticiones de invocación de operaciones y sus resultados sobre objetos CORBA, proporcionando la infraestructura de comunicación necesaria de forma que oculte todos los detalles de la comunicación entre objetos distribuidos. La \textbf{transparencia de localización} es la capacidad de acceder e invocar operaciones sobre un objeto CORBA sin necesidad de conocer dónde reside dicho objeto. La idea es que debería ser igualmente sencillo invocar una operación residente en una máquina remota que un método de un objeto en el mismo espacio de direcciones. La \textbf{transparencia de lenguaje de programación} proporciona la libertad de implementar la funcionalidad encapsulada un objeto usando el lenguaje más adecuado, bien por las habilidades de los programadores, la idoneidad del lenguaje para la tarea específica, o la elección de un ``tercer'' desarrollador que proporciona componentes ya creados (off-the-shelf components). La clave de este grado de libertad es un \textbf{lenguaje de definición de interfaces} que es neutral con respecto a la implementación, y que proporciona una separación entre la interfaz y su implementación. Dicho lenguaje se denomina \textbf{\textit{Interface Denifition Language}} (IDL), y se utiliza para definir las interfaces de los objetos, independientemente del lenguaje de programación en el que estén implementados. Es un lenguaje fuertemente declarativo, con un rico conjunto de tipos de datos para describir parámetros complejos. Una interfaz IDL actúa como un contrato entre los desarrolladores de objetos y los eventuales usuarios de dichos interfaces. También permite a los usuarios de CORBA compilar las definiciones de las interfaces en código oculto para la transmisión de peticiones de invocaciones a través de las redes y las arquitecturas de las máquinas sin necesitar ningún conocimiento sobre el protocolo de red utilizado, o incluso la localización del objeto involucrado.
\section{SOAP}
SOAP es el acrónimo de “Simple Object Access Protocol” y es el protocolo que se oculta tras la tecnología que comúnmente denominamos “Web Services” o servicios web. SOAP es un protocolo extraordinariamente complejo pensado para dar soluciones a casi cualquier necesidad en lo que a comunicaciones se refiere, incluyendo aspectos avanzados de seguridad, transaccionalidad, mensajería asegurada y demás.\\
Este tipo de servicio está basado en el intercambio de información mediante \textit{xml} y su principal protocolo de funcionamiento es el \textit{http}, sin embargo también puede ser enviado mediante otros protocolos como \textit{ftp}, \textit{pop3}, \textit{tcp} y colas de mensajería. \\
\textbf{Ventajas}:
\begin{itemize}
\item Permite agregar metadatos en sus atributos.
\item Permite definir espacios de nombres para evitar la ambigüedad en su descripción.
\item Cuenta con métodos de validación más potentes que los de \textit{json}.
\item Funciona bien en ambientes empresariales es decir comunicaciones de servidor a servidor.
\end{itemize}
Al usar SOAP con el protocolo \textit{http} el tipo de contenido del mensaje que utiliza este protocolo debe establecerse en \textit{xml} que requiere la agregación de una línea que contiene SOAPAction en que debe estar presente en la cabecera \textit{http}. Por último podemos decir que la entrada SOAPAction permite que los firewalls del destino identifiquen los mensajes SOAP.\\
Debido a estas cualidades, SOAP es ampliamente utilizado en entornos empresariales. Donde es requerida la existencia de un contrato claro entre cliente y servidor, y además la seguridad en las comunicaciones es muy importante. Entre sus desventajas está que al estar ampliamente estandarizado, es poco flexible y suele haber muchos errores a la hora de desarollo si no se conocen dichos estándares.  Al utilizar el protocolo TCP también tiene un peor rendimiento que otro tipo de Web Services.
\begin{vocabulary}[XML]
El formato estándar “Extensible Markup Language (XML), tiene varias características que lo hacen conveniente, entre las que podemos destacar: Es un estándar abierto, flexible y ampliamente utilizado para almacenar, publicar e intercambiar cualquier tipo de información.
\end{vocabulary}
\chapterimage{chapter_head_SFT.pdf} % Chapter heading image
\chapter{Unidad II}
\section{REST}
REST deriva de ``REpresentational State Transfer'', que traducido vendría a ser ``transferencia de representación de estado'', lo que tampoco aclara mucho, pero contiene la clave de lo que significa.  Porque la clave de REST es que un servicio REST \textbf{no tiene estado} (es stateless), lo que quiere decir que, entre dos llamadas cualesquiera, el servicio pierde todos sus datos. El estado lo mantiene el cliente y por lo tanto es el cliente quien debe pasar el estado en \textbf{cada llamada}. Si quiero que un servicio REST me recuerde, debo pasarle quien soy en cada llamada. Eso puede ser un usuario y una contraseña, un token o cualquier otro tipo de credenciales, pero debo pasarlas en cada llamada. Y lo mismo aplica para el resto de información.\\
Principios:
\begin{itemize}
\item Debe ser un sistema \textbf{cliente-servidor}.
\item Tiene que ser \textbf{sin estado}, es decir, no hay necesidad de que los servicios guarden las sesiones de los usuarios (cada petición al servicio tiene que ser independiente de las demás).
\item Debe soportar un sistema de cachés: la infraestructura de la red debería soportar caché en diferentes niveles.
\item Debe ser un sistema uniformemente accesible (con una interfaz uniforme): Esta restricción define cómo debe ser la interfaz entre clientes y servidores. La idea es simplificar y desacoplar la arquitectura, permitiendo que cada una de sus partes puede evolucionar de forma independiente. Una interfaz uniforme se debe caracterizar por:
\begin{itemize}
\item Estar basada en recursos: La abstracción utilizada para representar la información y los datos en REST es el recurso, y cada recurso debe poder ser accedido mediante una URI (Uniform Resource Identifier).
\item Orientado a representaciones: La interacción con los servicios tiene lugar a través de las representaciones de los recursos que conforman dicho servicio. Un recurso referenciado por una URI puede tener diferentes formatos (representaciones). Diferentes plataformas requieren formatos diferentes. Por ejemplo, los navegadores necesitan HTML, JavaScript requiere JSON (JavaScript Object Notation), y una aplicación Java puede necesitar XML.
\item Interfaz restringida: Se utiliza un pequeño conjunto de métodos bien definidos para manipular los recursos.
\item Uso de mensajes auto-descriptivos: cada mensaje debe incluir la suficiente información como para describir cómo procesar el mensaje. Por ejemplo, se puede indicar cómo "parsear" el mensaje indicando el tipo de contenido del mismo (xml, html, texto,...).
\item Uso de Hipermedia como máquina de estados de la aplicacion (HATEOAS): Los propios formatos de los datos son los que "dirigen" las transiciones entre estados de la aplicación. Como veremos más adelante con más detalle, el uso de HATEOAS (Hypermedia As The Engine Of Application State), va a permitir transferir de forma explícita el estado de la aplicacion en los mensajes intercambiados, y por lo tanto, realizar interacciones con estado.
\end{itemize}
\item Tiene que ser un sistema por capas: un cliente no puede "discernir" si está accediendo directamente al servidor, o a algún intermediario. Las "capas" intermedias van a permitir soprtar la escalabilidad, así como reforzar las políticas de seguridad
\end{itemize}
REST crea \textbf{una petición HTTP} que contiene toda la información necesaria, es decir, un REQUEST a un servidor tiene toda la información necesaria y solo espera una RESPONSE, ósea una respuesta en concreto. Se apoya sobre un protocolo que es el que se utiliza para las páginas web, que es \textbf{HTTP}, es un protocolo que existe hace muchos años y que ya está consolidado, no se tiene que inventar ni realizar cosas nuevas. Se apoya en los métodos de HTTP:
\begin{itemize}
\item Post: Para crear recursos nuevos.
\item Get: Para obtener un listado o un recurso en concreto.
\item Put: Para modificar.
\item Patch: Para modificar un recurso que no es un recurso de un dato, por ejemplo.
\item Delete: Para borrar un recurso, un dato por ejemplo de nuestra base de datos.
\end{itemize}
Todos los objetos se manipulan mediante URI, por ejemplo, si tenemos un recurso usuario y queremos acceder a un usuario en concreto nuestra URI seria /user/identificadordelobjeto, con eso ya tendríamos un servicio USER preparado para obtener la información de un usuario, dado un ID.
\subsection{REST vs SOAP}
REST es un conjunto de pautas que ofrece una implementación flexible, mientras que SOAP es un protocolo con requisitos específicos, como en el caso de la mensajería XML.\\
No obstante, cada uno ofrece ventajas diferentes: REST está considerado relativamente sencillo, no trabaja solo con XML, es más rápido y, en comparación con SOAP, más ligero. La libertad que caracteriza a REST a la hora de elegir si trabajar con XML (a menudo recurre a JSON) caracteriza a SOAP a la hora de elegir el protocolo. Aunque HTTP es a menudo la opción elegida, teóricamente SOAP funciona también en combinación con FTP, SMTP u otros protocolos.\\
Respecto a la seguridad, SOAP supone una gran ventaja: WS-Security (especificaciones para los criterios de seguridad respecto a servicios web) está anclado en el protocolo de red. También el tratamiento de los errores está mejor regulado en SOAP, porque incorpora directamente una función para la repetición de la solicitud.
\begin{table}[H]
\resizebox{\textwidth}{!}{
\begin{tabular}{ll}
\rowcolor[HTML]{DAE8FC} 
REST                                                                                                                                      & SOAP                                                \\
Pocas operaciones con muchos recursos                                                                                                     & Muchas operaciones con pocos recursos               \\
\begin{tabular}[c]{@{}l@{}}Se centra en la escalabilidad y rendimiento a gran escala\\ para sistemas distribuidos hipermedia\end{tabular} & Se centra en el diseño de aplicaciones distribuidas \\
HTTP (GET, POST, PUT, DEL)                                                                                                                & SMTP, HTTP POST, MQ                                 \\
XML auto descriptivo                                                                                                                      & Tipaso fuerte, XML, Schema                          \\
Sincrono                                                                                                                                  & Sincrono y asiscrono                                \\
HTTPS                                                                                                                                     & WS security                                         \\
Comunicaión P2P segura                                                                                                                    & Comunicación origen a destino seguro               
\end{tabular}}
\end{table}
\section{UML}
El Lenguaje Unificado de Modelado (UML) desempeña un rol importante no solo en el desarrollo de software, sino también en los sistemas que no tienen software en muchas industrias, ya que es una forma de mostrar visualmente el comportamiento y la estructura de un sistema o proceso.
\subsection*{Ventajas}
\begin{itemize}
\item Simplifica las complejidades 
\item Mantiene abiertas las líneas de comunicación 
\item Automatiza la producción de software y los procesos  
\item Ayuda a resolver los problemas arquitectónicos constantes 
\item Aumenta la calidad del trabajo 
\item Reduce los costos y el tiempo de comercialización 
\end{itemize}
Vamos a hablar sobre diagrama de \textbf{clases} UML las demás puedes revisar en internet.
\subsection{Clase abstracta}
Las cosas que existen y que nos rodean se agrupan naturalmente en categorías. Una clase es una categoría o grupo de cosas que tienen atributos (propiedades) y acciones similares. Un rectángulo es el símbolo que representa a la clase, y se divide en tres áreas. Un diagrama de clases está formado por varios rectángulos de este tipo conectados por líneas que representan las asociaciones o maneras en que las clases se relacionan entre si. No se puede crear una clase que defina a la clase abstracta pues este solo sirve como un contenerdor o plantilla de clases. La clase abstracta se escribe en cursiva.
\subsection{Clase concreta}
Cuando una hemos usado una clase abstracta como plantilla para crear un objeto. Por ejemplo, clase abstracta animal, y clase concreta perro.
\begin{itemize}
\item \textbf{Clase/nombre}: Indica la clase, puedes encontrar una forma de englobar una categoría de objetos, por ejemplo: aviones, vehículos, muebles, dispositivos móviles, etc.
\item \textbf{Atributo}: Un dato importante que contiene valores que describen cada instancia de esa clase. Incluye variable y propiedades.
\item \textbf{Métodos}: Operaciones o funciones especifican el comportamiento de la clase.
\end{itemize}
\begin{figure}[H]
\centering
\includegraphics[width=0.5\linewidth]{soft/soft4.png}
\caption{Bloque de clases}
\end{figure}
\subsubsection*{Visibilidad}
Define la accesibilidad para ese atributo o método.
\begin{itemize}
\item \textbf{Privado (-)}: Ni una clase o subclase puede acceder a ellos.
\item \textbf{Público (+)}: Cualquier clase puede acceder a ello.
\item \textbf{Protegido (\#)}: Solo la misma clase o sus subclases puede acceder a ellos.
\item \textbf{Paquete/defecto (~)}: Puede ser usada siempre y cuando pertenezcan al mismo paquete. Paquete es el contenedor de clases.
\end{itemize}
\subsection{Relaciones}
\begin{figure}[H]
\centering
\includegraphics[width=0.3\linewidth]{soft/soft2.png}
\caption{Relaciones UML}
\end{figure}
\subsubsection*{Asociación}
\begin{flushright}
\textit{Usa un verbo que funcione de manera bidireccionalmente para ambas clases: trabaja, compra, hace, etc.}
\end{flushright}
Crea un vinculo entre dos clases pero no dependen entre si. Por ejemplo, antena \textbf{transmite} ondas electromagnéticas. No hay dependencia pero están relacionadas.
\subsubsection*{Herencia o heritaje}
\begin{flushright}
\textit{Relación tipo ``es un''}
\end{flushright}
Podemos unir clases a otra clase, de esta forma, las primera serán llamadas clases derivadas o subclases y la que reune a todas será llamada superclase o clase principal. Esto nos índica que las subclases \textbf{heredan} todos los atributos y métodos. Además cada subclase pueden tener sus atributos propios. Para definir herencia se debe usar la flecha indicando cual será la clase principal.
\subsubsection*{Agregación}
\begin{flushright}
\textit{Relación tipo ``está formado por''}
\end{flushright}
Indica que una parte puede ser parte de un todo pero no tiene que serlo sin dejar de existir. Por ejemplo, un perro es independiente como tal, si unimos varios perros forman una jauría. El perro deja la jauria y sigue siendo perro, no deja de existir. La relación de una parte que puede existir fuera del todo.
\subsubsection*{Composición}
\begin{flushright}
\textit{Relación tipo ``es parte de''}
\end{flushright}
Lo contrario a asociación, cuando un objeto derivado no puede existir sin el objeto principal: Baño pertenece al estadio, si el estadio se demuele, el baño deja de existir.
\subsubsection{Multiplicidad}
Indica la cantidad de cada clase, estos se escriben en los vínculos entre clases, las variable son:
\begin{itemize}
\item \textbf{0..1}: Cero o uno.
\item \textbf{n}: Indica la cantidad especifica.
\item \textbf{1..*}: Uno a muchos.
\item \textbf{m..n}: Indica un rango específico.
\end{itemize}
\begin{figure}[H]
\centering
\includegraphics[width=0.9\linewidth]{soft/soft3.png}
\caption{Ejemplo diagrama de clases.}
\end{figure}
\section{Patrón de diseño}
Los patrones de diseño o design patterns, son una solución general, reutilizable y aplicable a diferentes problemas de diseño de software. Se trata de plantillas que identifican problemas en el sistema y proporcionan soluciones apropiadas a problemas generales a los que se han enfrentado los desarrolladores durante un largo periodo de tiempo, a través de prueba y error.
\subsection{Tipos de patrones}
\begin{itemize}
\item \textbf{Patrones creacionales}: Los patrones de creación proporcionan diversos mecanismos de creación de objetos, que aumentan la \textbf{flexibilidad y la reutilización del código existente de una manera adecuada a la situación}. Esto le da al programa más flexibilidad para decidir qué objetos deben crearse para un caso de uso dado.
\begin{itemize}
\item Abstract Factory: En este patrón, una interfaz crea conjuntos o familias de objetos relacionados sin especificar el nombre de la clase.
\item Builder Patterns: Permite producir diferentes tipos y representaciones de un objeto utilizando el mismo código de construcción. Se utiliza para la creación etapa por etapa de un objeto complejo combinando objetos simples. La creación final de objetos depende de las etapas del proceso creativo, pero es independiente de otros objetos.
\item Factory Method: Proporciona una interfaz para crear objetos en una superclase, pero permite que las subclases alteren el tipo de objetos que se crearán. Proporciona instanciación de objetos implícita a través de interfaces comunes
\item Prototype: Permite copiar objetos existentes sin hacer que su código dependa de sus clases. Se utiliza para restringir las operaciones de memoria / base de datos manteniendo la modificación al mínimo utilizando copias de objetos.
\item Singleton: Este patrón de diseño restringe la creación de instancias de una clase a un único objeto. 
\end{itemize}
\item \textbf{Patrones estructurales}: Facilitan soluciones y estándares eficientes con respecto a las composiciones de clase y las estructuras de objetos. El concepto de herencia se utiliza para componer interfaces y definir formas de componer objetos para obtener nuevas funcionalidades.
\begin{itemize}
\item Adapter: Se utiliza para vincular dos interfaces que no son compatibles y utilizan sus funcionalidades. El adaptador permite que las clases trabajen juntas de otra manera que no podrían al ser interfaces incompatibles.
\item Bridge: En este patrón hay una alteración estructural en las clases principales y de implementador de interfaz sin tener ningún efecto entre ellas. Estas dos clases pueden desarrollarse de manera independiente y solo se conectan utilizando una interfaz como puente.
\item Composite: Se usa para agrupar objetos como un solo objeto. Permite componer objetos en estructuras de árbol y luego trabajar con estas estructuras como si fueran objetos individuales.
\item Decorator: Este patrón restringe la alteración de la estructura del objeto mientras se le agrega una nueva funcionalidad. La clase inicial permanece inalterada mientras que una clase decorator proporciona capacidades adicionales.
\item Facade: Proporciona una interfaz simplificada para una biblioteca, un marco o cualquier otro conjunto complejo de clases.
\item Flyweight: El patrón Flyweight se usa para reducir el uso de memoria y mejorar el rendimiento al reducir la creación de objetos. El patrón busca objetos similares que ya existen para reutilizarlo en lugar de crear otros nuevos que sean similares.
\item Proxy: Se utiliza para crear objetos que pueden representar funciones de otras clases u objetos y la interfaz se utiliza para acceder a estas funcionalidades
\end{itemize}
\item \textbf{Patrones de comportamiento}: El patrón de comportamiento se ocupa de la comunicación entre objetos de clase. Se utilizan para detectar la presencia de patrones de comunicación ya presentes y pueden manipular estos patrones. Estos patrones de diseño están específicamente relacionados con la comunicación entre objetos.
\begin{itemize}
\item Chain of responsibility: El patrón de diseño Chain of Responsibility es un patrón de comportamiento que evita acoplar el emisor de una petición a su receptor dando a más de un objeto la posibilidad de responder a una petición.
\item Command: Convierte una solicitud en un objeto independiente que contiene toda la información sobre la solicitud. Esta transformación permite parametrizar métodos con diferentes solicitudes, retrasar o poner en cola la ejecución de una solicitud y respaldar operaciones que no se pueden deshacer.
\item Interpreter: Se utiliza para evaluar el lenguaje o la expresión al crear una interfaz que indique el contexto para la interpretación.
\item Iterator: Su utilidad es proporcionar acceso secuencial a un número de elementos presentes dentro de un objeto de colección sin realizar ningún intercambio de información relevante.
\item Mediator: Este patrón proporciona una comunicación fácil a través de su clase que permite la comunicación para varias clases.
\item Memento: El patrón Memento permite recorrer elementos de una colección sin exponer su representación subyacente.
\item Observer: Permite definir un mecanismo de suscripción para notificar a varios objetos sobre cualquier evento que le suceda al objeto que está siendo observado.
\item State: En el patrón state, el comportamiento de una clase varía con su estado y, por lo tanto, está representado por el objeto de contexto.
\item Strategy: Permite definir una familia de algoritmos, poner cada uno de ellos en una clase separada y hacer que sus objetos sean intercambiables.
\item Template method: Se usa con componentes que tienen similitud donde se puede implementar una plantilla del código para probar ambos componentes. El código se puede cambiar con pequeñas modificaciones.
\item Visitor: El propósito de un patrón Visitor es definir una nueva operación sin introducir las modificaciones a una estructura de objeto existente.
\end{itemize}
\end{itemize}
%----------------------------------------------------------------------------------------
%	NEW CHAPTER
%----------------------------------------------------------------------------------------
\part{Microelectrónica en RF}
\chapterimage{chapter_head_MR.pdf} % Chapter heading image
\chapter{Osciladores}
\section{Ondas electromagnéticas}
Las ecuaciones de Maxwell, propuestas por primera vez por James Clerk Maxwell, reunen las leyes experimentales de la electricidad y el magnetismo. Estas ecuaciones desempeñan en el electromagnetismo un papel análogo, a las leyes de Newton en la Mecánica. Con estas ecuaciones Maxwell pudo demostrar la existencia de las ondas electromagnéticas. Estas ondas electromagnéticas son originadas por cargas eléctricas aceleradas y fueron producidas por primera vez en el laboratorio por Heinrich Hertz en 1887.
Maxwell mostró que la velocidad de las ondas electromagnéticas en el \textbf{espacio vacío} es:
\begin{equation}
c=\frac{1}{\sqrt{\mu_0\epsilon_0}}\approx 3\basedec{8} m/s
\label{eq:vel luz}
\end{equation}
En otro medio es:
\begin{equation}
v=\frac{c}{\sqrt{\epsilon_r}}
\label{eq:velocidad relativa}
\end{equation}
Los campos E y B de la OEM son \textbf{perpendiculares} entre sí. Ambos son perpendiculares a la dirección de propagación de la onda (onda transversal).Las magnitudes de E y B están en fase y se relacionan por la expresión:
\begin{displaymath}
E=c\cdot B
\end{displaymath}
\begin{figure}[H]
\centering
\includegraphics[width=0.8\linewidth]{mrf/mrf1.png}
\caption{Ondas electromagnéticas}
\end{figure}
\begin{figure}[H]
\centering
\includegraphics[width=0.9\linewidth]{mrf/mrf2.png}
\caption{Espectro electromagnético}
\end{figure}
Las ondas electromagnéticas se presentan cuando se aceleran las cargas eléctricas. Las antenas deben tener ciertas impedancias, en esto es importante el tamaño de las antenas par evitar ROE, es por eso el tamaño el tamaño de las antenas deben estar en resonancia con la longitud de onda. Para este curso, trabajaremos con frecuencias desde $\basedec{3}-\basedec{6}$Hz aproximadamente. Altas frecuencias ya estaríamos hablando de microondas.
\begin{figure}[H]
\centering
\includegraphics[width=0.9\linewidth]{mrf/mrf3.png}
\caption{Espectro electromagnético con ejemplos.}
\end{figure}
\subsection*{Infrarojos}
La radiación infrarroja (IR) es una radiación electromagnética cuya longitud de onda comprende desde los 760-780 nm, limitando con el color rojo en la zona visible del espectro, hasta los 10.000 o 15.000 nm (según autores), limitando con las microondas. 
\subsection*{Ultravioleta}
Se denomina radiación ultravioleta o radiación UV a la radiación electromagnética cuya longitud de onda está comprendida aproximadamente entre los 100 nm ($100\cdot\basedec{-9}$ m) y los 400 nm ($400\cdot\basedec{-9}$ m). Su nombre proviene del hecho de que su rango empieza desde longitudes de onda más cortas de lo que el ojo humano identifica como luz violeta, pero dicha luz o longitud de onda, es invisible al ojo humano al estar por encima del espectro visible. Esta radiación es parte integrante de los rayos solares y produce varios efectos en la salud al ser una radiación entre no-ionizante e ionizante.
\subsection*{Rayos X}
Los rayos X son ondas de energía extremadamente alta con longitudes de onda entre 0.03 y 3 nanómetros, no mucho más largas que un átomo. Los rayos X son emitidos por fuentes que producen temperaturas muy altas como la corona del sol, que es mucho más caliente que la superficie.
\subsection*{Rayos gamma}
Rayos gamma. Longitud de onda igual o menor a $\basedec{-17}$ m. Frecuencia igual o superior a $\basedec{25}$ Hz. La región de los rayos gamma del espectro electromagnético se solapa con la de los rayos X. La radiación gamma es producto principalmente de los núcleos inestables de materiales radiactivos artificiales y naturales. También son un componente de la llamada radiación cósmica, radiación que baña la Tierra procedente del espacio exterior. Son rayos muy penetrantes y producen daños serios si son absorbidos por tejidos vivos. En consecuencia, quienes trabajan cerca de este tipo de radiación peligrosa, deben estar protegidos con materiales de gran absorción, como gruesas capas de plomo.
\section{Distorsión}
Tengamos un sistema donde la respuesta ideal es \textit{x(t)}, en realidad no se obtendrá esa señal, por el contrario, obtendremos \textit{y(t)}, esta señal no es la ideal pues puede haberse visto alterada por distintos factores. Se define el error como:
\begin{equation}
e(t)=y(t)-x(t)
\end{equation}
Con esto en mente, podemos definir la distorsión como:
\begin{definition}[Distorsión]
Es cualquier cambio en una señal que altera su forma de onda básica (en el dominio del tiempo) o bien, altera la relación entre sus componentes espectrales (domino de la frecuencia). La distorsión puede ser del tipo \textbf{lineal} o del tipo \textbf{no lineal}. Relación de las potencia medias del error y de la señal.
\begin{equation}
D=\frac{\langle e^2(t)\rangle}{\langle y^2(t)\rangle}=\frac{P_e}{P_y}
\end{equation}
Donde:
\begin{itemize}
\item $e(t)$: Error.
\item $y(t)$: Señal obtenida.
\item $P_e$: Potencia del error. (W)
\item $P_y$: Potencia de la señal recibida. (W)
\end{itemize}
\end{definition}
La distorsión nace de los elementos reales: transmisor, línea de transmisión, antenas, etc. La distorsión no solo afecta la amplitud, puede afectar fase, información dada, etc.
\subsection{Distorsión lineal}\index{Distorsión lineal}
Una distorsión es lineal cuando la respuesta en frecuencia del circuito que trata a dicha
señal no es de una amplitud constante, o cuando se producen variaciones de fase en dicha señal.
En otras palabras, una distorsión lineal provoca el mismo porcentaje de error en una señal grande
que en una pequeña.\\
Existen distorsión de amplitud en la conversión de FM-AM por ejemplo. Ecos y reflexiones múltiples, los rebotes o propagación con trayectoria.\\
En un canal sin distorsión, supongamos que tenemos la siguiente señal:
\begin{displaymath}
y(t)=G\cdot x(t-\tau)
\end{displaymath}
Tenemos un desfase en $\tau$. Nuestra respuesta el lineal e invariante con el tiempo; en amplitud es contante con $\omega$ y la respuesta en fase es lineal con $\omega$.
\begin{figure}[H]
\centering
\includegraphics[width=.8\linewidth]{mrf/mrf5.png}
\caption{Canal de distorsión.}
\end{figure}
En una \textbf{distorsión de amplitud}, la respuesta espectral varía en amplitud con la frecuencia, la modulación PM/FM genera una modulación AM. 
\begin{figure}[H]
\centering
\includegraphics[width=.8\linewidth]{mrf/mrf6.png}
\caption{Distorsión en FM-AM.}
\end{figure}
Para \textbf{distorsión de fase}, 
\begin{figure}[H]
\centering
\includegraphics[width=.8\linewidth]{mrf/mrf7.png}
\caption{Distorsión de fase: retardo no uniforme.}
\end{figure}
\subsection*{Distorsión lineal por ecos}
\subsubsection*{Ecos y reflexiones múltiples}
En la actualidad existen \textit{n} portadoras de donde sea (artificiales y naturales), para un sistema OFDM, nuestra señal es dividida en pedazos pequeños o portadoras desfasadas, que hacen que reboten en distintos puntos antes de llegar al receptor. El objetivo es evitar que la toda la información se pierda por perdida de línea.
\subsubsection*{Distorsión por ecos}
Las señales que se retardan pueden sumarse a la fundamental, estos pueden ser causadas por ecos, es decir tienen la misma información pero están corridas en el tiempo, esto produce respuestas en amplitud y fase ondulada. Para resolver este problema usamos un ecualizador para cancelar ecos.
\begin{figure}[H]
\centering
\includegraphics[width=.8\linewidth]{mrf/mrf8.png}
\caption{Cancelar ecos: receptor Rake.}
\end{figure}
Si hay llegado múltiples portadoras, habrá un amplificador para ajustar la amplitud a la llegada, luego de ellos se les añadirá un retardo ajustable (aparte del retardo con el que llegó la señal) hasta que los retardos de las portadoras coincidan con un múltiplo del receptor principal. Finalmente resta la señal a la señal principal (Fundamentos del receptor Rake).
\begin{example}[Considere un sistema formado por un transmisor de TV analógica y una antena unidos por un cable coaxial, que consideramos sin pérdidas, de 150m de longitud y lleno de un dieléctrico de constante dieléctrica $\epsilon=2$. La señal de televisión está formada por una portadora de 600 MHz con una banda de modulación de 7.5 MHz. El coeficiente de reflexión a la salida del transmisor es de -10dB, y por un fallo en la conexión a la antena se produce una reflexión a su entrada de -5dB.]
\textbf{Determinando el retardo de la señal a lo largo del cable}:\\
Para el cable coaxial (que no es el vacío), el modo TEM, la velocidad de propagación usando la ecuación \ref{eq:velocidad relativa} es:
\begin{displaymath}
v=\frac{3\basedec{8}}{\sqrt{2}}=212\basedec{6}m/s
\end{displaymath}
Usando la ecuación del tiempo en función de la longitud:
\begin{displaymath}
\tau=\frac{150m}{212\basedec{6}m/s}=707.5ns
\end{displaymath}
Este retardo nos servirá solo si tenemos una medida con quien comparar, de lo contrario es imperceptible.
\textbf{¿Cuál será la amplitud de la señal reflejada en ambos extremos respecto de la señal directa?}\\
Aproximadamente, y si la reflexión es pequeña, el eco generado por reflexión tiene una relación de amplitud y fase en tensión con la señal directa dada por:
\begin{displaymath}
\frac{Eco}{Señal}\approx\Gamma_1\Gamma_2\exp(-2jkL)
\end{displaymath}
Pasando las unidades de decibelios a unidades lineales (recordando que son decibelios de voltaje) tenemos:
\begin{displaymath}
Eco\approx 0.1736 Señal
\end{displaymath}
La parte $\exp(-2jkL)$ viene del movimiento senoidal/cosenoidal y puede ser tomada como +1 y -1 pues son los máximos.\\
\textbf{Determine el rizado en amplitud y fase de la función de transferencia}\\
Si las reflexiones son pequeñas, y salvando un facto de fase, la función de transferencia puede aproximarse por:
\begin{displaymath}
H(\omega)\approx 1+\Gamma_1\Gamma_2\exp(-2jkL)=1+\Gamma_1\Gamma_2\exp(-2j\omega\tau)
\end{displaymath}
La fase de reflexión varía dentro de la banda $2(\omega_2-\omega_1)\tau=66.9 rad$ (más de diez vueltas), y, por tanto, los valores máximos y mínimos de amplitud y fase dentro de la banda serán:
\begin{align*}
|H(\omega)|_{máx}\approx 1+\Gamma_1\Gamma_2=1.18 \simeq 1.35dB\\
|H(\omega)|_{min}\approx 1-\Gamma_1\Gamma_2=0.83 \simeq -1.70dB\\
arg\left[\right]_{máx}\approx atan(\Gamma_1\Gamma_2)=0.176 rad \simeq 10.1°\\
arg\left[\right]_{min}\approx a-tan(\Gamma_1\Gamma_2)=-0.176 rad \simeq -10.1°
\end{align*}
\end{example}
\subsection{Distorsión no lineal}\index{Distorsión no lineal}
Esta distorsión es más difícil de tratar, tiene un comportamiento aleatorio, al no ser lineales las funciones de transferencia poseen valores cuadráticos.
\begin{figure}[H]
\centering
\includegraphics[width=0.8\linewidth]{mrf/mrf9.png}
\caption{Distorsión no lineal: Función de transferencia polinómica.}
\end{figure}
Vemos que el resultado es de una forma polinómica, y nosotros al querer de bajar de grado los polinomios usando identidades trigonométricas vamos a generar constantes, que se tratan como componentes en \textbf{continua}, estos pueden ser eliminados con condensadores. Tenemos saturación, componentes que hacen que la amplitud de nuestra señal se eleve considerablemente, estos consumirán energía y se controlan con atenuación pasiva .Por último tenemos \textbf{armónicos}, estos pueden consumen energía y consumen espectros de otras bandas que no se nos fueron asignadas, debemos aplicar filtros para controlarlos. Esto se llama \textbf{distorsión armónica}.
\begin{figure}[H]
\centering
\includegraphics[width=0.6\linewidth]{mrf/mrf10.png}
\caption{Ganancia y potencia de salida en un proceso de saturación.}
\end{figure}
Nota que se da un margen de 1 dB para poder calcular la potencia de saturación. Esta potencia a 1dB se llama \textbf{punto de compresión de 1dB}: Potenica a la salida en que la ganancia se ha reducido un decibelio.\\
Como lo visto, cuando se inyecta un tono sinusoidal puro en el modelo no lineal, la salida contiene varias componentes de diferentes frecuencias que toman las amplitudes siguientes:
Entrada:
\begin{displaymath}
v_1=A\cos(\omega_0t)
\end{displaymath}
Salida:
\begin{align*}
v_2&=k_1A\cos(\omega_0t)+k_2A^2\cos^2(\omega_0t)+k_3A^3\cos^2(\omega_0t)+\cdots\\
&=\frac{k_2A^2}{2}+\left(k_1+\frac{3k_3A^2}{4}\right)A\cos(\omega_0t)+\cdots
\end{align*}
Donde $P=\frac{(k_1A)^2}{2}$ es la potencia de salida en régimen lineal. El punto de compresión de 1 dB se obtiene para una ganancia de tensión que se ha reducido en 1 dB: $g_v=0.981k_1$. En ese punto, la potencia de salida viene dada por $P_{1dB}=0.794P$. Según el modelo polinómico se cumple entonces:
\begin{displaymath}
P_{1dB}=-0.058\frac{k_1^3}{K_3}
\end{displaymath}
En amplificadores muy lineales la relación es pequeña (0.3dB), mientras que en amplificadores poco lineales puede ser grande (2dB), siendo típicamente del orden de 1dB.\\
En el estudio de procesos no lineales en amplificadoes de potencia, es frecuente referir la potencia de salida a la de saturación. Puede entonces definirse un \textit{potencia de salida relativa a saturación} (OPBO):
\begin{equation}
OPBO (dB)=P_{sat}(dBm)- P_0(dBm)
\label{eq:opbo}
\end{equation}
\begin{example}[Un amplificador de potencia trabaja con una señal de RF cuya relación de potencia media a potencia pico es de 0.2. Si el nivel de saturación está 2dB por encima del punto de compresión de 1dB que es de 50dBm:]
\textbf{Determine la potencia máxima de salida para que nunca se sature:}\\
Si se asocia la máxima potencia de salida al punto de compresión de 1dB, la potencia media será 7dB menor que la potencia de pico (factor 0.2).
\begin{align*}
P_{med}&=P_{pico}\cdot 0.2\\
\frac{P_{pico}}{P_{med}}&=5\simeq 6.99 dB
\end{align*}
La potencia de pico a la salida corresponde al punto de compresión de 1 dB. En realidad hay que estimar 1dB más por efectos de la compresión de ganancia, de forma que la potencia media es:
\begin{displaymath}
P_{med}(dBm)=P_{pico}(dBm)-7=\underbrace{P_{1dB}+1}_{P_{pico}}-7=44dBm
\end{displaymath}
\textbf{Determine la OPBO}:\\
El nivel de salida realtivo a saturación se puede obtener como:
\begin{displaymath}
OPBO(dB)=\underbrace{P_{sat}(dBm)}_{P_{pico}+1dB}- P_{media}(dBm)=52dB-44dB=8dB
\end{displaymath}
\end{example}
\subsubsection{Armónicos}\index{Distorsión de armónicos}
La saturación provoca la aparación de armónicos de la señal de entrada, son cortes en las señales saturan la señal transportadora. Los armónicos se eliminan por filtrado. Se mide en porcentaje la tensión eficaz de armónico respecto a la tensión de salida o en dB de potencia.
\begin{definition}[Nivel de armónico de orden N]
Para calcular el nivel de un armónico de orden N ($P{AN}$) en un punto de potencia de salida ($P_0$) diferente del especificado, se puede hacer el siguiente cálculo:
\begin{equation}
P_{AN}=P_{AN0}\left(\frac{P_{out}}{P_{ref}}\right)^N
\label{eq:potencia del armonico}
\end{equation}
Donde:
\begin{itemize}
\item $P_{out}$: Potencia de salida. (W)
\item $P_{ref}$: Potenia de salida en un punto tomado como referencia para la medida del nivel del armónico. (W)
\item $P_{AN0}$: Potencia de referencia para un nivel de armónico. (W)
\end{itemize}
La ecuación \ref{eq:potencia del armonico} se puede expresar en dBm como:
\begin{equation}
P_{AN}(dBm)=P_{AN0}(dBm)+N(P_0(dBm)-P_{ref}(dBm))
\label{eq:potencia armonico dbm}
\end{equation}
\end{definition}
\begin{example}[Un amplificador de potencia lejos de la saturación genera un distorsión no lineal de forma que, al poner a la entrada una señal formada por un tono puro con una potencia de 10 dBm, a la salida produce una potencia n fundamental de 22 dBm y, medido en el analizador de espectros, se ha detectado un segundo armónico de -10 dBm y un tercer armónico de potencia igual a -18 dBm Determine La potencia de segundo y tercer armónico para una potencia de salida de 26 dBm]
La potencia de referenia en la medida de armónios es $P_{ref}=22 dBm$. De acuerdo con la ecuación \ref{eq:potencia armonico dbm}, hemos aumentado la potencia de salida en $P_0-P_{ref}=4dB$, lo que significa que el segundo armónico aumentará en 8 y el tercero en 12 dB:
\begin{align*}
P_{A2}(dBm)&=P_{A20}(dBm)+2(P_0(dBm)-P_{ref}(dBm))=-10+8=-2dBm\\
P_{A3}(dBm)&=P_{A30}(dBm)+3(P_0(dBm)-P_{ref}(dBm))=-18+12=-6dBm
\end{align*}
\includegraphics[width=0.8\linewidth]{mrf/mrf11.png}
\end{example}
\subsubsection{Intermodulación 3er orden: 2 tonos}
Ahora en vez de meter una sola señal (tono) a un amplificador no linean, vamos a meter dos tonos en diferente frecuencia:
\begin{figure}[H]
\centering
\includegraphics[width=0.8\linewidth]{mrf/mrf12.png}
\caption{Amplificador no lineal a 2 tonos.}
\end{figure}
Igualmente debemos ``bajar'' los exponentes a orden uno, aunque los términos aumentarán. La salida en frecuencia se vería así:
\begin{figure}[H]
\centering
\includegraphics[width=0.8\linewidth]{mrf/mrf13.png}
\caption{Intermodulación de tercer orden con dos tonos.}
\end{figure}
Nota que las combinaciones de frecuencias están a lo largo del dominio de frecuencia debido a la función de transferencia no lineal. A pesar de dos tonos, la amplificación esta generando 3 tonos.
\begin{figure}[H]
\centering
\includegraphics[width=0.8\linewidth]{mrf/mrf14.png}
\caption{Intermodulación de tercer orden con dos tonos: potencia.}
\end{figure}
La potencia asociada a los productos de intermodulación se define como:
\begin{equation}
I_3=CP_0^3
\end{equation}
Para potencia total de las señales de entrada ($P_{out}$) ubicado en el punto de intersección de tercer orden ($P_{out}=P\cdot I_3$):
\begin{align*}
&P_{out}=I_3=PI_3 &C=\frac{1}{P\cdot I_3^2}
\end{align*}
En función del punto de interseción ($PI_3$) y de la potencia de salida puede obtenerse la potencia de las componentes de intermodulación o su relación con la potencia de salida como:
\begin{equation}
I_3=P_{out}\left(\frac{P_{out}}{PI_3}\right)^2 \Rightarrow \frac{P_{out}}{I_3}=\left(\frac{PI_3}{P_{out}}\right)^2
\end{equation}
Donde $P_0$ es la potencia de salida en señal (las dos componentes), $PI_3$ es la potencia de slida en el punto de cruce e $I_3$ es la potencia total en los productos de intermodulación (las dos componentes).\\
Es frecuente trabajar con las potencia en unidades logarítmicas (dBm) o con las relaciones de potencias en dB, lo que aplicado a las expresiones anteriores queda como:
\begin{align}
I_3(dBm)&=3\cdot P_{out}(dBm)-2\cdot PI_3(dBm)\\
\left(\frac{P_0}{I_3}\right)(dB)&=2\cdot PI_3(dBm)-2\cdot P_{out}(dBm)
\end{align}
%ejercico diapo 29-distorsion
\section{Circuitos resonantes}
Un circuito resonante es una combinación de elementos sensibles a la frecuencia, conectados para obtener una respuesta selectora de frecuencia.
\subsection*{Circuito resonante en paralelo}
\begin{figure}[H]
\centering
\includegraphics[width=0.5\linewidth]{mrf/mrf16.png}
\caption{Circuito resonante en paralelo.}
\end{figure}
\begin{equation}
Y(\iu\omega)=G+\iu\omega C+\frac{1}{\iu\omega L}
\end{equation}
En paralelo, al buscar una igualdad entre la reactancia capacitiva y la inductiva, estos no entrarán en corto circuito, sino entrarán en circuito abierto. De este modo, la admitancia será tan solo la conductancia:
\begin{displaymath}
Y(\iu\omega_0)=G
\end{displaymath}
\begin{notation}
En serie, la impedancia en los elementos L y C tiende a 0, mientras que en paralelo tiende a infinito cuando el circuito esta en resonancia (funcionando con frecuencia de resonancia).
\end{notation}
Cuando se logra una impedancia o admitancia puramente real es cuando se trabaja en la \textbf{frecuencia de resonancia}, cuando se logra eso se dice que el circuito mismo está en \textbf{resonancia}. En la resonancia, el voltaje y la corriente están en \textbf{fase} y, por consiguiente, el \textbf{ángulo de fase es cero} y el \textbf{factor de potencia es unitario}.\\
En el caso en \textbf{serie}, en la resonancia, la \textbf{impedancia} es un \textbf{mínimo} y, por consiguiente, la \textbf{corriente es máxima} para un voltaje dado.
\begin{notation}
A bajas frecuencias, la impedancia del circuito en serie está dominado por el término capacitivo y la admitancia del circuito en paralelo está dominada por el término inductivo.\\
A altas frecuencias, la impedancia del circuito en serie está dominado por el término inductivo y la admitancia del circuito en paralelo está dominada por el término capacitivo.
\end{notation}
\begin{figure}[H]
\centering
\includegraphics[width=0.5\linewidth]{mrf/mrf17.png}
\caption{Gráficas del comportamiento en la frecuecnia de resonancia para serie y paralelo.}
\end{figure}
\begin{definition}[Función de transferencia en paralelo]
Realizando una asociación de admitancias en paralelo para hallar la función de transferencia sobre una fuente de corriente se obtiene:
\begin{equation}
H=\frac{1}{1+\iu\left(\omega\cdot C-\frac{1}{\omega\cdot L}\right)R}
\label{eq:funcion transferencia paralelo}
\end{equation}
\begin{definition}[Frecuencia de resonancia]
\begin{equation}
f_r=\frac{1}{2\pi\sqrt{L\cdot C}}
\label{eq:frec resonancia}
\end{equation}
Donde:
\begin{itemize}
\item $f_r$: Frecuencia de resonancia. (Hz)
\item \textbf{L}: Inductancia. (H)
\item \textbf{C}: Capacitancia. (F)
\end{itemize}
\end{definition}
\end{definition}
\begin{definition}[Factor de calidad-paralelo]
En esencia \textit{Q} es una medida de la capacidad de almacenamiento de energía de un circuito en relación con su capacidad de disipación de energía.
\begin{equation}
Q=\omega_r\cdot C\cdot R=\frac{R}{\omega_r\cdot L}=R\sqrt{\frac{C}{L}}
\label{eq:factor calidad paralelo}
\end{equation}
Donde:
\begin{itemize}
\item $\omega_r$: Frecuencia de resonancia. (rad/s)
\item \textbf{C}: Capacitancia. (F)
\item \textbf{R}: Resistancia. ($\Omega$)
\item \textbf{L}: Inductancia. (H)
\end{itemize}
\end{definition}
\begin{remark}
Cuando esta en resonancia, la función de transferencia debe ser igual a la unidad.
\end{remark}
Con el factor de calidad se puede llegar a una expresión para la función de transferencia expresada en la ecuación \ref{eq:funcion transferencia paralelo} :
\begin{equation}
H=\frac{1}{1+\iu Q\left(\frac{\omega}{\omega_0}-\frac{\omega_0}{\omega}\right)}
\end{equation}
Donde su magnitud y fase están dadas por:
\begin{align}
|H|&=\frac{1}{\left[+Q^2\left(\frac{\omega}{\omega_0}-\frac{\omega_0}{\omega}\right)^2\right]^{\frac{1}{2}}}\\
\phi&=-\tan^-1Q\left(\frac{\omega}{\omega_0}-\frac{\omega_0}{\omega}\right)
\end{align}
A ciertas frecuencias se obtiene los siguientes valores:
\begin{table}[H]
\centering
\begin{tabular}{|c|c|c|c|c|c|}
\hline
$\omega$ & 0   & $\omega_1$           & $\omega_0$ & $\omega_2$            & $\infty$ \\ \hline
|H|      & 0   & $\frac{1}{\sqrt{2}}$ & 1          & $-\frac{1}{\sqrt{2}}$ & 0        \\ \hline
$\phi$   & 90° & 45°                  & 0°         & -45°                  & -90°     \\ \hline
\end{tabular}
\caption{Valores importantes para el módulo y fase de H.}
\end{table}
Notamos que tenemos dos frecuecias: $\omega_1$ y $\omega_2$, que corresponden a $H=\frac{1}{\sqrt{2}}$. Examinando la ecuación de la magnitud H, se observa que $H=\frac{1}{\sqrt{2}}$ se presenta cuando:
\begin{equation}
Q^2\left(\frac{\omega}{\omega_r}-\frac{\omega_r}{\omega}\right)^2=1
\label{eq:h resonancia}
\end{equation}
La ecuación \ref{eq:h resonancia} se puede reordenar en forma de una ecuación de cuarto grado, en función de $\omega$. Despejando los valores de interés, se obtiene:
\begin{definition}[Frecuencias de media potencia]
Estos parámetros indican desde que frecuencias hasta que frecuencias opera nuestro circuito, en otras palabras limita el ancho de banda:
\begin{align}
\omega_1&=\omega_0\sqrt{1+\parentesis{\frac{1}{2Q}}^2}-\frac{\omega_r}{2Q}\\
\omega_2&=\omega_0\sqrt{1+\parentesis{\frac{1}{2Q}}^2}+\frac{\omega_r}{2Q}
\label{eq:limites de banda}
\end{align}
\end{definition}

\begin{definition}[Ancho de banda]
El ancho de banda (WB) del circuito se define como el intervalo que se encuentra entre las dos frecuencias donde la magnitud de la ganancia es $\frac{1}{\sqrt{2}}$ (-3dB). El ancho de banda de un circuito selectivo en frecuencia es el intervalo entre las frecuencias donde la magnitud de la ganancia cae a $\frac{1}{\sqrt{2}}$ veces el valor máximo.
\begin{equation}
WB=\omega_2-\omega_1=\frac{\omega_r}{Q}
\label{eq:ancho de banda resonancia}
\end{equation}
Donde:
\begin{itemize}
\item \textbf{WB}: Ancho de banda. (rad/s)
\item $\omega_2$: Frecuencia superior en -3dB. (rad/s)
\item $\omega_1$: Frecuencia inferior en -3dB. (rad/s)
\item $\omega_r$: Frecuencia de resonancia. (rad/s)
\item \textbf{\textit{Q}}: Factor de calidad.
\end{itemize}
\end{definition}
La ecuación \ref{eq:ancho de banda resonancia} ilustra que el ancho de banda es inversamente proporcional a \textit{Q}. Por tanto, la selectividad de frecuencia del circuito esta determinada por el valor de \textit{Q}. Un circuito de \textit{Q} alta tiene un ancho de banda pequeño y, por consiguiente, el circuito es muy selectivo.
\begin{figure}[H]
\centering
\includegraphics[width=0.5\linewidth]{mrf/mrf18.png}
\caption{Cuanto más alta \textit{Q}, tanto más pequeño el ancho de banda.}
\end{figure}
Cuando Q>10, las expresiones \ref{eq:limites de banda} se reducen a:
\begin{align}
\omega_1&\cong\omega_r-\frac{WB}{2}\\
\omega_2&\cong\omega_r+\frac{WB}{2}
\end{align}
Cuando Q>10, la curva de magnitud tiene una simetría aritmética aproximada entorno $\omega_r$. Independientemente de Q, la respuesta es simétrica en una escala logarítmica de frecuencia.
\begin{remark}
Si la banda de frecuencia que se va a seleccionar o a rechazar es estrecha, el \textit{Q} del circuito resonante debe ser alto. Si la banda de frecuencias es amplia, el \textit{Q} debe ser bajo.
\end{remark}
Para pequeñas desviaciones de frecuencias con respecto a $\omega_0$, \textit{Q} es relativamente alta y definimos $\delta$ como:
\begin{displaymath}
\delta=\frac{\omega-\omega_r}{\omega_r}=\frac{\omega}{\omega_r}-1
\end{displaymath}
donde $\delta$ representa la cantidad propocional por la que la frecuencia se desvía de $\omega_r$, entonces podemos escribir el siguiente factor en términod de $\delta$:
\begin{displaymath}
\frac{\omega}{\omega_r}-\frac{\omega_r}{\omega}=\parentesis{\delta+1}+\parentesis{\frac{1}{\delta+1}}=\frac{\parentesis{\delta+1}^2-1}{\delta+1}=\frac{\delta^2+2\delta}{\delta+1}
\end{displaymath}
Usando $\delta\ll 1$ para pequeñas desviaciones de $\omega_r$:
\begin{displaymath}
\frac{\omega}{\omega_r}-\frac{\omega_r}{\omega}\cong 2\delta
\end{displaymath}
Entonces H es:
\begin{equation}
H=\frac{1}{1+\iu 2Q\delta}
\end{equation}
Que es una aproximación valida siempre que $\delta\ll 1$.
\subsection*{Circuito resonante en serie}
\begin{figure}[H]
\centering
\includegraphics[width=0.5\linewidth]{mrf/mrf15.png}
\caption{Circuito resonante en serie.}
\end{figure}
Para un circuito RLC en serie, la \textbf{impedancia de entrada} se define:
\begin{equation}
Z(\iu\omega)=R+\iu\omega L+\frac{1}{\iu\omega C}
\end{equation}
Vemos que las reactancias inductivas y capacitivas se mueven, independientemente de sus valores L y C, con la frecuencia $\omega$, lo que se busca es que las reactancias se anulen para que nos quede la impedancia como un elemento netamente resistivo:
\begin{displaymath}
Z(\iu\omega_0)=R
\end{displaymath}
\begin{notation}
Que las reactancias se anulen significa que esos elementos funcionarán como un corto circuito.
\end{notation}
\begin{definition}[Función de transferencia en serie]
La función de transferencia para un circuito en serie esta dada por:
\begin{equation}
H=\frac{1}{1+\iu\parentesis{\frac{\omega\cdot L}{R}-\frac{1}{\omega CR}}}
\label{eq:funcion transferencia serie}
\end{equation}
\end{definition}
\begin{definition}[Factor de calidad-serie]
\begin{equation}
Q=\frac{\omega_r L}{R}=\frac{1}{\omega_rCR}=\frac{1}{R}\sqrt{\frac{L}{C}}
\label{eq:factor calidad serie}
\end{equation}
Donde:
\begin{itemize}
\item $\omega_r$: Frecuencia de resonancia. (rad/s)
\item \textbf{C}: Capacitancia. (F)
\item \textbf{R}: Resistancia. ($\Omega$)
\item \textbf{L}: Inductancia. (H)
\end{itemize}
\end{definition}
Con el factor de calidad se puede definir la ecuación \ref{eq:funcion transferencia serie} como:
\begin{equation}
H=\frac{1}{1+\iu Q\parentesis{\frac{\omega}{\omega_0}-\frac{\omega_0}{\omega}}}
\end{equation}
\begin{figure}[H]
\centering
\includegraphics[width=0.8\linewidth]{mrf/mrf19.png}
\caption{Resumen de osciladores: $\omega_0$ es la frecuencia de resonancia $\omega_r$}
\end{figure}
\section{Osciladores}
\subsection{Sistemas retroalimentados}
\begin{figure}[H]
\centering
\includegraphics[width=0.8\linewidth]{mrf/mrf20.png}
\caption{Sistemas retroalimentados.}
\label{fig:sis retro}
\end{figure}
De esta figura se desprenden dos ecuaciones que caracterizan a la función de transferencia:\\
\textbf{Lazo abierto: ganancia}
\begin{equation}
G(s)=\frac{x_s(s)}{x_{er}(s)}
\end{equation}
\textbf{Lazo cerrado: ganancia retroalimentada}:
\begin{equation}
\frac{x_s(s)}{x_e(s)}=\frac{G(s)}{1+G(s)\cdot H(s)}
\label{eq:fun tran retro}
\end{equation}
\begin{notation}
Si la retroalimentación es negativa, en la ecuación de lazo cerrado es positiva; igualmente si es positiva la retroalimentación, la ecuación es negativa.
\end{notation}
Algunos casos particulares que tenemos son:
\begin{itemize}
\item Realimentación negativa:
\begin{displaymath}
|1+G(s)\cdot H(s)|>1
\end{displaymath}
\item Alta ganancia de lazo: Cuando la ganancia es muy grande
\begin{displaymath}
\frac{x_s(s)}{x_e(s)}=\frac{1}{H(s)}
\end{displaymath}
\item Realimentación positiva:
\begin{displaymath}
|1+G(s)\cdot H(s)|<1
\end{displaymath}
\item Oscilación
\begin{displaymath}
|1+G(s)\cdot H(s)|=0
\end{displaymath}
La \textbf{oscilación} es indeseada en servosistemas y deseada en osciladores.
\end{itemize}
\subsection*{Caso de oscilación}
Si tenemos en cuenta la condición de oscilación, entonces se tendrá $x_s(s)/x_e(s)\rightarrow\infty$. Por lo tanto se genera una señal de salida ($x_s$) aunque no haya entrada ($x_e$)
\begin{figure}[H]
\centering
\includegraphics[width=0.6\linewidth]{mrf/mrf21.png}
\caption{Inversión de la red de retroalimentación}
\end{figure}
Cuando se esta oscilando, se tiene:
\begin{displaymath}
|G(s)\cdot H(s)|=1
\end{displaymath}
\begin{displaymath}
\fase{ }{G(s)\cdot H(s)}=180^{\circ}
\end{displaymath}
En oscilación:
\begin{displaymath}
|G(\iu\omega_r)\cdot H(\iu\omega_r)|=1
\end{displaymath}
\begin{displaymath}
\fase{ }{G(\iu\omega_r)\cdot H(\iu\omega_r)}=180°
\end{displaymath}
Graficando la oscilación en la figura \ref{fig:oscilacion}, nota como la señal sale en fase de la planta, en la salida llega la onda amplificada mientras que en la red de retroalimentación se atenúa  la señal un poco pero esta totalmente desfasada. Por mejor del bloque inversor esta se acoplada nuevamente a la onda. Con esto nos aseguramos que la oscilación ha comenzado.
\begin{wrapfigure}{r}{0.5\linewidth}%tamaño del campo asignado
  \begin{center}
    \includegraphics[width=.9\linewidth]{mrf/mrf22.png}%tamaño del la imgen dentro del campo
  \end{center}
  \caption{Condición de oscilación.}
  \label{fig:oscilacion}
\end{wrapfigure}
\subsubsection{Condición de oscilación}\index{Condición de oscilación}
\textbf{¿Qué tiene que suceder para que comience la oscilación?}\\
Los circuitos osciladores son un tanto especiales, necesitamos para arrancar una señal de entrada que sirva como una señal de arranque ($x_e$) (Fig. \ref{fig:condi osci}).
\begin{figure}[]
\centering
\includegraphics[width=0.6\linewidth]{mrf/mrf23.png}
\caption{Condición de oscilación}
\label{fig:condi osci}
\end{figure}
Cuando el sistema termine de arrancar, cerramos el switch y podemos quitar la señal de entrada. Hay que asegurarnos que $|G\parentesis{\iu\omega_{osc}}\cdot H\parentesis{\iu\omega_{osc}}|>1$, cuando el desfase es 180$^{\circ}$, entonces podemos hacer que la salida del
lazo de realimentación haga las funciones de la magnitud de entrada.
\begin{wrapfigure}{r}{0.55\linewidth}
  \begin{center}
    \includegraphics[width=0.9\linewidth]{mrf/mrf24.png}
  \end{center}
  \caption{Oscilación.}
  \label{fig:oscilacion 1}
\end{wrapfigure}
En la figura \ref{fig:oscilacion 1} el sistema ya esta arrancando, pero no termina ahí, ahora debe calibrarse para que pueda auto-arrancar por si solo.\\\textbf{Momento de oscilación}
\begin{figure}[H]
\centering
\includegraphics[width=0.6\linewidth]{mrf/mrf25.png}
\caption{Transición a estable.}
\end{figure}
\begin{notation}
$G_{pm}(s)$ significa función de transferencia de pequeña magnitud. $G_{gm}(s)$: función de transferencia de gran magnitud.
\end{notation}
\subsubsection*{Criterio de Nyquist}
Para que \textbf{empiece} la oscilación:
\begin{itemize}
\item Tiene que existir una $\omega_{osc}$ a la que se cumpla:
\begin{displaymath}
\fase{}{G(\iu\omega_{osc})\cdot H(\iu\omega_{osc})}=180^{\circ}
\end{displaymath}
\item A esa $\omega_{osc}$ tiene que cumplirse:
\begin{displaymath}
|G(\iu\omega_{osc})\cdot H(\iu\omega_{osc})|>1
\end{displaymath}
\end{itemize}
Cuando se \textbf{estabiliza} la oscilación:
\begin{itemize}
\item Disminuye la ganancia
\begin{displaymath}
G(\iu\omega_{osc})
\end{displaymath}
hasta que
\begin{displaymath}
|G(\iu\omega_{osc})\cdot H(\iu\omega_{osc})|=1
\end{displaymath}
cuando
\begin{displaymath}
\fase{}{G(\iu\omega_{osc})\cdot H(\iu\omega_{osc})}=180^{\circ}
\end{displaymath}
\end{itemize}
\subsubsection*{Diagramas de Bode}
Observa la linea punteada, de la figura \ref{fig:cuando oscila}, debe ser corregida pues tanto la fase como la magnitud deben coincidir en la misma frecuencia a 0dB (o $|G(\iu\omega_{osc})\cdot H(\iu\omega_{osc})|$)
\begin{figure}[]
\centering
\subfloat[Cuando ya oscila.]{\includegraphics[width=0.25\linewidth]{mrf/mrf26.png}\label{fig:cuando oscila}}
\subfloat[Para que no oscile.]{\includegraphics[width=0.25\linewidth]{mrf/mrf27.png}}
\caption{Condición de oscilación}
\end{figure}
\begin{figure}[H]
\centering
\includegraphics[width=0.7\linewidth]{mrf/mrf28.png}
\caption{Resumen de arranque de osciladores.}
\end{figure}
%Osciladores hartley
\subsection{Osciladores de frecuencia muy constante}
\begin{wrapfigure}{r}{0.5\linewidth}
  \begin{center}
    \includegraphics[width=0.9\linewidth]{mrf/mrf29.png}
  \end{center}
  \caption{Símbolo cristal oscilador}
\end{wrapfigure}
\subsubsection*{Cristales osciladores}
Es un cristal de cuarzo (u otro material piezoeléctrico) que consta de una placa de cuarzo en medio de dos contactos metálicos, encapsulados. Se aplica voltaje en sus terminales para que vibren de manera tan exacta que es muy precisa.
\begin{wrapfigure}{l}{0.5\linewidth}
  \begin{center}
    \includegraphics[width=.9\linewidth]{mrf/mrf30.png}
  \end{center}
  \caption{Cristal oscilador}
\end{wrapfigure}
\begin{figure}[]
\centering
\includegraphics[width=0.6\linewidth]{mrf/mrf31.png}
\caption{Circuito equivalente de un cristal de cuarzo.}
\end{figure}
\chapterimage{chapter_head_MR.pdf} % Chapter heading image
\chapter{Circuitos de enganche}
\section{Lazos enganchados por fase-PLL}
\begin{wrapfigure}{r}{0.5\linewidth}
  \begin{center}
    \includegraphics[width=0.5\linewidth]{mrf/mrf32.png}
  \end{center}
  \caption{Retroalimentación por corto circuito.}
  \label{fig:retro pll}
\end{wrapfigure}
Recordando el sistema de la figura \ref{fig:sis retro}, y su respectiva ecuación \ref{eq:fun tran retro}; hacemos notar que el bloque \textit{H(s)} puede tener distintos comportamientos como arreglos de capacitores, resistencias e inductores, inclusive podemos tener una resistencia al infinito o resistencia que tiende a cero. La última es la que nos interesa; que pasa si nuestro bloque \textit{H(s)} tiende a cero, nos quedaría un cortocircuito como la figura \ref{fig:retro pll}.\\
\begin{equation}
\frac{x_s(s)}{x_e(s)}=\frac{G(s)}{1+G(s)}
\label{eq:function transferencia pll}
\end{equation}
Con el arreglo de la figura \ref{fig:retro pll}, notamos que: $H(s)=1$, y si la ganancia es bastante grande ($G(s)>>1$), por lo tanto:
\begin{displaymath}
\frac{x_s(s)}{x_e(s)}=1\Rightarrow x_s(s)=x_e(s)
\end{displaymath}
El bloque $G(s)$, internamente vamos a analizar como la siguiente figura:
\begin{figure}[H]
\centering
\includegraphics[width=0.7\linewidth]{mrf/mrf33.png}
\caption{Estructura básica de un PLL.}
\end{figure}
\begin{notation}
Las unidades en estos sistemas no solo es voltios o amperios, también pueden ser fases.
\end{notation}
\begin{remark}
Como lo que se comparan son las \textbf{fases} de las señales de salida y entrada, además como la ganancia de la red de realimentación es 1, el sistema tenderá a anular la diferencia de fases entre estas señales. Los niveles de tensión de ambas no serán similares.
\end{remark}
\begin{figure}[H]
\centering
\subfloat[Diagrama de bloques]{\includegraphics[width=0.45\linewidth]{mrf/mrf34.png}}
\subfloat[Señal en el dominio de tiempo]{\includegraphics[width=0.45\linewidth]{mrf/mrf35.png}}
\caption{Nombre}
\end{figure}
Volviendo al diagrama de bloques inicial, tendremos lo siguiente:
\begin{figure}[H]
\centering
\includegraphics[width=0.8\linewidth]{mrf/mrf36.png}
\caption{Diagrama de bloques PLL.}
\end{figure}
Hay que localizar un punto de equilibrio para linealizar el
funcionamiento del sistema. La clave está en el VCO.
\begin{figure}[H]
\centering
\includegraphics[width=.7\linewidth]{mrf/mrf37.png}
\caption{VCO: Para $K_v$>0.}
\end{figure}
Por lo tanto, podemos escribir la frecuencia es oscilación como:
\begin{align}
f_{osc}&=f_{osc0}+K_v\cdot V_c\\
\omega_{osc}&=\omega_{osc0}+2\pi\cdot K_v\cdot V_c
\end{align}
La frecuencia de oscilación será igual a la frecuencia de oscilación en estado inicial más el producto del la constante del varicap por el voltaje de entrada.\\
Para relacionar la frecuencia con la fase absoluta ($\Phi$):
\begin{align}
\omega_{osc}&=\omega_{osc0}+2\pi\cdot K_v\cdot V_c\\
\Phi_{osc}&=\omega_{osc0}\cdot t+2\pi\cdot K_v\cdot \int_0^tV_c\cdot t
\intertext{Referimos la fase absoluta $\Phi_{osc}$ a la frecuencia $\omega_{osc0}$:}
\Phi_{osc}&=\omega_{osc0}\cdot t+\underbrace{\textcolor{red}{\phi_{osc}(V_c)}}_{relativa:\phi_e}\label{eq:phiosc}\\
\intertext{Referimos nuevamente a $\omega_{osc0}$ la fase absoluta $\Phi_e$:}
\Phi_e&=\omega_{osc0}\cdot t+\phi_e\label{eq:phie}
\end{align}
\begin{figure}[H]
\centering
\includegraphics[width=0.8\linewidth]{mrf/mrf38.png}
\caption{Diagrama de bloques relativo a $\omega_{osc0}$}
\end{figure}
Las ecuaciones:\\
\begin{align}
\intertext{VCO:}
\phi_{osc}(V_c)&=2\pi\cdot K_v\cdot \int_0^TV_c\cdot t\\
\intertext{Filtro pasa-bajos:}
V_c&=F(V_{\Delta\Phi})\\
\intertext{Convertidor $\Phi/V$:}
V_{\Delta\Phi}=K_{\Delta\Phi}\cdot\parentesis{\Phi_e-\Phi_{osc}}=K_{\Delta\Phi}\cdot\parentesis{\phi_e-\phi_{osc}}
\end{align}
\begin{notation}
Nota que aún no hemos definido la función de transferencia del filtro, arbitrariamente la asumimos. Más adelante se hablará sobre ello.
\end{notation}
Tomando las tranformadas de Laplace para las ecuaciones y hallando su función de transferencia:
\begin{align}
\intertext{VCO:}
\frac{\phi_{osc}(s)}{V_c(s)}&=2\pi\cdot \frac{K_v}{s}\\\intertext{Filtro pasa-bajos:}
\frac{V_c(s)}{V_{\Delta\Phi}(s)}&=F(s)
\intertext{Convertidor $\Phi/V$}
\frac{V_{\Delta\Phi}(s)}{\Delta\phi(s)}&=K_{\Delta\Phi}\\
\intertext{Restador de fases:}
\Delta\phi(s)&=\phi_e(s)-\phi_{osc}(s)
\end{align}
\begin{figure}[H]
\centering
\includegraphics[width=0.8\linewidth]{mrf/mrf38.png}
\caption{Diagrama de bloques en Laplace.}
\label{fig:bloques laplace}
\end{figure}
Usando como referencia la ecuación \ref{eq:fun tran retro}, reescribimos el sistema de la figura \ref{fig:bloques laplace} como su función de transferencia lazo cerrado:
\begin{equation}
T_{\phi osc-\phi e}(s)=\frac{\phi_{osc}(s)}{\phi_e(s)}=\frac{2\pi\cdot K_v\cdot K_{\Delta\Phi}\cdot\frac{F(s)}{s}}{1+2\pi\cdot K_v\cdot K_{\Delta\Phi}\cdot\frac{F(s)}{s}}=\frac{2\pi\cdot K_v\cdot K_{\Delta\Phi}\cdot F(s)}{s+2\pi\cdot K_v\cdot K_{\Delta\Phi}\cdot F(s)}
\label{eq:lazo cerrado laplace}
\end{equation}
Tenemos dos funciones de transferencia más: Salida del filtro y lazo abierto, respectivamente son:
\begin{equation}
T_{\Delta\phi osc-\phi e}(s)=\frac{\Delta\phi(s)}{\phi_e(s)}=1-T_{\phi osc-\phi e}(s)=\frac{s}{s+2\pi\cdot K_v\cdot K_{\Delta\Phi}\cdot F(s)}
\label{eq:lazo cerrado interrumpido laplace}
\end{equation}
\begin{equation}
T_{\phi osc-\Delta\phi}(s)=\frac{\phi_{osc}(s)}{\Delta\phi(s)}=2\pi\cdot K_v\cdot K_{\Delta\Phi}\cdot \frac{F(s)}{s}
\label{eq:lazo abierto laplace}
\end{equation}
Ordenando los resultados, volvemos a dibujar el diagrama como:
\begin{figure}[H]
\centering
\includegraphics[width=0.7\linewidth]{mrf/mrf38.png}
\caption{Diagrama de bloques}
\end{figure}
Escribiendo la función de transferencia con la ecuación \ref{eq:lazo abierto laplace} junto con la ecuación \ref{eq:function transferencia pll}
\begin{equation}
T_{\phi osc-\phi e}(s)=\frac{T_{\phi osc-\Delta\phi}(s)}{1+T_{\phi osc-\Delta\phi}(s)}
\label{eq:func tran simplificada}
\end{equation}
\subsection{Orden y tipo}
Tomando la ecuación \ref{eq:lazo cerrado interrumpido laplace}, aplicamos un escalón:
\begin{displaymath}
\phi_e(s): \phi_e(s)=\frac{\phi_{e1}}{s}
\end{displaymath}
Entonces la función de transferencia puede ser escrita como:
\begin{displaymath}
\Delta\phi(s)=T_{\Delta\phi osc-\phi e}(s)\cdot\frac{\phi_{e1}}{s}\Rightarrow\frac{\phi_{e1}}{s+2\pi\cdot K_v\cdot K_{\Delta\Phi}\cdot F(s)}
\end{displaymath}
Por le teorema de valor final:
\begin{displaymath}
\lim_{t\to\infty}\Delta\phi(t)=\lim_{s\to 0}s\cdot\Delta\phi(s)=\frac{\phi_{e1}\cdot s}{s+2\pi\cdot K_v\cdot K_{\Delta\Phi}\cdot F(s)}
\end{displaymath}
Nota que \textit{F(s)} no puede tener un cero en cero, por ejemplo, F(s)=1/(1+RCs) vale como filtro:
\begin{figure}[H]
\centering
\includegraphics[width=0.5\linewidth]{mrf/mrf40.png}
\caption{Filtro RC}
\end{figure}
Con la F.T. del filtro y reemplazando en \ref{eq:lazo cerrado laplace} y dando valores nominales:
\begin{itemize}
\item $K_v=\basedec{5}$Hz/V
\item $RC=\basedec{-6}/\pi s$
\item $K_{\Delta\Phi}$=1-100 V/rad
\end{itemize}
Podemos graficar el diagrama de Bode correspondiente:
\begin{figure}[H]
\centering
\includegraphics[width=0.6\linewidth]{mrf/mrf41.png}
\caption{Diagrama de Bode de la F.T.}
\end{figure}
Volviendo con la ecuación \ref{eq:lazo abierto laplace} y \ref{eq:func tran simplificada} podemos identificar el orden y el tipo:
\begin{enumerate}
\item[Orden]: Número de polos de $T_{\phi Osc-\phi_e}$(eq. \ref{eq:func tran simplificada})
\item[Tipo]: Número de polos en s=0 de $T_{\phi Osc-\Delta\phi}$(eq. \ref{eq:lazo abierto laplace})
\end{enumerate}
\begin{example}[Orden y tipo de un PLL]
Con una red RC como filtro:
\begin{displaymath}
F(s)=\frac{1}{1+R\cdot C\cdot s}
\end{displaymath}
Procedemos a hallar el orden y tipo. Para ello primero en la ecuación \ref{eq:lazo cerrado laplace}:
\begin{displaymath}
T_{\phi osc-\phi e}(s)=\frac{2\pi\cdot K_v\cdot K_{\Delta\Phi}\cdot F(s)}{s+2\pi\cdot K_v\cdot K_{\Delta\Phi}\cdot F(s)}=\underbrace{\frac{2\pi\cdot K_v\cdot K_{\Delta\Phi}}{R\cdot C\cdot s^2+s+2\pi\cdot K_v\cdot K_{\Delta\Phi}}}_{Orden 2(2 polos)}
\end{displaymath}
Usando la ecuación de lazo abierto (eq. \ref{eq:lazo abierto laplace}):
\begin{displaymath}
T_{\phi osc-\Delta\phi}(s)=2\pi\cdot K_v\cdot K_{\Delta\Phi}\cdot \frac{F(s)}{s}=\underbrace{\frac{2\pi\cdot K_v\cdot K_{\Delta\Phi}}{s(1+R\cdot C\cdot s)}}_{Tipo 1(1 polo en s=0)}
\end{displaymath}
Como siempre la función de transferencia del integrador tiene un polo en cero, el tipo mínimo posible es 1.
\end{example}
\subsubsection{PLL de orden 1 y tipo 1}
La ecuación \ref{eq:lazo cerrado laplace} puede ser escrita como:
\begin{displaymath}
T_{\phi_{osc}-\phi_e}(s)=\frac{1}{\tau\cdot s+1}\Leftarrow \tau=\frac{1}{2\pi\cdot K_v\cdot K_{\Delta\Phi}\cdot F_1}
\end{displaymath}
Si le agregamos un escalón en frecuencia en la entrada $\omega_e(s)=\omega_{e1}/s$, por lo tanto:
\begin{displaymath}
\omega_{osc}(s)=\frac{\omega_{e1}}{s(\tau\cdot s+1)}
\end{displaymath}
\begin{figure}[]
\centering
\subfloat[Escalón de $\pi/2$]{\includegraphics[width=.8\linewidth]{mrf/mrf42.png}}\\
\subfloat[Escalón de 0.25$\omega_{osc0}$: No termina de converger]{\includegraphics[width=.8\linewidth]{mrf/mrf43.png}}
\caption{Respuestas al escalón.}
\end{figure}
\subsubsection{PLL orden 2 y tipo 1}
\begin{wrapfigure}{r}{0.5\linewidth}
  \begin{center}
    \includegraphics[width=0.45\linewidth]{mrf/mrf44.png}
  \end{center}
  \caption{Filtro orden 2.}
  \label{fig:filtro orden 2}
\end{wrapfigure}
Con el filtro de la figura \ref{fig:filtro orden 2}, se tiene:
\begin{displaymath}
F(s)=\frac{1+\frac{s}{\omega_z}}{1+\frac{s}{\omega_p}}
\end{displaymath}
Reemplazando los valores de los componentes en la función de transferencia del filtro:
\begin{displaymath}
F(s)=\frac{1+R_2\cdot C\cdot s}{1+(R_1+R_2)\cdot C\cdot s}
\end{displaymath}
Tiene un polo y cero, siendo:
\begin{align*}
\omega_z&=\frac{1}{R_2\cdot C}\\
\omega_p&=\frac{1}{(R_1+R_2)\cdot C}
\end{align*}
Hallando el tipo con la ecuación \ref{eq:lazo abierto laplace}:
\begin{displaymath}
T_{\phi osc-\Delta\phi}(s)=\underbrace{\frac{2\pi\cdot K_v\cdot K_{\Delta\Phi}(1+R_2\cdot C\cdot s)}{s\left[1+(R_1+R_2)\cdot C\cdot s\right]}}_{Tipo 1 (1 polo en s=0)}
\end{displaymath}
Usando la ecuación \ref{eq:func tran simplificada} y después de un proceso algebraico:
\begin{displaymath}
T_{\phi osc-\phi e}(s)=\underbrace{\frac{1+R_2\cdot C\cdot s}{\frac{(R_1+R_2)\cdot C}{2\pi\cdot K_v\cdot K_{\Delta\Phi}}\cdot s^2+\frac{1+2\pi\cdot K_v\cdot K_{\Delta\Phi}\cdot R_2\cdot C}{2\pi\cdot K_v\cdot K_{\Delta\Phi}}\cdot s+1}}_{Orden 2(2 polos)}
\end{displaymath}
Con: $\omega_z=\frac{1}{R_2\cdot C}$, $\omega_p=\frac{1}{(R_1+R_2)\cdot C}$, $K=2\pi\cdot K_v\cdot K_{\Delta\Phi}$ y escalón en la frecuencia de entrada: $\omega_e(s)=\omega_{e1}/s$:
\begin{displaymath}
\omega_{osc}(s)=\frac{(1+\frac{s}{\omega_z})\cdot\omega_{e1}}{s\parentesis{\frac{s^2}{\omega_p\cdot K}+s\parentesis{\frac{1+\frac{K}{\omega_z}}{K}}+1}}
\end{displaymath}
\begin{figure}[H]
\centering
\includegraphics[width=0.8\linewidth]{mrf/mrf45.png}
\caption{Respuesta a la frecuencia: Con $\omega\neq\infty$ existe más posibilidad de optimizar la respuesta dinámica}
\end{figure}
\subsubsection{PLL orden 2 y tipo 2}
\begin{wrapfigure}{r}{0.5\linewidth}
  \begin{center}
    \includegraphics[width=0.45\linewidth]{mrf/mrf46.png}
  \end{center}
  \caption{PLL activo orden 2 y tipo 2}
\end{wrapfigure}
\begin{displaymath}
F(s)=\omega_p\cdot\frac{1+\frac{s}{\omega_z}}{s}
\end{displaymath}
Reemplazando los elementos:
\begin{displaymath}
F(s)=\frac{1+(R_1+R_2)\cdot C\cdot s}{R_1\cdot C\cdot s}
\end{displaymath}
Tiene un polo en cero y un cero, siendo:
\begin{align*}
\omega_z&=\frac{1}{(R_1+R_2)\cdot C}\\
\omega_p&=\frac{1}{R_1\cdot C}
\end{align*}
Usando la ecuación \ref{eq:lazo abierto laplace}:
\begin{displaymath}
T_{\phi osc-\Delta\phi}(s)=\frac{2\pi\cdot K_v\cdot K_{\Delta\Phi}\cdot\left[1+(R_1+R_2)\cdot C\cdot s\right]}{\underbrace{s^2}_{Tipo 2 (2 polos en s=0)}\cdot R_1\cdot C}
\end{displaymath}
Reemplazando en la ecuación \ref{eq:lazo cerrado laplace} y después de un proceso algebraico:
\begin{displaymath}
T_{\phi osc-\phi e}(s)=\frac{1+(R_1+R_2)\cdot C\cdot s}{\underbrace{\frac{R_1\cdot C}{2\pi\cdot K_v\cdot K_{\Delta\Phi}}\cdot s^2+(R_1+R_2)\cdot C\cdot s+1}_{Orden 2(2 polos)}}
\end{displaymath}
Reagrupando los términos:
\begin{displaymath}
T_{\phi osc-\phi e}(s)=\frac{1+\frac{s}{\omega_z}}{\frac{s^2}{\omega_p\cdot K}+s\frac{1+\frac{K}{\omega_z}}{k}+1}
\end{displaymath}
EL resultado es semejante al obtenido en el PLL de Orden 2 y Tipo 1 anterior. Luego se puede optimizar de igual forma la respuesta dinámica. La ventaja es que al ser de Tipo 2 se anula la diferencia de fases en régimen permanente ante un escalón de frecuencia.\\
\begin{wrapfigure}{r}{0.5\linewidth}
  \begin{center}
    \includegraphics[width=0.5\linewidth]{mrf/mrf47.png}
  \end{center}
  \caption{Filtro activo}
  \label{fig:opam filtro}
\end{wrapfigure}
Si usamos el circuito de la figura \ref{fig:opam filtro}:
\begin{displaymath}
F(s)=-\frac{1+R_2\cdot C\cdot s}{R_1\cdot C\cdot s}\Rightarrow F(s)=-\omega_p\frac{1+\frac{s}{\omega_z}}{s}
\end{displaymath}
con: 
\begin{align*}
\omega_z&=\frac{1}{R_2\cdot C}\\
\omega_p&=\frac{1}{R_1\cdot C}
\end{align*}
hacemos el análisis como el circuito anterior, vamos a llegar a la siguiente expresión:
\begin{displaymath}
T_{\phi osc-\phi e}(s)=\frac{1+\frac{s}{\omega_z}}{\frac{s^2}{-\omega_p\cdot K}+\frac{s}{\omega_z}+1}
\end{displaymath}
Para que salga lo mismo que en el caso anterior, K tiene que ser negativa. Como $K=2\pi\cdot K_v\cdot K_{\Delta\Phi}$ o bien $K_v<0$ o $K_{\Delta\Phi}<0$. En caso contrario, el PLL sería inestable, al menos que el detector de fases cambie el signo de $K_{\Delta\Phi}$ en función de la diferencia de fases.
\section{Realización física}
\begin{figure}[H]
\centering
\includegraphics[width=0.8\linewidth]{MRF/mrf48.png}
\caption{PLL: Diagrama de bloques}
\end{figure}
\subsection{Detector de fases}
El detector de fases tiene dos entradas en las cuales \textbf{compara fases}. Existen dos tipos:
\begin{itemize}
\item Detector analógico
\begin{itemize}
\item Detector basado en un mezclador.
\end{itemize}
\item Detector digital:
\begin{itemize}
\item Detector basado en ``puerta exclusiva''.
\item Detector basado en ``biestable RS activado con flancos''.
\item Detector fase-frecuencia
\end{itemize}
\end{itemize}
\subsubsection{Mezclador}\index{PLL con mezclador}
\begin{figure}[H]
\centering
\includegraphics[width=0.7\linewidth]{MRF/mrf49.png}
\caption{Detector de fases basado en mezclador}
\end{figure}
Se tienen ambas entradas y se usa un mezclador (multiplicador) de ambas como un detector de fases, la \textbf{salida} puede ser escrita como:
\begin{equation}
V_{\Delta\Phi}=K_m\cdot V_e\sin\parentesis{\Phi_e}\cdot V_{osc}\sin\parentesis{\Phi_{osc}}
\label{eq:eq1}
\end{equation}
Usando la producto de senos en \ref{eq:eq1}:
\begin{equation}
V_{\Delta\Phi}=K_{\Delta\Phi}\left[\cos\parentesis{\Phi_e-\Phi_{osc}}-\cos\parentesis{\Phi_e+\Phi_{osc}}\right]
\label{eq:eq2}
\end{equation}
Donde:
\begin{displaymath}
K_{\Delta\Phi}=V_e\cdot V_{osc}\cdot \frac{K_m}{2}
\end{displaymath}
Recordando las ecuaciones \ref{eq:phiosc} y \ref{eq:phie} y reemplazando en \ref{eq:eq1}:
\begin{equation}
V_{\Delta\Phi}=K_{\Delta\Phi}\left[\cos\parentesis{\phi_e-\phi_{osc}}-\cos\parentesis{\phi_e+\phi_{osc}+2\cdot\omega_{osc0}\cdot t}\right]
\end{equation}
Podemos eliminar el segundo término del paréntesis por filtrado, quedando:
\begin{equation}
V_{\Delta\Phi}=K_{\Delta\Phi}\cdot\cos\parentesis{\phi_e-\phi_{osc}}=K_{\Delta\Phi}\cdot\sin\parentesis{\frac{\pi}{2}+\phi_e-\phi_{osc}}
\end{equation}
Para \textbf{valores pequeños}, el seno puede ser ignorado:
\begin{equation}
V_{\Delta\Phi}\approx K_{\Delta\Phi}\cdot\parentesis{\phi_e-\underbrace{\phi_{osc}+\frac{\pi}{2}}_{\phi'_{osc}}}
\end{equation}
Se comporta como se ha previsto, pero $\phi'_{osc}$ retrasada en 90° con relación al comportamiento teórico ($\phi_{osc}$)
\begin{figure}[H]
\centering
\includegraphics[width=0.7\linewidth]{MRF/mrf50.png}
\caption{Comparación y error: sin(x) y x.}
\end{figure}
\begin{corollary}[Aproximación seno]
Se comporta bastante lineal si:
\begin{displaymath}
|\phi_e-\phi'_{osc}|<60°
\end{displaymath}
es decir:
\begin{displaymath}
|90°+\phi_e-\phi_{osc}|<60°
\end{displaymath}
\end{corollary}
El límite sería:
\begin{equation}
|\phi_e-\phi'_{osc}|<90°
\end{equation}
Devolviendo el valor de $\phi'_{osc}$ y arreglando términos se llega a:
\begin{equation}
-180°<\parentesis{\phi_e-\phi_{osc}}<0°
\end{equation}
\begin{figure}[H]
\centering
\subfloat[Entrada]{\includegraphics[width=0.35\linewidth]{MRF/mrf51.png}}
\subfloat[Entrada]{\includegraphics[width=0.35\linewidth]{MRF/mrf52.png}}
\caption{Comparación y error.}
\end{figure}
\begin{remark}
En caso de que se superen estos límites, cambia el signo de $K_\Delta\Phi$, lo que genera problemas de estabilidad en $T_{\phi_o-\phi_e}(s)$. El lazo se desenganchará. 
\end{remark}
\textbf{Ventajas}:
\begin{itemize}
\item Trabaja con señales analógicas, por lo que puede operar hasta frecuencias muy altas (el límite depende de la tecnología del mezclador).
\item El filtro es del doble de la frecuencia de la señal generada.
\end{itemize}
\textbf{Desventajas}:
\begin{itemize}
\item El valor de la constante $K_\Delta\Phi$ es $K_\Delta\Phi = V_e\cdot V_{osc}·K_m/2$, es decir,
depende de la amplitud de las señales. A veces hay que limitarlas para acotar el valor de $K_\Delta\Phi$.
\item La diferencia de fases máxima posible es de $180^\circ$. En este caso:
$-180^\circ < (\phi_e – \phi_{osc}) < 0^\circ$. Tiene como \textbf{máximo} desde $-90^\circ hasta +90^\circ$ de margen de error.
\end{itemize}
\subsubsection{XOR}\index{PLL con XOR}
Ahora el detector de fases se basa en una compuerta lógica \textbf{XOR}:
\begin{figure}[H]
\centering
\includegraphics[width=0.7\linewidth]{mrf/mrf53.png}
\caption{PLL con XOR}
\end{figure}
Usando una tabla de verdad de la compuerta XOR, se puede representar el diagrama de tiempo a la salida:
\begin{figure}[H]
\centering
\includegraphics[width=0.7\linewidth]{mrf/mrf54.png}
\caption{Diagrama de tiempo XOR.}
\end{figure}
Ahora aplicaremos señales con distintos valores de desfase:
\begin{figure}[H]
\centering
\includegraphics[width=0.7\linewidth]{mrf/mrf55.png}
\caption{Diagrama de tiempo XOR con desfases.}
\label{fig:desfasexor}
\end{figure}
Si tomamos los \textcolor{blue}{promedios} de la figura \ref{fig:desfasexor} de obtiene la gráfica que se ha representado. El problema con esto, es que no tienen simetría respecto al 0. Para arreglar esto usaremos un offset en DC para hacerlo simétrico:
\begin{figure}[H]
\centering
\includegraphics[width=0.7\linewidth]{mrf/mrf56.png}
\caption{Diagrama de tiempo XOR con desfases.}
\label{fig:desfasexor}
\end{figure}
\textbf{Ventajas}:
\begin{itemize}
\item El circuito digital es relativamente sencillo, por lo que puede
operar hasta frecuencias bastante altas.
\item El valor de la constante $K_\Delta\Phi$ es $K_\Delta\Phi = V_{\Delta\Phi max}/\pi$, es decir, no
depende de la amplitud de las señales.
\item El filtro es del doble de la frecuencia de la señal generada.
\end{itemize}
\textbf{Desventajas}:
\begin{itemize}
\item La \textbf{diferencia de fases máxima} posible es de $180^\circ$. En este caso: $0<(\phi_e-\phi_{osc})<180$
\end{itemize}
\subsubsection{Biestale RS}\index{PLL con biestable RS}
Existe dos tipos: por flanco de subida o bajada. 
\begin{figure}[H]
\centering
\includegraphics[width=0.7\linewidth]{mrf/mrf57.png}
\caption{1 en B solo en el flanco de bajada/subida de A.}
\end{figure}
Si unimos en de esos en un biestable Set-Reset por blanco de bajada se puede construir un biestable pro flanco:
\begin{figure}[H]
\centering
\includegraphics[width=\linewidth]{mrf/mrf58.png}
\caption{Biestable activado por flanco descendente.}
\end{figure}
Aplicando un desfase entre la estrada y el retorno:
\begin{figure}[H]
\centering
\includegraphics[width=\linewidth]{mrf/mrf59.png}
\caption{Biestable activado por flanco descendente: desfase.}
\end{figure}
Notamos nuevamente que no es simétrico respecto a cero, si realizamos su \textcolor{blue}{promedio} tenemos un rango de 360, para modificar su simetría modificamos el nivel de tensión y retrasamos $\phi_e-\phi_{osc} \pi$ radianes
\begin{center}
\includegraphics[width=0.7\linewidth]{mrf/mrf60.png}
\end{center}
Ahora $\phi_{osc}'=\phi_{osc}+\pi$. Por tanto, el desarrollo teórico seguido es válido para , estando $f_{osc}'$ adelantada $180^\circ$ con relación a la fase realmente existente, que es $f_{osc}$.\\
Como vemos el límite seria: $-180^\circ<(\phi_e-\phi_{osc}')<180\circ$, es decir: $0^\circ<(\phi_e-\phi_{osc}<360^\circ)$\\
El valor de la constante $K_{\Delta\Phi}$ es $K_{\Delta\Phi}=V_{\Delta\Phi max}/2$\\
\textbf{Ventaja}:
\begin{itemize}
\item La diferencia de fases máxima posible es de $360^\circ$. En este caso:
$0^\circ < (f_e – f_{osc}) < 360^\circ$
\item El valor de la constante $K_{\Delta\Phi}$ es $K_{\Delta\Phi}=V_{\Delta\Phi max}/2$, es decir, no depende de la amplitud de las señales.
\end{itemize}
\textbf{Desventaja}:
\begin{itemize}
\item El filtro es de la frecuencia de la señal generada.
\item El circuito digital es relativamente complejo, por lo que no puede operar a frecuencias muy altas.
\end{itemize}
Ahora tenemos la siguiente idea:
\textit{Conseguir tener el equivalente a dos detectores basados en biestables activados por flancos: uno que funcione para diferencias de fases relativas de entre $0^\circ$ y $360^\circ$ y otro entre $–360^\circ$ y $0^\circ$.}
\begin{center}
\includegraphics[width=0.7\linewidth]{mrf/mrf60.png}
\end{center}
Con esto lograríamos un rango de 720 grados de enganche:
\begin{center}
\includegraphics[width=\linewidth]{mrf/mrf61.png}
\end{center}
Si analizamos los casos de desfase para este PLL:
\begin{figure}[H]
\centering
\includegraphics[width=\linewidth]{mrf/mrf62.png}
\caption{Detector biestable de 720 grados.}
\end{figure}
El circuito presentado antes tiene el siguiente esquema eléctrico:
\begin{figure}[H]
\centering
\includegraphics[width=0.7\linewidth]{mrf/mrf64.png}
\caption{Esquema electrónico de PLL de 720 grados.}
\end{figure}
\textbf{Ventajas:}
\begin{itemize}
\item La diferencia de fases máxima posible es de $720^\circ$. En este caso: $-360^\circ < (f_e – f_{osc}) < 360^\circ$
\item El valor de la constante $K_{\Delta\Phi}$ no depende de la amplitud de las señales.
\item Es el detector de fase con mejor enganche
\end{itemize}
\textbf{Desventajas:}
\begin{itemize}
\item El filtro es de la frecuencia de la señal generada.
\item El circuito digital es relativamente complejo, por lo que no puede operar a frecuencias muy altas.
\end{itemize}
\begin{vocabulary}[Diodo varicap]
El diodo Varicap conocido como diodo de capacidad variable o varactor, es un diodo que aprovecha determinadas técnicas constructivas para comportarse, ante variaciones de la tensión aplicada, como un condensador variable. Polarizado en inversa, este dispositivo electrónico presenta características que son de suma utilidad en circuitos sintonizados (L-C), donde son necesarios los cambios de capacidad.
\end{vocabulary}
\subsection{VCO}
Parámetros característicos de los PLL
\begin{itemize}
\item \textbf{Margen de mantenimiento estático (hold-in range)}: Es la diferencia de frecuencias de entrada entre las que el lazo permanece enganchado en las siguientes condiciones: partimos del lazo enganchado y cambiamos la frecuencia de entrada muy lentamente.
\item \textbf{Margen de mantenimiento dinámico (\textit{pull-out range})}: Es la diferencia de frecuencias de entrada entre las que el lazo permanece enganchado en las siguientes condiciones: partimos del lazo enganchado y cambiamos la frecuencia de entrada bruscamente (es, por tanto, el valor del escalón de frecuencia de entrada que acabamos de dar).
\item \textbf{Margen de enganche lineal (\textit{lock-in range})}: Es la diferencia de frecuencias de entrada entre las que el lazo se engancha trabajando el  detector de fases de forma lineal.
\item \textbf{Margen de enganche no lineal (\textit{pull-in range})}: Es la diferencia de frecuencias de entrada entre las que el lazo se engancha aunque el detector de fases llegue a trabajar de forma no lineal. 
\end{itemize}
\section{Sintetizadores}
Un sintetizador nace básicamente de un PLL con un bloque en el lazo de realimentación, más específico: un divisor. Además no hay señal de entrada, tan solo hay un oscilador o cristal resonante fija. 
\begin{figure}[H]
\centering
\includegraphics[width=\linewidth]{mrf/mrf65.png}
\caption{Sintetizador: diagrama de bloques.}
\end{figure}
Un sintetizador ``engaña'' a un PLL para poder transformar una frecuencia fija en otra frecuencia. En otras palabras, \textbf{transformar} una frecuencia fija de entrada en otra usando un contador  \textit{N}:
\begin{equation}
f_{vco}=f_{xtal}\cdot N
\end{equation}
Ten en cuenta que los saltos en los que cambiará la salida ($\Delta f$) serán del tamaño de la frecuencia del cristal $f_{xtal}$. El problema con esto es que los contadores programables tienen frecuencias máximas de uso no muy altas, para solucionar esto necesitamos combinar contadores fijos y programables.
\begin{figure}[H]
\centering
\includegraphics[width=0.7\linewidth]{mrf/mrf66.png}
\caption{Sintetizador con divisores fijos y programable.}
\end{figure}
Ahora la frecuencia de salida es:
\begin{equation}
f_{vco}=N_F\cdot N_P\cdot f_{xtal}
\end{equation}
Siendo ahora el cambio de los escalones:
\begin{equation}
\Delta f=N_F\cdot f_{xtal}
\end{equation}
El problema ahora es que la frecuencia del cristal cuando es muy pequeña, los cambios se realizan lentos. Para solucionar esto usamos sintetizadores de \textbf{doble módulo}.
\subsubsection{Sintetizador doble módulo}
\begin{figure}[H]
\centering
\includegraphics[width=0.8\linewidth]{mrf/mrf67.png}
\caption{Sintetizadores doble módulo}
\end{figure}
Para sintetizador doble módulo:
\begin{equation}
f_{vco}=N\cdot f_{xtal}
\end{equation}
Siendo:
\begin{equation}
N=N_p\cdot P+A
\end{equation}
El valor de \textit{N} estará dentro de un rango de valores:
\begin{displaymath}
N_{Pmax}\geq N_P\geq N_{Pmin}
\end{displaymath}
y
\begin{displaymath}
A_{max}\geq A\geq 1
\end{displaymath}
Necesariamente:
\begin{equation}
N_{P min}\geq A_{max}
\end{equation}
Después de un análisis, el número total de pulsos \textit{N} para completar un ciclo de conteo a la salida del bloque \textit{N} es:
\begin{equation}
N=N_P\cdot P+A
\end{equation}
Si queremos que varíe la generación de frecuencias a escalones siempre constante, entonces debe cumplirse
\begin{equation}
\underbrace{\parentesis{N_P\cdot P+A_{max}+1}_{Aumentar en 1 el valor A_{max}}=\underbrace{\parentesis{N_P+1}\cdot P+1}}_{Poner el mínimo en A(=1) y aumentar N_P en 1}
\end{equation}
Si:
\begin{itemize}
\item $A_{max}>P$: la misma frecuencia se puede generar con dos combinaciones distintas de \textit{A} y \textit{$N_P$}.
\item $A_{max}<P$: quedan frecuencias sin generar.
\end{itemize}
Por tanto, siempre:
\begin{equation}
A_{max}\geq P
\end{equation}
Los escalones de frecuencias de salida son:
\begin{equation}
\Delta f=\parentesis{N_P\cdot P+A}\cdot f_{xtal}-\parentesis{N_P\cdot P+A-1}\cdot f_{xtal}=f_{xtal}
\end{equation}
Algunos valores normalizados de P son: 5, 8, 15, 20, 32, 40 y 80.
%Falta: Ejemplo diapo 9-15 sintetizadores 1.pdf
\subsubsection{Sintetizador de frecuencia con PLLs y con mezclador}
\begin{figure}[H]
\centering
\includegraphics[width=0.7\linewidth]{mrf/mrf68.png}
\caption{Sintetizador PLL con mezclador}
\end{figure}
\begin{equation}
f_{vco}=f_{xtal1}\cdot N_P+f_{xtal2}
\end{equation}
\begin{figure}[H]
\centering
\includegraphics[width=0.7\linewidth]{mrf/mrf69.png}
\caption{Sintetizador PLL con mezclador y pll paralelo.}
\end{figure}
\begin{equation}
f_{vco1}=f_{xtal1}\cdot N_{P1}+f_{xtal2}\cdot N_{P2}
\end{equation}
\section{Receptores homodinos y heterodino}
La estructura de un receptor RF es:
\begin{figure}[H]
\centering
\includegraphics[width=\linewidth]{mrf/mrf70.png}
\caption{Diagrama de bloques de un receptor RF.}
\end{figure}
\textbf{Cualidades de un receptor}:
\begin{itemize}
\item \textbf{Sensibilidad}: capacidad de recibir señales débiles. Se mide como tensión en la entrada necesaria para obtener una relación determinada entre señal y ruido a la salida.
\item \textbf{Selectividad}: capacidad de rechazar frecuencias indeseadas. Se mide como cociente de potencias de entrada de las señales de frecuencias indeseadas y de la deseada que generan la misma señal de salida.
\item \textbf{Fidelidad}: o sensitividad, capacidad de reproducir las señales de banda base para una distorsión especificada.
\item \textbf{Margen dinámico}: cociente entre niveles máximos y mínimos de potencia de entrada que garantizan funcionamiento correcto del receptor.
\end{itemize}
\textbf{Tipos de receptores:}
\begin{itemize}
\item Homodino o de detección directa o de conversión directa
\item Reflex
\item Regenerativo o receptores de reacción
\item Superregenerativo o receptores a superacción
\item Superheterodino
\item SDR:
\end{itemize}
\subsection{Receptor homodino}
\begin{figure}[H]
\centering
\includegraphics[width=0.9\linewidth]{mrf/mrf71.png}
\caption{Receptor homodino: etapas}
\label{fig:recep homo}
\end{figure}
En la figura \ref{fig:recep homo}, como primer filtro tenemos la antena, pues solo aceptará OEM especificas, luego tenemos los filtros RF y su respectivo amplificador, aunque tenemos \textit{n} etapas, si tenemos etapas en cascada esto puede generar una reducción en el ancho de banda. Luego de las etapas, tenemos el demodulador que se encarga de eliminar la portadora y recupera la moduladora.\\
Cálculo del número de etapas en función de la frecuencia a recibir y del ancho de banda deseado:
\begin{equation}
\Delta f_o\approx \frac{\sqrt{2^{\frac{1}{n}-1}}\cdot f_o}{Q}=\frac{\sqrt{2^{\frac{1}{n}-1}}\cdot 2\pi\cdot f_o^2\cdot L}{R}
\end{equation}
Las limitaciones del receptor homodino:
\begin{itemize}
\item Necesidad de muchos filtros cuando $f_o>>\Delta f_o$ o de filtros muy agudos.
\item Muchos filtros variables si la frecuencia es variable.
\item Dificultad de mantenimiento del ancho de banda de recepción en el margen de frecuencias de recepción (selectividad variable en función de la frecuencia de recepción). Se quiere recepcionar varias frecuencias.
\end{itemize}
Un receptor homodino es útil si:
\begin{itemize}
\item El demodulador es del tipo detector \textbf{coherente}
\end{itemize}
\subsubsection*{Detector coherente}
\begin{figure}[H]
\centering
\includegraphics[width=0.8\linewidth]{mrf/mrf72.png}
\caption{Receptor homodino coherente}
\end{figure}
\begin{definition}[Receptor coherente]
La detección coherente se obtiene al multiplicar la señal modulada por un tono de la
misma frecuencia y fase que la portadora y extraer, mediante filtrado, la señal de banda
base resultante.
\end{definition}
\begin{figure}[H]
\centering
\includegraphics[width=0.7\linewidth]{mrf/mrf73.png}
\caption{Señales coherentes.}
\label{fig:señal coherente}
\end{figure}
En la figura \ref{fig:señal coherente}, se ve la señal a la entrada modulada en ASK, cuando entre al detector, un oscilador con frecuencia diferente y un voltaje diferente. Esa señal se interactuará con la señal recibida, se puede restar la \textcolor{green}{señal recibida} con la \textcolor{blue}{señal del mezclador}, y se obtiene un cambio de portadora.
\subsubsection*{Demodulación Banda base sup portadora-SSB}
Si la frecuencia del oscilador local es igual a la de la portadora recibida
%Falta 52:09
\chapterimage{chapter_head_MR.pdf} % Chapter heading image
\chapter{Mezcladores y circuitos discriminadores RF}
\section{Mezcladores}
La \textbf{idea fundamental} es obtener una señal cuya frecuencia se la \textbf{suma o diferencia} de la frecuencia de otras dos:
\begin{figure}[H]
\centering
\includegraphics[width=0.6\linewidth]{mrf/mrf74.png}
\caption{Idea fundamental del mezclador}
\end{figure}
\begin{figure}[H]
\centering
\includegraphics[width=\linewidth]{mrf/mrf75.png}
\caption{Mezclador ideal}
\end{figure}
\subsubsection*{Cómo generar una señal con frecuencias ($f_1+f_2$) y |$f_1-f_2$|}
Para lograr eso nos debemos basar en las identidades trigonométricas:
\begin{theorem}[Coseno de suma]
\begin{align}
\cos(A+B)&=\cos A\cdot\cos B-\sin A\cdot\sin B\\
\cos(A-B)&=\cos A\cdot\cos B+\sin A\cdot\sin B
\end{align}
Acomodando:
\begin{align}
\cos A\cdot\cos B&=0.5\corchetes{\cos(A+B)+\cos(A-B)}\label{eq:coscos}\\
\sin A\cdot\sin B&=0.5\corchetes{\cos(A-B)-\cos(A+B)}\label{eq:sensen}
\end{align}
\end{theorem}
\begin{theorem}[Seno de suma]
\begin{align}
\sin(A+B)&=\sin A\cdot\cos B+\sin B\cdot\cos A\\
\sin(A-B)&=\sin A\cdot\cos B-\sin B\cdot\cos A
\end{align}
Acomodando:
\begin{align}
\sin A\cdot\cos B&=0.5\corchetes{\sin(A+B)+\sin(A-B)}\label{eq:sencon}\\
\sin B\cdot\cos A&=0.5\corchetes{\sin(A+B)-\sin(A-B)}\label{eq:cossen}
\end{align}
\end{theorem}
\begin{theorem}[Coseno del ángulo doble]
\begin{align}
\cos(2A)&=\cos^2A-\sin^2A\\
1&=\cos^2A+\sin^2A
\end{align}
Luego
\begin{align}
\cos^2A&=0.5\corchetes{1+\cos(2A)}\label{eq:coscuadrado}\\
\sin^2A&=0.5\corchetes{1-\cos(2A)}\label{eq:sincuadrado}
\end{align}
\end{theorem}
Notando la expresión \ref{eq:coscos}, vemos que tenemos las componentes frecuenciales que hemos planteado en la idea fundamental.\\
¿Qué pasa si las señales que se mezclan no están en fase? Si introducimos señales que no están en fase en la expresión \ref{eq:coscos}:
\begin{displaymath}
\cos\omega_1t\cdot\cos(\omega_wt+\phi)=0.5\cos\parentesis{(\omega_1+\omega_2)t+\phi}+0.5\cos\parentesis{(\omega_1-\omega_2)t-\phi}
\end{displaymath}
Tenemos las componentes deseadas y tenemos el desfase, el desfase sólo provoca desfases, NO nuevas componentes.
\textbf{¿Cómo multiplicar dos señales?}\\
Usando un multiplicador analógico clásico, el cual NO es adecuado para alta frecuencia.
\subsubsection*{Dispositivos de respuesta cuadrática}
\begin{figure}[H]
\centering
\includegraphics[width=0.5\linewidth]{Mrf/mrf76.png}
\caption{Dispositivo de respuesta cuadrática.}
\end{figure}
\begin{align*}
v_s&=V_0+k\cdot\parentesis{V_1\cos\omega_1t+V_2\cos\omega_2t}^2\\
&=V_0+k\parentesis{V_1^2\cos^2\omega_1t+V_2^2\cos^2\omega_2^t+2V_1\cos\omega_1t\cdot V_2\cos\omega_2t}
\end{align*}
Usando la identidad :
\begin{align*}
v_s=&\underbrace{V_0+0.5k\cdot V_1^2+0.5k\cdot V_2^2}_{Componente continua}+\underbrace{0.5k\cdot V_1^2\cos\parentesis{2\omega_1t}}_{Componentes de frecuencia 2f_1}+\underbrace{0.5k\cdot V_2^2\cos\parentesis{2\omega_2t}}_{Componentes de frecuencia 2f_2}+\\
&\underbrace{k\cdot V_1V_2\cos\parentesis{\omega_1+\omega_2}t}_{Componentes de frecuencia f_1+f_2}+\underbrace{k\cdot V_1V_2\cos\parentesis{\omega_1-\omega_2}t}_{Componente de frecuencia |f_1-f_2|}
\end{align*}
\begin{corollary}
Nos sobran las componentes de continua y de frecuencia $2f_1$ y $2f_2$.
\end{corollary}
\begin{figure}[]
\centering
\subfloat[Señales en el tiempo]{\includegraphics[width=0.4\linewidth]{mrf/mrf78.png}}
\subfloat[Respuesta en frecuencia]{\includegraphics[width=0.4\linewidth]{mrf/mrf79.png}}
\caption{Dispositivo cuadrático: $V_0=0$, $V_1=V_2$ y $k=0.5$.}
\end{figure}
Es más dificil filtrar el caso real (cuadrático) para aislar una única frecuencia.
\subsubsection*{Dispositivos de respuesta proporcional más una cuadrática}
\begin{figure}[H]
\centering
\includegraphics[width=0.5\linewidth]{mrf/mrf77.png}
\caption{Respuesta proporcional más cuadrática.}
\label{fig:proporcional cuadratica}
\end{figure}
\begin{align*}
v_s&=V_0+K_A\cdot\parentesis{V_1\cos\omega_1t+V_2\cos\omega_2t}+k_B\cdot\parentesis{V_1\cos\omega_1t+V_2\cos\omega_2t}^2\\
&=V_0+k_A\cdot\parentesis{V_1\cos\omega_1t+V_2\cos\omega_2t}+k_B\cdot\parentesis{V_1^2\cos^2\omega_1t+V_2^2\cos^2\omega_2t+2V_1\cos\omega_1tV_2\cos\omega_2t}
\end{align*}
Usando 
\begin{align*}
v_s=&\underbrace{V_0+0.5k_B\cdot V_1^2+0.5k_B\cdot V_2^2}_{Componente de continua}+\underbrace{k_A\cdot V_1\cos\omega_1t}_{Componente de frecuencia f_1}+\underbrace{k_A\cdot V_2\cos\omega_2t}_{Componente de frecuencia f_2}+\\
&\underbrace{0.5k_B\cdot V_1^2\cos(2\omega_1t)}_{Componente de frecuencia 2f_1}+\underbrace{0.5k_B\cdot V_2^2\cos\parentesis{2\omega_2t}}_{Componente de frecuencia 2f_2}\\
&+\underbrace{k_B\cdot V_1V_2\cos\parentesis{\omega_1+\omega_2}t}_{Componente de frecuencia f_1+f_2}+\underbrace{k_B\cdot V_1V_2\cos(\omega_1-\omega_2)t}_{Componente de frecuencia |f_1-f_2|}
\end{align*}
\begin{corollary}
Nos sobran las componentes de continua y de frecuencia $f_1$, $f_2$, $2f_1$ y $2f_2$.
\end{corollary}
\begin{figure}[]
\centering
\subfloat[Señales en el tiempo]{\includegraphics[width=0.4\linewidth]{mrf/mrf80.png}}
\subfloat[Respuesta en frecuencia]{\includegraphics[width=0.4\linewidth]{mrf/mrf81.png}}
\caption{Dispositivo proporcional más cuadrático: $V_0=0$, $V_1=V_2$, $k_A=0.25$ y $k_B=0.5$}
\end{figure}
Más difícil de filtrar para aislar una única frecuencia.
\subsubsection*{Dispositivos de respuesta no lineal (general)}
%PDF pag 8
Es importante que nuestro mezclado genere el mínimo número posibles de mezcla porque de esta manera el filtrado se facilita.\\
\textbf{Componentes de frecuencias:}
\begin{itemize}
\item \textbf{Mezclador ideal}: $(f_{1A}+f_2), (f_{1B}+f_2), |f_{1A}+f_2|, |f_{1A}-f_2|, |f_{1B}-f_2|$.
\item \textbf{Mezclador cuadrático}: $0, (f_{1A}+f_2), (f_{1B}+f_2), |f_{1A}-f_2|, |f_{1B}-f_2|, 2f_{1A}, 2f_{1B}, 2f_2$
\item \textbf{Mezclador proporcional + cuadrático}: $0, (f_{1A}+f_2), (f_{1B}+f_2), |f_{1A}-f_2|, |f_{1B}-f_2|, f_{1A}, f_{1B}, f_2, 2f_{1A}, 2f_{1B}, 2f_2$
\end{itemize}
\subsection{Objetivo de la realización física}
\begin{itemize}
\item Comportamiento adecuado a las frecuencias de trabajo
\item Uso de dispositivos con comportamiento lo más parecido a cuadrático, sin términos apreciables en x, $x^3$, $x^4$, etc.
\item Cancelación de componentes indeseadas por simetrías en los circuitos. 
\end{itemize}
\begin{center}
\schema
{
	\schemabox{Tipos de mezcladores}
}
{
	\schema
		{
			\schemabox{Pasivos (diodos)}
		}
		{
			\schemabox{Simples \\ Equilibrados \\ Doblemente equilibrados}}
	\schema
		{
			\schemabox{Activos (transistores)}
		}
		{
			\schemabox{Simples \\ Equilibrados \\ Doblemente equilibrados}}

} 
\end{center}
\subsection{Mezcladores con diodos}
\begin{figure}[H]
\centering
\includegraphics[width=0.4\linewidth]{mrf/mrf82.png}
\caption{Diodo}
\end{figure}
Diodo cuya curva de respuesta:
\begin{equation}
I_D=I_S\cdot\parentesis{e^{\frac{V_D}{V_T}}-1}
\label{eq:diodo}
\end{equation}
Con los valores de $I_S=1\mu A$ y $V_T=26 mV$ y graficando la respuesta de la ecuación \ref{eq:diodo} (\textcolor{blue}{Modelo exponencial}):
\begin{figure}[H]
\centering
\includegraphics[width=0.7\linewidth]{mrf/mrf83.png}
\caption{Gráfica del diodo y modelo exponencial}
\label{fig:grafica diodo}
\end{figure}
Si escribimos la ecuación de la figura \ref{fig:proporcional cuadratica} en función de $V_D$ y ajustando valores obtenemos \textcolor{red}{Modelo proporcional + cuadrático} en la figura \ref{fig:grafica diodo}.
\begin{equation}
I_D=K_A\cdot V_D+K_B\cdot V_D^2
\label{eq:diodo aprox}
\end{equation}
Con $K_A=4.467\basedec{-5}$ y $K_B=7.984\basedec{-4}$\\
Se tiene un error en un margen de tensiones de $\pm 30 mV$. Solo se tomará el valor donde se aprecia un semejanza notoria, pues si lo extendemos mucho el comportamiento es diferente y ya tendría un comportamiento más complejo y con ello generando otras componentes de frecuencia:
\begin{figure}[H]
\centering
\includegraphics[width=0.7\linewidth]{mrf/mrf84.png}
\caption{Comportamiento distinto en dominio más extenso.}
\end{figure}
\subsubsection{Teoría mezclador con un diodo}
\begin{figure}[H]
\centering
\subfloat[Idea general]{\includegraphics[width=0.4\linewidth]{mrf/mrf85.png}}
\subfloat[Realización práctica sin terminal común en las fuentes]
{\includegraphics[width=0.4\linewidth]{mrf/mrf86.png}\label{fig:diodo reali}}
\caption{Mezclador con un diodo}
\end{figure}
De la figura \ref{fig:diodo reali} se puede escribir que:
\begin{equation}
V_S+V_D=V_1+V_2
\end{equation}
Donde
\begin{equation}
V_S=R\cdot I_D
\label{eq:voltaje salida diodo}
\end{equation}
Asumiremos que:
\begin{equation}
V_S<<V_D, V_1, V_2
\end{equation}
Por lo tanto
\begin{equation}
V_D\approx V_1+V_2
\end{equation}
Y usando la ecuación \ref{eq:diodo aprox} para reemplazar en \ref{eq:voltaje salida diodo}:
\begin{equation}
    \begin{split}
        V_S&\approx R[5k_BV_1^2+0.5k_BV_2^2+k_AV_1\cos\omega_1t+k_AV_2\cos\omega_2t\\
&+0.5k_BV_1^2\cos(2\omega_1t)+0.5k_BV_2^2\cos(2\omega_2t)+k_BV_1V_2\cos(\omega_1+\omega_2)t\\
&+k_BV_1V_2\cos(\omega_1-\omega_2)t]
    \end{split}
    \label{eq:Vs mezclador 1 diodo}
\end{equation}
\begin{notation}
Nos sobran componentes de continua y de frecuencias $f_1$, $f_2$, $2f_1$, $2f_2$.
\end{notation}
Acomodando el circuito de la figura \ref{fig:diodo reali}, colocando las fuentes a tierra y haciendo uso de un transformador para que de esta manera ambas fuentes estén en tierra sin ningún tipo de discriminación, la otra forma de colocar ambas fuentes es en \textbf{paralelo}, sin embargo aquí ya se habla de impedancias; para que este arreglo funcione es necesario que ambas fuentes tengan la misma impedancia para que ni una fuente se comporte como carga, al contrario, deben mirar ambas a \textit{R} como la carga del sistema.
\begin{figure}[H]
\centering
\subfloat[Realización práctica con terminal común en las fuentes y la carga]{\includegraphics[width=0.4\linewidth]{mrf/mrf87.png}}
\subfloat[Realización práctica sin transformador y con terminal común en las fuentes y la carga]{\includegraphics[width=0.4\linewidth]{mrf/mrf88.png}}
\caption{Asociación de fuentes con tierra común.}
\end{figure}
\subsubsection{Teoría del mezclador equilibrado con dos diodos}
\begin{figure}[H]
\centering
\includegraphics[width=0.6\linewidth]{mrf/mrf89.png}
\caption{Mezclador equilibrado con dos diodos.}
\label{fig:mezc equi 2 diodos}
\end{figure}
Dividiendo las dos mallas del circuito de la figura \ref{fig:mezc equi 2 diodos} y ayudándonos del análisis de la ecuación \ref{eq:Vs mezclador 1 diodo}:
\begin{equation}
    \begin{split}
        V_{S1}&\approx R[5k_BV_1^2+0.5k_BV_2^2+k_AV_1\cos\omega_1t+k_AV_2\cos\omega_2t\\
&+0.5k_BV_1^2\cos(2\omega_1t)+0.5k_BV_2^2\cos(2\omega_2t)+k_BV_1V_2\cos(\omega_1+\omega_2)t\\
&+k_BV_1V_2\cos(\omega_1-\omega_2)t]
    \end{split}
    \label{eq:Vs1 mezclador 2 diodos}
\end{equation}
\begin{equation}
    \begin{split}
        V_{S2}&\approx R[5k_BV_1^2+0.5k_BV_2^2-k_AV_1\cos\omega_1t+k_AV_2\cos\omega_2t\\
&+0.5k_BV_1^2\cos(2\omega_1t)+0.5k_BV_2^2\cos(2\omega_2t)-k_BV_1V_2\cos(\omega_1+\omega_2)t\\
&-k_BV_1V_2\cos(\omega_1-\omega_2)t]
    \end{split}
    \label{eq:Vs2 mezclador 2 diodos}
\end{equation}
Realizando la suma algebraica para hallar $V_S$ usando las expresiones \ref{eq:Vs1 mezclador 2 diodos} y \ref{eq:Vs2 mezclador 2 diodos}:
\begin{equation}
V_S=V_{S1}-V_{S2}=2R\corchetes{k_AV_1\cos\omega_1t+k_BV_1V_2\cos(\omega_!+\omega_2)t+k_BV_1V_2\cos(\omega_1-\omega_2)t}
\end{equation}
\begin{notation}
Sólo nos sobra la componente de frecuencia $f_1$ ($k_AV_1\cos\omega_1t$).
\end{notation}
\begin{figure}[H]
\centering
\includegraphics[width=0.7\linewidth]{mrf/mrf90.png}
\caption{Se puede lograr el mismo circuito usando transformadores para que las fuentes estén a tierra.}
\end{figure}
\subsection{Mezclador doblemente equilibrado con cuatro diodos}
\begin{figure}[H]
\centering
\subfloat[Mezclador doblemente equilibrado.]{\includegraphics[width=0.4\linewidth]{mrf/mrf91.png}\label{fig:mezclador doble equi}}
\subfloat[Entrada]{\includegraphics[width=0.4\linewidth]{mrf/mrf95.png}}
\caption{Mezclado doblemente equilibrado}
\end{figure}
Analizando la figura \ref{fig:mezclador doble equi}, y recordando la ecuación \ref{eq:diodo aprox} que depende de $V_D$, se pueden escribir las ecuaciones de los demás diodos como:
\begin{align}
I_{D1}\approx f(V_1+V_2)\\
I_{D2}\approx f(-V_1+V_2)\\
I_{D3}\approx f(V_1-V_2)\\
I_{D4}\approx f(-V_1-V_2)\\
\end{align}
Por lo tanto:
\begin{equation}
V_S=V_{13}-V_{24}=I_{13}R-I_{24}R=R\corchetes{I_{D1}-I_{D3}-(I_{D2}-I_{D4})}
\end{equation}
\begin{equation}
\therefore V_S\approx R\corchetes{f(V_1+V_2)-f(V_1-V_2)-f(-V_1+V_2)+f(-V_1-V_2)}
\end{equation}
Si analizamos cada una de las componentes, notaremos que se cancelan varias componentes, si sumamos lo restante se obtendrá:
\begin{equation}
V_S\approx 4Rk_B\corchetes{V_1V_2\cos(\omega_1+\omega_1)t+V_1V_2\cos(\omega_1-\omega_2)t}
\end{equation}
\begin{figure}[]
\centering
\subfloat[Realización practica]{\includegraphics[width=0.4\linewidth]{mrf/mrf92.png}}
\subfloat[Anillo de diodos]{\includegraphics[width=0.4\linewidth]{mrf/mrf93.png}}\\
\subfloat[Anillo de diodos 2.]{\includegraphics[width=0.4\linewidth]{mrf/mrf94.png}}
\caption{Realización práctica}
\end{figure}
\subsection{Carga a la salida de mezclador de diodos}
\begin{wrapfigure}{r}{0.5\linewidth}
  \begin{center}
    \includegraphics[width=.95\linewidth]{mrf/mrf96.png}
  \end{center}
  \caption{}
\end{wrapfigure}
Hasta ahora hemos tratado que un mezclador tiene una carga resistiva, cuando lo usual es conectar un filtro a la salida. Dependiendo de la frecuencia que queramos se configura el filtro, sea pasa-bajos o pasa-altos. La impedancia del filtro ($Z_f$) no va ser resistiva, sino que va depender de la frecuencia. Es por ello que se busca un tipo de filtro con $Z_f$ independiente de la frecuencia. Para ello se puede usar un \textbf{diplexor}.
\begin{figure}[H]
\centering
\includegraphics[width=0.4\linewidth]{mrf/mrf97.png}
\caption{Diplexor}
\end{figure}
Las funciones de transferencias son:
\begin{align}
\frac{V_{S1}}{V_e}&=\frac{1}{LCs^2+\sqrt{2LC}s+1}\\
\frac{V_{S2}}{V_e}&=\frac{LCs^2}{LCs^2+\sqrt{2LC}s+1}
\end{align}
\begin{figure}[H]
\centering
\includegraphics[width=0.4\linewidth]{mrf/mrf98.png}
\caption{Frecuencia de corte de las funciones de transferencia.}
\end{figure}
Donde:
\begin{itemize}
\item $f_c=\sqrt{f_{sum}\cdot f_{dif}}$
\item $f_c=\frac{1}{2\pi\sqrt{LC}}$
\item $L/C=2R^2$
\end{itemize}
\begin{notation}
Donde $f_{sum}$ y $f_{dif}$ representa la suma y la diferencia de las frecuencias.
\end{notation}
\subsection{Mezclados con un transistor bipolar}
\begin{figure}[H]
\centering
\subfloat[Idea general]{\includegraphics[width=0.3\linewidth]{mrf/mrf99.png}}
\subfloat[Realización son terminal común en las fuentes]{\includegraphics[width=0.3\linewidth]{mrf/mrf100.png}}
\caption{Mezclador con transistor}
\end{figure}
Se realiza con las ecuaciones:
\begin{equation}
V_{BE}=v_1+v_2
\end{equation}
\begin{equation}
v_s=R\cdot i_c
\end{equation}
\begin{equation}
I_{c(v_{EB}=0)}=-I_{sc}
\end{equation}
\begin{equation}
i_c\approx I_{sc}+k_A\cdot v_{BE}+k_B\cdot v_{BE}^2
\end{equation}
\begin{align}
v_s&\approx R[I_{sc}+0.5k_BV_1^2+0.5k_BV_2^2+k_AV_1\cos\omega_1t+k_AV_2\cos\omega_2t+\\
&0.5k_BV_1^2\cos(2\omega_1t)+0.5k_BV_2^2\cos(2\omega_2t)+k_BV_1V_2\cos(\omega_1+\omega_2)t+\\
&k_BV_1V_2\cos(\omega_1-\omega_2)t]
\end{align}
\begin{notation}
Nos sobran las componentes de continua y de frecuencia $f_1$, $f_2$, $2f_1$ y $2f_2$.
\end{notation}
Se puede conseguir cancelación de componentes indeseadas por simetrías.
\begin{figure}[H]
\centering
\includegraphics[width=0.7\linewidth]{mrf/mrf101.png}
\caption{Mezclador con varios transistores bipolares}
\end{figure}
En la cual solo nos sobra componentes de frecuencia $f_1$, donde se cancelaron $f_2$, $2f_1$ y $2f_2$.
%----------------------------------------------------------------------------------------
%	ANEXOS
%----------------------------------------------------------------------------------------
\part{Anexos}
\section{Constantes universales}
\begin{itemize}
\item $\mu_0$: Permeabilidad en el vacío.
\begin{displaymath}
\mu_0=1.2566\cdot\basedec{-6} N/A^2
\end{displaymath}
\item $\epsilon_0$: Permitividad en el vacío:
\begin{displaymath}
\epsilon_0=8.854\cdot\basedec{-12} F/m
\end{displaymath}
\end{itemize}
\section{Voltajes}
\begin{center}
%\includegraphics[width=0.7\linewidth]{Electro/Electro5.png}
\end{center}
\begin{equation}
V_{rms}=\frac{1}{\sqrt{2}}\cdot V_p=0.7071\cdot V_p
\label{eq:vrms}
\end{equation}
\begin{equation}
V_{pr}=0.636\cdot V_p
\label{eq:vprom}
\end{equation}
%---------------------------------------------------------------
% 		ENDING
%---------------------------------------------------------------
\stopcontents[part] % Manually stop the 'part' table of contents here so the previous Part page table of contents doesn't list the following chapters

%----------------------------------------------------------------------------------------
%	BIBLIOGRAPHY
%----------------------------------------------------------------------------------------

\chapterimage{Final2.jpg} % Chapter heading image
\chapterspaceabove{6.75cm} % Whitespace from the top of the page to the chapter title on chapter pages
\chapterspacebelow{7.25cm} % Amount of vertical whitespace from the top margin to the start of the text on chapter pages

%------------------------------------------------
\chapter*{Bibliography}
\markboth{\sffamily\normalsize\bfseries Bibliography}{\sffamily\normalsize\bfseries Bibliography} % Set the page headers to display a Bibliography chapter name
\addcontentsline{toc}{chapter}{\textcolor{ocre}{Bibliography}} % Add a Bibliography heading to the table of contents

\section*{Articles}
\addcontentsline{toc}{section}{Articles} % Add the Articles subheading to the table of contents

%\printbibliography[heading=bibempty,type=article] % Output article bibliography entries

\section*{Books}
\addcontentsline{toc}{section}{Books} % Add the Books subheading to the table of contents

\printbibliography[heading=bibempty,type=book] % Output book bibliography entries
%----------------------------------------------------------------------------------------
%	APPENDICES
%----------------------------------------------------------------------------------------
%
%\chapterimage{Final1.jpg} % Chapter heading image
%\chapterspaceabove{6.75cm} % Whitespace from the top of the page to the chapter title on chapter pages
%\chapterspacebelow{7.25cm} % Amount of vertical whitespace from the top margin to the start of the text on chapter pages
%
%\begin{appendices}
%
%\renewcommand{\chaptername}{Appendix} % Change the chapter name to Appendix, i.e. "Appendix A: Title", instead of "Chapter A: Title" in the headers
%
%%------------------------------------------------
%
%\chapter{Appendix Chapter Title}
%
%\section{Appendix Section Title}
%
%Lorem ipsum dolor sit amet, consectetur adipiscing elit. Aliquam auctor mi risus, quis tempor libero hendrerit at. Duis hendrerit placerat quam et semper. Nam ultricies metus vehicula arcu viverra, vel ullamcorper justo elementum. Pellentesque vel mi ac lectus cursus posuere et nec ex. Fusce quis mauris egestas lacus commodo venenatis. Ut at arcu lectus. Donec et urna nunc. Morbi eu nisl cursus sapien eleifend tincidunt quis quis est. Donec ut orci ex. Praesent ligula enim, ullamcorper non lorem a, ultrices volutpat dolor. Nullam at imperdiet urna. Pellentesque nec velit eget est euismod pretium.
%%------------------------------------------------
%\end{appendices}
%----------------------------------------------------------------------------------------
%	INDEX
%----------------------------------------------------------------------------------------

\cleardoublepage % Make sure the index starts on an odd (right side) page
\phantomsection
\setlength{\columnsep}{0.75cm} % Space between the 2 columns of the index
\addcontentsline{toc}{chapter}{\textcolor{ocre}{Index}} % Add an Index heading to the table of contents
\printindex % Output the index
%----------------------------------------------------------------------------------------
%	CHAPTER
%----------------------------------------------------------------------------------------

\end{document}
